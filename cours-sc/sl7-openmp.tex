%
% Historique
%   2014/08/30 : pda : conception
%

\newcounter {ompexemple}
\def\inc{inc7-openmp}

\titreA {Introduction à OpenMP}

%%%%%%%%%%%%%%%%%%%%%%%%%%%%%%%%%%%%%%%%%%%%%%%%%%%%%%%%%%%%%%%%%%%%%%%%%%%%%%
% Introduction
%%%%%%%%%%%%%%%%%%%%%%%%%%%%%%%%%%%%%%%%%%%%%%%%%%%%%%%%%%%%%%%%%%%%%%%%%%%%%%

\titreB {Introduction}

\begin {frame} {Introduction}
    Objectifs~:
    
    \begin {itemize}
	\item faciliter la programmation parallèle...
	\item ...  sur les architectures à mémoire partagée
	\item supporter C, C++ et Fortran
	\item permettre la maintenance d'un programme unique pour
	    les versions séquentielle et parallèle
	\item permettre la portabilité \implique standard de-facto
	\item large support des industriels
	\item attractif pour les utilisateurs
    \end {itemize}

\end {frame}

\begin {frame} {Historique}

    \begin {itemize}
	\item milieu des années 1990 :
	    \begin {itemize}
		\item retour en grâce des architectures à mémoire partagée
		\item programmation par passage de message fastidieuse
	    \end {itemize}
	\item 1997 : première spécification d'OpenMP pour Fortran
	\item 1998 : première spécification d'OpenMP pour C/C++
	\item 2000 : OpenMP 2.0 pour Fortran
	\item 2002 : OpenMP 2.0 pour C/C++
	\item 2005 : OpenMP 2.5 unifié pour Fortran et C/C++
	\item 2008 : OpenMP 3.0 unifié pour Fortran et C/C++
	\item 2013 : OpenMP 4.0 unifié pour Fortran et C/C++
    \end {itemize}

    \implique OpenMP est très dynamique \\
    (dans cette introduction, on se limitera à OpenMP 1.0)

\end {frame}

\begin {frame} {Structuration d'OpenMP}
    \begin {center}
	\includegraphics [width=.9\textwidth] {\inc/couches}
    \end {center}

    \begin {itemize}
	\item programme rédigé avec des directives \texttt {\#pragma}
	    et des appels à la bibliothèque OpenMP
	\item compilation (\texttt {-fopenmp}) : utilise
	    le «~\emph {run-time}~» OpenMP

	\item «~\emph {run-time}~» OpenMP s'appuie sur les Threads
	    POSIX
	\item comportement du programme modifiable dynamiquement
	    avec des variables d'environnement

    \end {itemize}

\end {frame}

\begin {frame} {Directives \texttt {\#pragma}}
    Directives \texttt {\#pragma} : moyen normalisé d'introduire des
    directives spécifiques au compilateur C ou des extensions du langage

    \vspace* {3mm}

    Exemples~:
    \begin {itemize}
	\item \texttt {\#pragma message "Je compile " \_\_FILE\_\_}

	    \begin {itemize}
		\item «~\texttt {\#pragma message}~» : reconnu
		    par GCC et CLANG
		\item affiche un message lors de la compilation
	    \end {itemize}

	\item \texttt {\#pragma omp parallel private (i)}

	    \begin {itemize}
		\item
		    les directives OpenMP commencent par «~\texttt
		    {\#pragma omp}~»
		    \\
		    \implique attention de ne pas oublier \texttt {omp}
		    (sinon \texttt {\#pragma} ignoré)

		\item
		    la directive \texttt {parallel} admet des clauses
		    (ex: \texttt {private (i)})

	    \end {itemize}

    \end {itemize}

\end {frame}

\begin {frame} {Compilation}
    Compilation~:

    \begin {itemize}
	\item inclusion des définitions de prototypes, de macros, etc~:

	    \hspace* {5mm} \code {\#include "omp.h"}

	\item compilation avec prise en compte des \code {\#pragma}
	
	    \hspace* {5mm} \texttt {gcc -fopenmp hello.c}
    \end {itemize}

    Mise au point~:
    \begin {itemize}
	\item plus facile de déboguer la version séquentielle
	\item compilation sans \texttt {-fopenmp}
	    \begin {itemize}
		\item pas de prise en compte des \code {\#pragma omp}
		\item pas de région parallèle, pas de multiples threads
		\item si utilisation de fonctions de la bibliothèque OpenMP \\
		    \implique édition de liens avec la bibliothèque OpenMP

		    Exemple (spécifique \texttt {gcc})~:
			\texttt {gcc .... -lgomp}

	    \end {itemize}
    \end {itemize}
\end {frame}


%%%%%%%%%%%%%%%%%%%%%%%%%%%%%%%%%%%%%%%%%%%%%%%%%%%%%%%%%%%%%%%%%%%%%%%%%%%%%%
% Régions parallèles
%%%%%%%%%%%%%%%%%%%%%%%%%%%%%%%%%%%%%%%%%%%%%%%%%%%%%%%%%%%%%%%%%%%%%%%%%%%%%%

\titreB {Régions parallèles}

\begin {frame} {Régions parallèles}
    Région parallèle = brique de base d'OpenMP

    \begin {center}
	\includegraphics [width=.9\textwidth] {\inc/fork-join}
    \end {center}

    \begin {itemize}
	\item le thread principal génère des threads (ou réactive
	    des threads existants) lorsqu'une région parallèle est
	    spécifiée

	\item synchronisation implicite (barrière) en fin de région

	\item les régions parallèles peuvent être imbriquées

    \end {itemize}

    \vspace* {2mm}

    Note : en réalité, le thread maître fait partie des threads
    (travailleurs) de la région parallèle

\end {frame}

\refstepcounter {ompexemple}

\begin {frame} [fragile] {Exemple \theompexemple{} -- Hello world}
    \lstinputlisting [basicstyle=\scriptsize\lstmonstyle, firstline=3] {\inc/hello.c}

    \begin {itemize}
	\item \code {\#pragma omp parallel} crée une région parallèle
	\item utilise un nombre de threads «~optimum~»
	\item \code {omp\_get\_num\_threads} : nombre de threads
	    créés \\
	    \implique 1 avant la région parallèle
	\item \code {omp\_get\_thread\_num} : numéro du thread
	    courant
    \end {itemize}

\end{frame}

\begin {frame} {Exemple \theompexemple{} -- Hello world}

    \begin {itemize}
	\item \code {private}~: variables placées dans l'espace du
	    thread au début de la région parallèle et détruites à
	    la fin de la région

	    \begin {itemize}
		\item \code {i} et \code {j} : valeurs différentes
		    dans chaque thread
	    \end {itemize}

	\item \code {shared}~: variables partagées par tous les
	    threads

	    \begin {itemize}
		\item \code {pid} : valeur constante
		\item \code {n} : modifiée (globalement)
		    par chaque thread
		\item attention aux modifications non atomiques !
	    \end {itemize}

    \end {itemize}

    \begin {center}
	\includegraphics [width=.8\textwidth] {\inc/hello}
    \end {center}

\end {frame}

\begin {frame} {Exemple \theompexemple{} -- Hello world}
    Choix du nombre de threads :

    \begin {itemize}
	\item par défaut : nombre de threads « optimum »
	    \\
	    \implique choisi par OpenMP

	\item variable d'environnement \code {OMP\_NUM\_THREADS}
	    \\
	    \implique valeur forcée par l'utilisateur du programme

	    \code {OMP\_NUM\_THREADS=5} \\
	    \code {export OMP\_NUM\_THREADS}

	\item \code {\#pragma omp parallel ... num\_threads (5)}
	    \\
	    \implique forcé par le programmeur dans le programme
	    \\
	    \implique prioritaire sur \code {OMP\_NUM\_THREADS}

    \end {itemize}
\end {frame}

\refstepcounter {ompexemple}

\begin {frame} [fragile] {Exemple \theompexemple{} -- Réduction}
    \lstinputlisting [basicstyle=\scriptsize\lstmonstyle, firstline=2] {\inc/reduc.c}

    \begin {itemize}
	\item directive \code {\#pragma} sur plusieurs lignes

	\item clause \code {reduction (+:s)}

	    \begin {itemize}
		\item un opérateur (\code {+})
		\item une ou plusieurs variables (\code {s})
	    \end {itemize}
    \end {itemize}
\end{frame}

\begin {frame} {Exemple \theompexemple{} -- Réduction}

    \begin {itemize}
	\item réduction : applique un opérateur sur une valeur
	    calculée par chaque thread

	\item opérateurs : opérateurs associatifs (\code {+},
	    \code {-}, \code {*} \code {\&}, \code {|}, \code
	    {\^~}, \code {\&\&} et \code {||}, mais pas \code {/})

    \end {itemize}

    \begin {center}
	\includegraphics [width=.8\textwidth] {\inc/reduc}
    \end {center}

\end {frame}

\begin {frame} {Autres clauses des régions parallèles}
    \begin {itemize}
	\item \code {\#pragma omp parallel ... if (...)}

	    \begin {itemize}
		\item expression du \code {if} vraie,
		    \implique région exécutée en parallèle
		\item si non \implique  région exécutée en séquentiel
	    \end {itemize}

	\item \code {\#pragma omp parallel ... default (none)}

	    \begin {itemize}

		\item \code {default (shared)}~: par défaut, toutes
		    les variables sont partagées
		    (\implique comportement implicite)

		\item \code {default (none)}~: rien n'est défini
		    par défaut, il faut spécifier explicitement toutes
		    les variables
		    (\implique conseillé)

	    \end {itemize}

	\item \code {\#pragma omp parallel ... firstprivate (...)}

	    \begin {itemize}
		\item idem \code {private}, sauf...
		\item ... la copie de chaque variable est initialisée avec
		    la valeur de la variable avant la
		    région parallèle
	    \end {itemize}

	\item \code {\#pragma omp parallel ... copyin (...)}

	    \begin {itemize}
		\item voir plus loin (partage des données,
		    \code {threadprivate})
	    \end {itemize}

    \end {itemize}

\end {frame}



%%%%%%%%%%%%%%%%%%%%%%%%%%%%%%%%%%%%%%%%%%%%%%%%%%%%%%%%%%%%%%%%%%%%%%%%%%%%%%
% Partage du travail
%%%%%%%%%%%%%%%%%%%%%%%%%%%%%%%%%%%%%%%%%%%%%%%%%%%%%%%%%%%%%%%%%%%%%%%%%%%%%%

\titreB {Partage du travail}

\begin {frame} {Partage du travail}
    
    Objectif : répartir le travail à réaliser entre les différents
    threads d'une section parallèle

    \begin {itemize}
	\item \code {\#pragma omp for}

	    Répartit les itérations d'une boucle sur les différents
	    threads

	\item \code {\#pragma omp sections}

	    Répartit des codes indépendants (sections) sur les
	    différents threads

	\item \code {\#pragma omp single}

	    Spécifie un code devant être exécuté par un thread unique

    \end {itemize}
\end {frame}

\refstepcounter {ompexemple}

\begin {frame} [fragile] {Exemple \theompexemple{} -- Boucle parallèle}
    \lstinputlisting [basicstyle=\scriptsize\lstmonstyle, firstline=4] {\inc/parfor.c}

    \begin {itemize}
	\item la directive OpenMP \code {\#pragma omp for} suppose
	    une boucle C associée (ligne suivante)

	\item \code {omp\_get\_wtime} : heure actuelle (en s)
	    par rapport à une référence stable pendant la durée
	    du programme

    \end {itemize}
\end{frame}

\begin {frame} {Exemple \theompexemple{} -- Boucle parallèle}
    \begin {center}
	\includegraphics [width=.8\textwidth] {\inc/parfor}
    \end {center}

    \begin {itemize}
	\item restrictions sur la boucle \code {for(}\emph {init}\code {;}\emph {test}\code {;}\emph {incr}\code {)}~:
	    \begin {itemize}
		\item \emph {init}~: de la forme \emph {var}\code {=}\emph {expr}

		\item \emph {test}~: de la forme \emph {var op expr} (\emph {op}
		    $\in \{$ \code {<}, \code {<=}, \code {>},
		    \code {>=} $\}$)

		\item \emph {incr}~: de la forme \emph {var}\code {=}\emph
		    {var}\code {+}/\code {-}\emph {expr} (ou
		    \emph {var}\code{++}, \emph {var}\code {+=}...)

		\item la variable \emph {var} est implicitement privée

		\item \emph {expr} : expression invariante, sans effet
		    de bord

	    \end {itemize}

	\item simplification de l'écriture~: contraction des directives
	    \code {parallel} et \code {for} en~:

	    \hspace* {5mm} \code {\#pragma omp parallel for \emph {clauses}...}

    \end {itemize}
\end {frame}

\begin {frame} {Boucle parallèle -- Clauses 1/2}
    La boucle parallèle admet les clauses suivantes~:
    \begin {itemize}
	\item les clauses des régions parallèles~:

	    \begin {itemize}
		\item \code {\#pragma omp for ... private (...)}
		\item \code {\#pragma omp for ... firstprivate (...)}
		\item \code {\#pragma omp for ... reduction (...)}
	    \end {itemize}

	\item \code {\#pragma omp for ... lastprivate (...)}

	    \begin {itemize}
		\item idem \code {private}, sauf...
		\item ... à la fin de la boucle, la dernière valeur
		    calculée est recopiée dans le thread principal

	    \end {itemize}

	\item \code {\#pragma omp for ... ordered}

	    \begin {itemize}
		\item les itérations sont réalisées dans l'ordre
		\item intéressant si la boucle est dans une région
		    parallèle plus importante
	    \end {itemize}
    \end {itemize}
\end {frame}

\begin {frame} {Boucle parallèle -- Clauses 2/2}
    \begin {itemize}

	\item \code {\#pragma omp for ... schedule (}\emph {méthode}
	    [\code {,}\emph {taille\/}]\code {)}

	    \begin {itemize}
		\item \emph {taille}~: nombre d'itérations traitées
		    en un seul bloc indivisible

		\item méthode \code {static}~: les itérations sont
		    affectées aux threads de manière statique
		    (i.e. avant le démarrage de la boucle)

		\item méthode \code {dynamic}~: les itérations sont
		    affectées aux threads au fur et à mesure de leur
		    avancement

		\item méthode \code {guided}~: adapte dynamiquement
		    la taille des blocs indivisibles

		\item méthode \code {runtime}~: utilise la variable
		    \code {OMP\_SCHEDULE}

	    \end {itemize}

	\item \code {\#pragma omp for ... nowait}

	    \begin {itemize}
		\item supprime la barrière implicite en fin de boucle

	    \end {itemize}

    \end {itemize}
    
\end {frame}

\refstepcounter {ompexemple}

\begin {frame} [fragile] {Exemple \theompexemple{} -- Sections indépendantes}
    \lstinputlisting [basicstyle=\scriptsize\lstmonstyle, firstline=4] {\inc/parsect.c}

    \begin {itemize}
	\item \code {\#pragma omp sections}
	    introduit des sections indépendantes devant être exécutées
	    en parallèle

    \end {itemize}
\end{frame}

\begin {frame} {Exemple \theompexemple{} -- Section indépendantes}
    \begin {center}
	\includegraphics [width=.8\textwidth] {\inc/parsect}
    \end {center}

    \begin {itemize}
	\item chaque section débute par \code {\#pragma omp section}

	    \begin {itemize}
		\item moins de sections que de threads \implique threads
		    inactifs

		\item plus de sections que de threads \implique un thread
		    peut exécuter plus d'une section

	    \end {itemize}

	\item chaque section doit être un bloc structuré, sans
	    «~goto~» vers l'extérieur ou depuis l'extérieur (\code
	    {break}, etc.)

	\item simplification de l'écriture~: contraction des directives
	    \code {parallel} et \code {sections} en~:

	    \hspace* {5mm} \code {\#pragma omp parallel sections \emph {clauses}...}

    \end {itemize}
\end {frame}

\begin {frame} {Sections indépendantes -- Clauses}
    La directive \code {\#pragma omp sections} admet les clauses suivantes~:
    \begin {itemize}
	\item \code {\#pragma omp sections ... private (...)}
	\item \code {\#pragma omp sections ... firstprivate (...)}
	\item \code {\#pragma omp sections ... lastprivate (...)}
	\item \code {\#pragma omp sections ... reduction (...)}
	\item \code {\#pragma omp sections ... nowait}
    \end {itemize}

    \vspace* {3mm}

    La directive \code {\#pragma omp section} (délimiteur de section)
    n'admet aucune clause
\end {frame}

\refstepcounter {ompexemple}

\begin {frame} [fragile] {Exemple \theompexemple{} -- Exécution unique}
    \lstinputlisting [basicstyle=\scriptsize\lstmonstyle, firstline=4] {\inc/parsgl.c}

    \begin {itemize}
	\item Le bloc qui suit \code {\#pragma omp single}
	    est exécuté par un seul thread

    \end {itemize}
\end{frame}

\begin {frame} {Exemple \theompexemple{} -- Exécution unique}
    \begin {center}
	\includegraphics [width=.8\textwidth] {\inc/parsgl}
    \end {center}

    \begin {itemize}
	\item chaque bloc d'exécution unique est exécuté par un
	    et un seul thread

	    \begin {itemize}
		\item le choix du thread n'est pas déterministe
		\item les autres threads sautent le bloc et attendent
		    à la barrière implicite
	    \end {itemize}

    \end {itemize}
\end {frame}

\begin {frame} {Exécution unique -- Clauses}
    La directive \code {\#pragma omp single} admet les clauses suivantes~:
    \begin {itemize}
	\item \code {\#pragma omp single ... private (...)}
	\item \code {\#pragma omp single ... firstprivate (...)}
	\item \code {\#pragma omp single ... copyprivate (...)}
	\item \code {\#pragma omp single ... nowait}
    \end {itemize}

    \vspace* {3mm}
\end {frame}


\begin {frame} {Exécution unique -- Cas particulier}
    Directive \code {\#pragma omp master} : comparable à
    l'exécution unique, mais le bloc est exécuté par le thread
    principal

\end {frame}



%%%%%%%%%%%%%%%%%%%%%%%%%%%%%%%%%%%%%%%%%%%%%%%%%%%%%%%%%%%%%%%%%%%%%%%%%%%%%%
% Partage des données
%%%%%%%%%%%%%%%%%%%%%%%%%%%%%%%%%%%%%%%%%%%%%%%%%%%%%%%%%%%%%%%%%%%%%%%%%%%%%%

\titreB {Partage des données}

\begin {frame} {Partage des données}
    À l'intérieur d'une région parallèle, les variables (scalaires
    ou tableau) peuvent être~:

    \begin {itemize}
	\item partagées~: accessibles par tous les threads

	    \code {\#pragma omp parallel shared (i, tab)}

	    \begin {itemize}
		\item mode par défaut...
		\item ... modifiable avec la clause \code {default}
		\item attention aux modifications non atomiques !
	    \end {itemize}

	\item privées~: copie créée dans chaque thread

	    \code {\#pragma omp parallel private (i, tab)}

	    \begin {itemize}
		\item Valeur initiale indéfinie dans chaque thread...
		\item ...sauf si \code {\#pragma omp ... firstprivate()}
		\item Valeurs finales détruites en fin de région
		    parallèle...
		\item ...sauf si \code {\#pragma omp ... lastprivate()}
	    \end {itemize}
    \end {itemize}

    Exercice~: quid des variables référencées par des réductions ?
\end {frame}

\begin {frame} [fragile] {Partage des données}
    Attention à la portée lexicale~:
    \lstinputlisting [basicstyle=\scriptsize\lstmonstyle, firstline=4] {\inc/priv.c}

    \begin {itemize}
	\item des copies de \code {v} sont créées dans \code {main}
	\item \code {f} manipule la variable \code {v} originale
	\item chaque thread manipule donc une variable \code {v} non
	    initialisée !
    \end {itemize}
\end{frame}

\begin {frame} [fragile] {Partage des données -- \texttt {threadprivate}}
    Directive \code {\#pragma omp threadprivate}~: crée une copie
    privée (d'une variable globale) pour chaque thread

    \lstinputlisting [basicstyle=\scriptsize\lstmonstyle, firstline=4] {\inc/thrpriv.c} 
\end{frame}

\begin {frame} {Partage des données -- \texttt {threadprivate}}
    \begin {itemize}
	\item les variables globales marquées par \code {threadprivate}
	    sont dupliquées dans l'espace de chaque thread

	\item chaque copie a une valeur initiale indéfinie à l'entrée
	    de la première région parallèle rencontrée

	\item chaque copie conserve sa valeur entre régions
	    parallèles identiques (même nombre de threads, etc.)

    \end {itemize}
\end {frame}

\begin {frame} [fragile] {Partage des données -- \texttt {threadprivate}}

    À l'entrée d'une région parallèle, la clause \code {copyin}
    copie la valeur indiquée dans chaque copie~:

    \lstinputlisting [basicstyle=\scriptsize\lstmonstyle, firstline=4] {\inc/thrcopy.c}

\end{frame}

\begin {frame} {Partage des données -- Variables partagées}

    Accès multiples (lecture et écriture) à une variable partagée \\
    \implique résultat indéfini (cohérence des caches non garantie)

    \vspace* {3mm}

    \begin {itemize}
	\item directive \code {\#pragma omp flush (...)} : synchronise
	    les variables indiquées de sorte que la vue du thread soit
	    cohérente avec la mémoire

	\item \emph {flush} implicite lors d'une barrière, d'une
	    entrée ou d'une sortie d'une région parallèle ou d'une
	    section critique

    \end {itemize}




\end{frame}

%%%%%%%%%%%%%%%%%%%%%%%%%%%%%%%%%%%%%%%%%%%%%%%%%%%%%%%%%%%%%%%%%%%%%%%%%%%%%%
% Synchronisation
%%%%%%%%%%%%%%%%%%%%%%%%%%%%%%%%%%%%%%%%%%%%%%%%%%%%%%%%%%%%%%%%%%%%%%%%%%%%%%

\titreB {Synchronisation}

\begin {frame} {Synchronisation}
    OpenMP utilise des barrières implicites~:
    \begin {itemize}
	\item en fin de région parallèle
	\item en présence de blocs spécifiques~:
	    \begin {itemize}
		\item \code {\#pragma omp for}
		\item \code {\#pragma omp single}
		\item \code {\#pragma omp sections}
	    \end {itemize}
	    \implique sauf si clause \code {nowait}
	\item lorsque des portions de boucle doivent être ordonnées
	\item etc.
    \end {itemize}

    OpenMP offre des mécanismes de synchronisation explicites~:
    \begin {enumerate}
	\item barrière explicite
	\item section critique
	\item mise à jour atomique de variable
	\item verrou
    \end {enumerate}
\end {frame}

\refstepcounter {ompexemple}

\begin {frame} [fragile] {Exemple \theompexemple{} -- Barrière}
    \lstinputlisting [basicstyle=\scriptsize\lstmonstyle, firstline=4] {\inc/barrier.c}

    \begin {itemize}
	\item Pas de clause pour cette directive

    \end {itemize}
\end{frame}


\refstepcounter {ompexemple}

\begin {frame} [fragile] {Exemple \theompexemple{} -- Section critique}
    \lstinputlisting [basicstyle=\scriptsize\lstmonstyle, firstline=4] {\inc/critical.c}

    \begin {itemize}
	\item un seul thread à la fois dans toutes les sections
	    critiques de nom \code {modif\_c}

	\item pas de nom \implique un seul thread à la fois
	    dans toutes les sections critiques anonymes

    \end {itemize}
\end{frame}


\refstepcounter {ompexemple}

\begin {frame} [fragile] {Exemple \theompexemple{} -- Modification atomique}
    \lstinputlisting [basicstyle=\scriptsize\lstmonstyle, firstline=4] {\inc/atomic.c}

    \begin {itemize}
	\item l'instruction doit être de la forme \emph {var} \emph
	    {op}\code {=} \emph {expr} (ou \code {++}/\code {-{}-})

	\item overhead minimal car utilisation directe des facilités
	    disponibles du processeur

    \end {itemize}
\end{frame}


\begin {frame} {Verrous}
    OpenMP offre des verrous~:

    \begin {itemize}
	\item type \code {omp\_lock\_t}
	\item \code {void omp\_init\_lock (omp\_lock\_t *verrou)}
	\item \code {void omp\_destroy\_lock (omp\_lock\_t *verrou)}
	\item \code {void omp\_set\_lock (omp\_lock\_t *verrou)}
	\item \code {void omp\_unset\_lock (omp\_lock\_t *verrou)}
	\item \code {int omp\_test\_lock (omp\_lock\_t *verrou)}

    \end {itemize}
\end {frame}

