%
% Historique
%   2014/08/30 : pda : conception
%

\def\inc{inc3-excl}

\titreA {Exclusion mutuelle}

%%%%%%%%%%%%%%%%%%%%%%%%%%%%%%%%%%%%%%%%%%%%%%%%%%%%%%%%%%%%%%%%%%%%%%%%%%%%%%
% Variables critiques
%%%%%%%%%%%%%%%%%%%%%%%%%%%%%%%%%%%%%%%%%%%%%%%%%%%%%%%%%%%%%%%%%%%%%%%%%%%%%%

\titreB {Variables critiques}

\begin {frame} {Variable critique}
    Problème fondamental des systèmes concurrents~: l'accès aux
    variables critiques

    \vspace* {5mm}

    Variable critique = variable partagée en mémoire par plusieurs fils
    d'exécution, et dont les accès ne se font pas de manière atomique

\end {frame}

\begin {frame} [fragile] {Variable critique}
    Exemple en environnement multi-processeurs~:

    \vspace* {3mm}

    On suppose la variable \texttt {compte} en mémoire partagée
    par les deux processeurs.

    \vspace* {2mm}

\begin {minipage} {.45\textwidth}
\centerline {Processeur A}
\begin {lstlisting} [basicstyle=\scriptsize\lstmonstyle]
void f1 (void) {
  compte = compte + 50 ;
}
\end{lstlisting}
\end {minipage}
\hfill
\begin {minipage} {.45\textwidth}
\centerline {Processeur B}
\begin {lstlisting} [basicstyle=\scriptsize\lstmonstyle]
void f2 (void) {
  compte = compte + 10 ;
}
\end{lstlisting}
\end {minipage}

    \vspace* {2mm}

    En réalité, les accès à \texttt {compte} (en mémoire à l'adresse
    1000) ne sont pas atomiques. La compilation donne~:

    \vspace* {2mm}

\begin {minipage} {.45\textwidth}
\centerline {Processeur A}
\begin {lstlisting}
f1:  mov r0, 1000
     add r0, 50
     mov 1000, r0
\end{lstlisting}
\end {minipage}
\hfill
\begin {minipage} {.45\textwidth}
\centerline {Processeur B}
\begin {lstlisting}
f2:  mov r0, 1000
     add r0, 10
     mov 1000, r0
\end{lstlisting}
\end {minipage}

\end{frame}

\begin {frame} {Variable critique}
    Dans un système multi-processeurs~:

    \begin {itemize}
	\item les instructions indépendantes sont exécutées en parallèle
	\item les accès concurrents à la même case mémoire sont
	    effectués en séquence \\
	    (l'un après l'autre)
    \end {itemize}

\end {frame}

\begin {frame} {Variable critique}
    Normalement, après les deux opérations, \texttt {compte} devrait
    être augmenté de 60.

    \vspace* {3mm}

    Mais si les instructions sont exécutées concurremment~:

    \ctableau {\fB} {|c|l|c|c|c|} {
	\textbf {Proc} & \textbf {Instruction}
	    & \textbf {\texttt {r0}$_A$}
	    & \textbf {\texttt {r0}$_B$}
	    & \textbf {Mém 1000}
	    \\
	\multicolumn {2} {|l|} {initialement} &  ?  &  ?  & 200 \\
	A & \texttt {mov r0,1000} & 200 &  ?  & 200 \\
	B & \texttt {mov r0,1000} & 200 & 200 & 200 \\
	A & \texttt {add r0,50}   & 250 & 200 & 200 \\
	B & \texttt {add r0,10}   & 250 & 210 & 200 \\
	A & \texttt {mov 1000,r0} & 250 & 210 & 250 \\
	B & \texttt {mov 1000,r0} & 250 & 210 & 210 \\
    }

    Résultat~: \texttt {compte} est augmenté de 10~!
\end {frame}

\begin {frame} {Variable critique}
    Le problème se pose même en environnement monoprocesseur~:

    \begin {itemize}
	\item \texttt {f1} est exécuté par un premier processus
	\item \texttt {f2} est exécuté par un deuxième processus
	\item \texttt {compte} est en mémoire partagée (IPC System V)
    \end {itemize}

\end {frame}

\begin {frame} {Variable critique}
    Si l'ordonnancement des instructions est~:

    \begin {enumerate}
	\item le premier processus exécute \texttt {f1} jusqu'à \texttt {add}
	\item le système provoque une commutation de processus et donne le
	    processeur aux deuxième processus
	\item le deuxième processus exécute toute la fonction \texttt {f2}
	\item le premier processus est plus tard remis sur le processeur
	    et exécute le reste des instructions
    \end {enumerate}
    \implique alors le résultat est une augmentation de 50 et non de 60 !

    \vspace* {5mm}

    Pas besoin de multi-processeurs pour avoir des problèmes de
    concurrence

\end {frame}

\begin {frame} [fragile] {Variable critique}
    Le problème se pose même s'il n'y a qu'un seul processus...

\begin {minipage} {.45\textwidth}
\begin {lstlisting} [basicstyle=\scriptsize\lstmonstyle]
int compte ;

main ()
{
    signal (SIGINT, f) ;
    compte = compte + 10 ;
}
\end{lstlisting}
\end {minipage}
\hfill
\begin {minipage} {.45\textwidth}
\begin {lstlisting} [basicstyle=\scriptsize\lstmonstyle]
void f (int signo)
{
    compte = compte + 50 ;
}
\end{lstlisting}
\end {minipage}

    La variable \texttt {compte} est partagée par plusieurs fils
    d'exécution...

    \vspace* {7mm}

    \implique problème analogue rencontré à l'intérieur du système
    d'exploitation, avec les interruptions matérielles

\end{frame}


\begin {frame} [fragile] {Variable critique}
    Le problème se pose même sans mémoire physique partagée~:

    \vspace* {3mm}

\begin {minipage} {.45\textwidth}
\centerline {Processus 1}
\scriptsize
\begin {lstlisting} [basicstyle=\scriptsize\lstmonstyle]
main () {
  int v, fd ;
  fd = open ("compte", O_RDWR) ;
  read (fd, &v, sizeof (v)) ;
  v = v + 50 ;
  lseek (fd, 0, SEEK_SET) ;
  write (fd, &v, sizeof (v)) ;
  close (fd) ;
}
\end{lstlisting}
\end {minipage}
\hfill
\begin {minipage} {.45\textwidth}
\centerline {Processus 2}
\begin {lstlisting} [basicstyle=\scriptsize\lstmonstyle]
main () {
  int v, fd ;
  fd = open ("compte", O_RDWR) ;
  read (fd, &v, sizeof (v)) ;
  v = v + 10 ;
  lseek (fd, 0, SEEK_SET) ;
  write (fd, &v, sizeof (v)) ;
  close (fd) ;
}
\end{lstlisting}
\end {minipage}

    Ici (par rapport au cas multi-processeurs ou multi-processus)~:
    \begin {itemize}
	\item fichier $\approx$ mémoire partagée du cas précédent
	\item variable \texttt {v} de chaque processus $\approx$
	    registre \texttt {r0} de chaque processeur du cas précédent
    \end {itemize}

\end{frame}

\begin {frame} {Bilan}
    Les problèmes de concurrence sur l'accès à une variable critique
    se posent~:

    \begin {itemize}
	\item en environnement multi-processeurs
	\item même avec un seul processeur, et plusieurs
	    processus
	\item même avec un seul processus, et la réception des signaux
	\item même sans partage de mémoire physique, avec des fichiers

    \end {itemize}

    \implique condition de concurrence (\emph {race condition})

    \vspace* {3mm}

    Toujours le même problème~: accès concurrents non atomiques à
    un objet partagé

\end {frame}

%%%%%%%%%%%%%%%%%%%%%%%%%%%%%%%%%%%%%%%%%%%%%%%%%%%%%%%%%%%%%%%%%%%%%%%%%%%%%%
% Section critique
%%%%%%%%%%%%%%%%%%%%%%%%%%%%%%%%%%%%%%%%%%%%%%%%%%%%%%%%%%%%%%%%%%%%%%%%%%%%%%

\titreB {Section critique}

\begin {frame} {Section critique}
    Problème : comment rendre atomique l'accès aux variables critiques ?

    \begin {center}
	\includegraphics [width=.7\textwidth] {\inc/sect-crit}
    \end {center}

\end {frame}

\begin {frame} {Section critique}
    Le protocole de section critique doit satisfaire les conditions
    suivantes~:

    \begin {itemize}
	\item exclusion mutuelle~: un seul processus (ou processeur)
	    peut entrer dans la section critique à un instant donné

	\item absence d'interférence~: si aucun processus n'est dans
	    la section critique, seuls les processus en compétition
	    pour l'accès à la section critique participent au choix

	    peut entrer dans la section critique à un instant donné

	\item absence de famine~: lorsqu'un processus souhaite entrer
	    dans la section critique, son attente ne doit pas être
	    infinie

    \end {itemize}
\end {frame}

%%%%%%%%%%%%%%%%%%%%%%%%%%%%%%%%%%%%%%%%%%%%%%%%%%%%%%%%%%%%%%%%%%%%%%%%%%%%%%
% En environnement multi-processeurs
%%%%%%%%%%%%%%%%%%%%%%%%%%%%%%%%%%%%%%%%%%%%%%%%%%%%%%%%%%%%%%%%%%%%%%%%%%%%%%

\titreB {Section critique avec parallélisme réel}

\begin {frame} {Section critique}
    Comment résoudre le problème de la section critique en environnement
    multi-processeurs~?

    \begin {itemize}
	\item sans support matériel : algorithme de Peterson
	\item avec support matériel : cycle indivisible de
	    lecture-modification
    \end {itemize}
\end {frame}

\begin {frame} [fragile] {Section critique -- Sans support matériel}
    Solution naïve~:

    \begin {lstlisting} [basicstyle=\scriptsize\lstmonstyle]
    void verrouiller (int *verrou) {
      while (*verrou != 0)
        ;       /* attente active */
      *verrou = 1 ;
    }
    void deverrouiller (int *verrou) {
      *verrou = 0 ;
    }
    \end{lstlisting}

    \implique ne fonctionne pas...

\end{frame}

\begin {frame} [fragile] {Section critique -- Sans support matériel}
    Pourquoi cette solution ne fonctionne pas~? \\
    Problème~: \code {*verrou} est elle-même une variable critique

    \vspace* {3mm}

    Exemple de code généré par le compilateur~:

    \begin {lstlisting} [language=bash,basicstyle=\scriptsize\lstmonstyle]
    verrouiller:        # on suppose le parametre dans r1
       mov r0,(r1)      # r0 <- lecture a l'adresse r1
       tst r0           # tester le contenu de r0
       jnz verrouiller  # saut si precedente valeur != 0
       mov r0,1
       mov (r1),r0      # memoire a l'adresse r1 <- 1
       rtn
    \end{lstlisting}

\end{frame}

\begin {frame} {Section critique -- Sans support matériel}
    Si deux processeurs essayent d'exécuter \code {verrouiller}
    en même temps (on suppose que le verrou est à l'adresse 1000)~:

    \ctableau {\fC} {|c|l|c|c|c|} {
	\textbf {Proc} & \textbf {Instruction}
	    & \textbf {\texttt {r0}$_A$}
	    & \textbf {\texttt {r0}$_B$}
	    & \textbf {Mém 1000}
	    \\
	\multicolumn {2} {|l|} {initialement} & ? & ? & 0 \\
	A & \texttt {mov r0,(r1)} & 0 & ? & 0 \\
	B & \texttt {mov r0,(r1)} & 0 & 0 & 0 \\
	A & \texttt {tst r0}      & 0 & 0 & 0 \\
	B & \texttt {tst r0}      & 0 & 0 & 0 \\
	A & \texttt {mov r0,1}    & 1 & 0 & 0 \\
	B & \texttt {mov r0,1}    & 1 & 1 & 0 \\
	A & \texttt {mov (r1),r0} & 1 & 1 & 1 \\
	B & \texttt {mov (r1),r0} & 1 & 1 & 1 \\
    }

    \implique les deux processeurs sont entrés en section critique !

\end {frame}

\begin {frame} [fragile] {Section critique -- Sans support matériel}
    Algorithme de Peterson, pour 2 processeurs P$_i$ et P$_j$~:

    \begin {itemize}
	\item Pour P$_i$ :
    \begin {lstlisting} [basicstyle=\scriptsize\lstmonstyle]
    jeveux [i] = true ;
    tour = j ;
    while (jeveux [j] && tour == j)
       ;                /* attente active */
    /* section critique */
    jeveux [i] = false ;
    \end{lstlisting}

	\item Pour P$_j$ : algorithme symétrique
    \end {itemize}

    \vspace* {3mm}
    Problèmes de l'algorithme de Peterson~:
    \begin {itemize}
	\item attente active

	\item attention à l'optimisation du code par les compilateurs \\
	    Ex: test du \texttt {while} considéré comme constant...
    \end {itemize}

\end{frame}

\begin {frame} {Section critique -- Avec support matériel}
    Support matériel = instruction du processeur réalisant une
    opération indivisible (i.e. atomique) de lecture et de modification
    d'une case mémoire
    \\
    \implique aucun autre processeur n'a accès à la mémoire pendant
    la lecture et la modification.

    \vspace* {3mm}

    Exemples d'instructions, selon le processeur~:
    \begin {itemize}
	\item TAS (\emph {test and set}) sur Motorola 68000
	\item FAS (\emph {fetch and set})
	\item XCHG (\emph {exchange})
	\item CAS (\emph {compare and swap})
	\item préfixe LOCK sur x86
	\item etc.
    \end {itemize}

\end {frame}

\begin {frame} [fragile] {Section critique -- Avec support matériel}
    Fonctionnement de ces instructions (algorithmes en C)~:

    \begin {lstlisting} [basicstyle=\scriptsize\lstmonstyle]
    int fetch_and_set (int *adresse, int val) {
      int vieux ;
      vieux = *adresse ;
      *adresse = val ;
      return vieux ;
    }

    int compare_and_swap (int *adresse, int ok, int val) {
      int vieux ;
      vieux = *adresse ;
      if (vieux == ok)
        *adresse = val ;
      return vieux ;
    }
    \end{lstlisting}

    \implique implémentés en matériel dans le processeur

\end{frame}


\begin {frame} [fragile] {Section critique -- Avec support matériel}
    Solution avec l'instruction «~\emph {test and set}~»~:

    \begin {lstlisting} [basicstyle=\scriptsize\lstmonstyle]
    void verrouiller (int *verrou) {
      while (fetch_and_set (*verrou, 1) == 0)
        ;       /* attente active */
    }
    void deverrouiller (int *verrou) {
      *verrou = 0 ;
    }
    \end{lstlisting}

\end{frame}

\begin {frame} [fragile] {Section critique -- Avec support matériel}

    La fonction \texttt {verrouiller} en assembleur~:

    \begin {lstlisting} [language=bash,basicstyle=\scriptsize\lstmonstyle]
    verrouiller:        # on suppose le parametre dans r1
       mov r0,1
       fas (r1),r0      # r0tmp=*verrou, *verrou=r0, r0=r0tmp
       tst r0           # tester le contenu de r0
       jnz  verrouiller # saut si precedente valeur != 0
       rtn
    \end{lstlisting}

\end{frame}

\begin {frame} {Section critique -- Avec support matériel}
    Si deux processeurs essayent d'exécuter \texttt {verrouiller}
    en même temps (on suppose que le verrou est à l'adresse 1000)~:

    \ctableau {\fC} {|c|l|c|c|c|} {
	\textbf {Proc} & \textbf {Instruction}
	    & \textbf {\texttt {r0}$_A$}
	    & \textbf {\texttt {r0}$_B$}
	    & \textbf {Mém 1000}
	    \\
	\multicolumn {2} {|l|} {initialement} & ? & ? & 0 \\
	A & \texttt {mov r0,1}    & 1 & ? & 0 \\
	B & \texttt {mov r0,1}    & 1 & 1 & 0 \\
	B & \texttt {fas (r1),r0} & \textbf {0} & 0 & \textbf {1} \\
	B & \texttt {fas (r1),r0} & 0 & \textbf {1} & \textbf {1} \\
	A & \texttt {tst r0}      & 0 & 1 & 1 \\
	B & \texttt {tst r0}      & 0 & 1 & 1 \\
	A & \texttt {rtn}         & 0 & 1 & 1 \\
	B & \texttt {jnz ...}     & 0 & 1 & 1 \\
    }

    \implique seul A est entré dans la section critique, B est toujours
    dans l'attente active

\end {frame}

\begin {frame} {Section critique -- Avec support matériel}
    Les solutions précédentes utilisent toutes l'\textbf {attente
    active}~:

    \begin {itemize}
	\item mauvaise utilisation du processeur \\
	    \implique processeur pas utilisable par un
		autre processus

	\item généralement pas adaptée aux systèmes pouvant exécuter
	    plusieurs processus simultanément \\
	    \implique intéressant si machine dédiée à une seule
	    application

	\item tolérable dans les systèmes multi-processus pour des
	    durées extrêmement courtes

    \end {itemize}

    \vspace* {3mm}

    \implique support nécessaire du système d'exploitation pour
    réaliser une \textbf {attente passive} \\
    \implique \emph {overhead} plus important

\end {frame}

%%%%%%%%%%%%%%%%%%%%%%%%%%%%%%%%%%%%%%%%%%%%%%%%%%%%%%%%%%%%%%%%%%%%%%%%%%%%%%
% Dans un SE
%%%%%%%%%%%%%%%%%%%%%%%%%%%%%%%%%%%%%%%%%%%%%%%%%%%%%%%%%%%%%%%%%%%%%%%%%%%%%%

\titreB {Section critique avec parallélisme virtuel}

\begin {frame} {Section critique}
    Comment résoudre le problème de la section critique dans
    le contexte d'un système d'exploitation ? \\
    \implique environnement multi-processsus ou multi-threads \\
    (basé ou non sur un système multi-processeurs)

    \vspace* {3mm}
    Plusieurs contextes~:
    \begin {enumerate}
	\item sans dispositif spécifique de communication
	\item partage de mémoire avec les IPC System V
	\item partage de mémoire avec les threads
	\item partage de mémoire avec les signaux
    \end {enumerate}

\end {frame}

\begin {frame} [fragile] {Section critique -- Sans dispositif de communication}

    Exemple~: processus souhaitant communiquer via un fichier

    \vspace* {3mm}

    «~Hack~» traditionnel~:

    \begin {lstlisting} [basicstyle=\scriptsize\lstmonstyle]
    void verrouiller (char *verrou) {
      int fd ;

      do {
        fd = open (verrou, O_CREAT | O_WRONLY | O_EXCL, 0600) ;
        if (fd == -1 && errno == EEXISTS)
          sleep (1) ;        /* attente active ralentie */
      } while (fd == -1 && errno == EEXISTS) ;
      if (fd == -1)
        /* raler */
      close (fd) ;
    }

    void deverrouiller (char *verrou) {
      unlink (verrou) ;
    }
    \end{lstlisting}

\end{frame}

\begin {frame} {Section critique -- Sans dispositif de communication}
    L'opération «~\code {open (}...\code {O\_CREAT|O\_EXCL}...\code
    {)}~» est une opération indivisible de lecture et de modification~:

    \begin {itemize}
	\item lecture et modification de l'existence d'un fichier
	\item indivisible à l'échelle du système d'exploitation \\
	    \implique aucun autre processus ou thread ne peut s'intercaler
	    entre le test et la modification

    \end {itemize}

    Problème~: il reste l'attente active
    (\code {sleep} ne fait que \emph {ralentir} l'attente active, mais ne
    la rend pas passive pour autant)

\end {frame}

\begin {frame} [fragile] {Section critique -- Sans dispositif de communication}
    Utilisation de \code {lockf} sur système BSD~:

    \begin {lstlisting} [basicstyle=\scriptsize\lstmonstyle]
    void verrouiller (int fd) {
      if (lockf (fd, F_LOCK, 0) == -1)
        /* raler */
    }
    void deverrouiller (int fd) {
      if (lockf (fd, F_ULOCK, 0) == -1)
        /* raler */
    }
    \end{lstlisting}

    \implique attente \textbf {passive}

    \vspace* {3mm}

    POSIX~: \code {fcntl()} plus délicat à utiliser
\end{frame}


\begin {frame} {Section critique -- En shell}
    Commandes utilisables en Shell
    \begin {itemize}
	\item Linux, BSD :

	    \code {flock /tmp/toto.lock /bin/ls}

	\item BSD :

	    \code {lockf /tmp/toto.lock /bin/ls}
    \end {itemize}

\end {frame}

\begin {frame} {Section critique -- Sans dispositif de communication}
    D'autres mécanismes de synchronisation peuvent également être
    utilisés ou créés~:

    \begin {itemize}
	\item tubes
	\item files de messages IPC System V
	\item sémaphores IPC System V (plus tard)
	\item sémaphores Posix
	\item etc.
    \end {itemize}
\end {frame}

\begin {frame} {Section critique -- Entre threads}
    Comment réaliser une section critique entre threads ?

    POSIX fournit les opérations de verrouillage~:

    \begin {itemize}
	\item \code {\small int pthread\_mutex\_init (pthread\_mutex\_t *, pthread\_mutexattr\_t *)}
	\item \code {\small int pthread\_mutex\_destroy (pthread\_mutex\_t *)}
	\item \code {\small int pthread\_mutex\_lock (pthread\_mutex\_t *)}
	\item \code {\small int pthread\_mutex\_unlock (pthread\_mutex\_t *)}
	\item \code {\small int pthread\_mutex\_timedlock (pthread\_mutex\_t *, struct timespec *habs)}

	    Avec \code {\small struct timespec \{ time\_t tv\_sec; long tv\_nsec; \}}
	\item \code {\small int pthread\_mutex\_trylock (pthread\_mutex\_t *, struct timespec *habs)}
    \end {itemize}
\end {frame}

\begin {frame} [fragile] {Section critique -- Entre threads}
    Exemple~: initialisation d'un verrou

    \begin {lstlisting} [basicstyle=\scriptsize\lstmonstyle]
    pthread_mutex_t m ;

    void init (void) {
        if ((errno = pthread_mutex_init (&m, NULL)) != 0)
            ...
    }
    \end{lstlisting}


    On peut également utiliser l'initialisation statique~:

    \begin {lstlisting} [basicstyle=\scriptsize\lstmonstyle]
    pthread_mutex_t m = PTHREAD_MUTEX_INITIALIZER ;
    \end{lstlisting}

    \implique seulement dans le cas d'une initialisation lors de la
    déclaration de variable

\end{frame}

\begin {frame} {Section critique -- Entre threads}
    Les fonctions \code {pthread\_mutex\_*} réalisent un verrouillage
    avec attente \textbf {passive}

    \vspace* {3mm}

    POSIX fournit également les fonctions \code {pthread\_spin\_*} 
    qui utilisent l'attente \textbf {active}
    \\
    (sauf \texttt {\small pthread\_spin\_timedlock} qui n'existe pas)
    \\
    \implique attention à l'utilisation de ces fonctions \\
    (tolérable pour des durées extrêmement courtes)

\end {frame}

\begin {frame} {Section critique -- Interruptions et signaux}

    Il peut y avoir une condition de concurrence entre le fil d'activité
    «~normal~» et la fonction associée à un signal
    \\
    Problème similaire : code «~normal~» du système d'exploitation et une
    routine associée à une interruption

    \vspace* {3mm}

    Même solution dans les deux cas~:

    \begin {itemize}
	\item section critique protégée de manière asymétrique

	\item le code «~normal~» est protégé en désactivant
	    temporairement la prise en compte du signal (resp. de
	    l'interruption)

	\item la fonction associée au signal (resp. à l'interruption)
	    est naturellement protégée contre le signal
	    (resp. l'interruption)

    \end {itemize}

\end {frame}

\begin {frame} {Section critique -- Interruptions matérielles}
    \begin {minipage} {.40\textwidth}
	\includegraphics [width=.8\textwidth] {\inc/int-princ}
    \end {minipage}
    \begin {minipage} {.58\textwidth}
	\includegraphics [width=\textwidth] {\inc/int-ack}
    \end {minipage}

    \begin {itemize}
	\item À la fin de chaque instruction, le CPU vérifie, si
	    son registre d'état l'autorise, le statut de la ligne
	    d'interruption

	\item Si l'interruption n'est pas masquée, le CPU sauvegarde
	    l'état, et branche à la fonction correspondante

	\item La fonction interroge le périphérique pour savoir si
	    c'est lui qui interrompt. Si c'est le cas, il acquitte
	    l'interruption en écrivant une valeur dans le registre
	    d'état du périphérique
    \end {itemize}

\end {frame}

\begin {frame} {Section critique -- Interruptions matérielles}

    \begin {minipage} {.58\textwidth}
	Masquage des interruptions : 
	\begin {itemize}
	    \item écriture dans le registre
		d'état du processeur
	    \item l'interruption n'est pas prise en compte
	\end {itemize}
	Démasquage~:
	\begin {itemize}
	    \item écriture dans le registre
		d'état du processeur
	    \item interruption prise en compte
	    \item le périphérique sera acquitté
	\end {itemize}
	Pendant le traitement de l'interruption,
	celle-ci est automatiquement masquée
    \end {minipage}
    \begin {minipage} {.40\textwidth}
	\includegraphics [width=\textwidth] {\inc/int-mask}
    \end {minipage}

\end {frame}

\begin {frame} {Section critique -- Interruptions matérielles}
    Ce dispositif d'exclusion mutuelle entre code «~normal~» du
    système d'exploitation et routines d'interruptions est
    à la base du système Unix (mono-processeur)

\end {frame}

\begin {frame} {Section critique -- Signaux}
    Et pour les processus en mode «~utilisateur~» ?

    \begin {itemize}
	\item Signaux V7~: primitive \code {signal}

	    \implique insuffisants (perte de signaux, pas de masquage
	    possible, etc.)

	\item Signaux BSD (puis POSIX)~: primitive \code {sigaction}

	    \implique plus grande ressemblance avec les interruptions
	    matérielles

    \end {itemize}
\end {frame}

\begin {frame} {Section critique -- Signaux}

    \begin {minipage} {.48\textwidth}
	Masquage du signal : 
	\begin {itemize}
	    \item \code {sigprocmask} ou
		\code {pthread\_sigmask}
	    \item le signal est reçu, mais non pris en compte
	\end {itemize}
	Démasquage du signal~:
	\begin {itemize}
	    \item \code {sigprocmask} ou
		\code {pthread\_sigmask}
	    \item le signal en attente est pris en compte
	\end {itemize}
	Pendant le traitement du signal,
	celui-ci est automatiquement masqué
    \end {minipage}
    \begin {minipage} {.50\textwidth}
	\includegraphics [width=\textwidth] {\inc/sig-mask}
    \end {minipage}
\end {frame}

\begin {frame} [fragile] {Section critique -- Signaux}

    Pour le code «~normal~» devant être en exclusion mutuelle avec
    la fonction associée à \texttt {SIGUSR1}~:

    \begin {lstlisting} [basicstyle=\scriptsize\lstmonstyle]
    void verrouiller (int fd) {
      sigset_t s ;

      sigemptyset (&s) ;
      sigaddset (&s, SIGUSR1) ;
      sigprocmask (SIG_BLOCK, &s, NULL) ;
    }

    void deverrouiller (int fd) {
      sigset_t s ;

      sigemptyset (&s) ;
      sigaddset (&s, SIGUSR1) ;
      sigprocmask (SIG_UNBLOCK, &s, NULL) ;
    }
    \end{lstlisting}
    (utiliser \code {pthread\_sigmask} pour des threads)

    Pour la fonction de traitement de signal~: rien à faire
    \\
    (masquage implicite)

\end{frame}
