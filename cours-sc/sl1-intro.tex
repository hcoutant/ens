\def\inc{inc1-intro}

\def\cpudb{Stanford CPU database -- \url {http://cpudb.stanford.edu/}}

\titreA {Introduction aux systèmes concurrents}

%%%%%%%%%%%%%%%%%%%%%%%%%%%%%%%%%%%%%%%%%%%%%%%%%%%%%%%%%%%%%%%%%%%%%%%%%%%%%%
% Introduction
%%%%%%%%%%%%%%%%%%%%%%%%%%%%%%%%%%%%%%%%%%%%%%%%%%%%%%%%%%%%%%%%%%%%%%%%%%%%%%

\titreB {Introduction}

\begin {frame} {Introduction}

    Qu'est-ce qu'un système concurrent~?

    \begin {quote}
	Système dans lequel plusieurs activités s'exécutent
	simultanément tout en ayant des interactions les unes avec
	les autres

    \end {quote}

    \vspace* {3mm}

    Exemples~:

    \begin {itemize}
	\item serveur (Web, base de données, jeu, etc.) répondant à
	    des requêtes de plusieurs clients «~simultanés~»
	\item système de contrôle d'un véhicule effectuant des actions
	    (débit d'essence) en fonction d'informations provenant de
	    capteurs (tachymètre)
	\item application de commerce en ligne affichant l'état du
	    stock en fonction des commandes en cours

    \end {itemize}

\end {frame}

\begin {frame} {Introduction}

    Système concurrent \implique plusieurs activités logicielles
    partagent une ou plusieurs ressources matérielles

    \vspace* {3mm}

    Deux approches~:

    \begin {enumerate}
	\item Sérialiser les traitements
	    \begin {itemize}
		\item Sérialiser en bloc \implique pas de
		    concurrence
		\item Découper les traitements et sérialiser chaque élément \\
		    Exemple~: module \texttt {asyncio} avec Python

	    \end {itemize}
	    \implique performances médiocres pour des systèmes complexes

	\item Utiliser le parallélisme sous-jacent
	    \begin {itemize}
		\item parallélisme matériel
		\item parallélisme offert par le système d'exploitation
	    \end {itemize}
	    \implique meilleure utilisation du matériel
    \end {enumerate}

    Dans la suite, on s'intéressera à la deuxième approche

\end {frame}

%%%%%%%%%%%%%%%%%%%%%%%%%%%%%%%%%%%%%%%%%%%%%%%%%%%%%%%%%%%%%%%%%%%%%%%%%%%%%%
% Motivations
%%%%%%%%%%%%%%%%%%%%%%%%%%%%%%%%%%%%%%%%%%%%%%%%%%%%%%%%%%%%%%%%%%%%%%%%%%%%%%

\titreB {Motivations}

\begin {frame} {Motivations}
    Pourquoi des systèmes concurrents (ou parallèles)~?

    \begin {enumerate}
	\item besoin de performances
	\item besoin de \emph {scalabilité}
	\item besoin de fiabilité
	\item besoin de clarté
    \end {enumerate}
\end {frame}

\begin {frame} {Motivations -- Performances}
    Amélioration des performances~:
    \begin {itemize}
	\item pour diminuer le temps de calcul

	    \vspace* {1mm}

	    Ex~: prévision météo à 3 jours en moins de 3 jours
	    de calcul...

	\item pour traiter des problèmes plus grands

	    \vspace* {1mm}

	    Ex~: prévision météo plus précise signifie~:
	    \begin {itemize}
		\item maillage plus fin
		\item plus de paramètres influant sur le calcul
	    \end {itemize}
	    \implique temps de calcul plus important

    \end {itemize}

\end {frame}

\begin {frame} {Motivations -- Performances}

    Pour aller plus vite, il faut :

    \begin {enumerate}
	\item améliorer le débit \implique récupérer les temps morts
	\item améliorer le temps de réponse \implique partager la charge
	\item améliorer le temps d'exécution \implique collaborer
    \end {enumerate}
\end {frame}

\begin {frame} {Motivations -- Performances}
    Récupérer les temps morts~:

    \begin {center}
	\includegraphics [width=.9\textwidth] {\inc/motiv-inact}
    \end {center}

    Récupération des temps morts \implique plusieurs processus

    \vspace* {3mm}

    \implique parallélisme entre  plusieurs processus dans un système
    d'exploitation
\end {frame}

\begin {frame} {Motivations -- Performances}
    Partager la charge~:

    \begin {center}
	\includegraphics [width=.9\textwidth] {\inc/motiv-part}
    \end {center}

\end {frame}

\begin {frame} {Motivations -- Performances}
    Collaborer~:

    \begin {center}
	\includegraphics [width=.9\textwidth] {\inc/motiv-collab}
    \end {center}

\end {frame}

\begin {frame} {Motivations -- Performances}
    En pratique, accroître les performances passe par~:
    \begin {itemize}
	\item le découplage des unités de traitement
	\item le découplage de la mémoire
	\item le découplage des entrées/sorties
    \end {itemize}

    \vspace* {3mm}
    \implique multi-ordinateurs
\end {frame}

\begin {frame} {Motivations -- Performances}

    Faut-il passer par le parallélisme~? \\
    L'augmentation des performances du silicium ne suffit-elle pas~?

    \vspace* {3mm}

    Augmentation des fréquences d'horloge des processeurs~:

    \begin {itemize}
	\item réduction de la durée du cycle d'horloge
	\item unités fonctionnelles du processeur plus
	    rapides
    \end {itemize}
    \implique davantage d'instructions exécutées par seconde
\end {frame}

\begin {frame} {Motivations -- Performances}
    Augmentation des fréquences d'horloge des processeurs~:

    \begin {center}
	\includegraphics [width=.9\textwidth] {\inc/cpu-freq}

	\centerline {\tiny Données : \cpudb}
    \end {center}

\end {frame}

\begin {frame} {Motivations -- Performances}
    L'augmentation de la fréquence d'horloge entraîne~:
    \begin {itemize}
	\item interférences, notamment sur des «~longues~» distances \\
	    \implique i.e. dès qu'il faut sortir du carré de silicium
	\item consommation électrique
	\item dissipation thermique
    \end {itemize}
    \vspace* {3mm}
    \implique palier\footnote {Augmentation depuis le milieu des années
    1980 : processeurs RISC} depuis le milieu des années 2000

\end {frame}

\begin {frame} {Motivations -- Performances}
    Aller au delà de l'augmentation des fréquences \implique 2 pistes

    \vspace* {3mm}

    \begin {enumerate}
	\item réduire les distances \implique augmenter la finesse de gravure
	    \begin {itemize}
		\item réduire les interférences
		\item diminuer la tension (réduire les pertes électriques)
	    \end {itemize}
	\item augmenter le nombre de transistors \implique utiliser plus
	    de place et/ou réduire leur taille

	    \begin {itemize}
		\item placer davantage de mémoire sur la puce
		\item placer des unités fonctionnelles en parallèle
	    \end {itemize}
    \end {enumerate}
\end {frame}

\begin {frame} {Motivations -- Performances}
    Finesse minimum de gravure des processeurs, par année :
    \begin {center}
	\includegraphics [width=.9\textwidth] {\inc/cpu-gravure}

	{\tiny Données : \cpudb}
    \end {center}
\end {frame}

\begin {frame} {Motivations -- Performances}
    Peut-on augmenter encore la finesse de gravure ?

    \vspace* {5mm}

    \implique de plus en plus difficile~: proche de la taille
    d'un atome

\end {frame}

\begin {frame} {Motivations -- Performances}
    Loi de Moore (1975)~: \emph {le nombre de transistors dans les
    microprocesseurs double tous les deux ans}
    \\
    \implique C'est une supposition...

    \begin {center}
	\includegraphics [width=.9\textwidth] {\inc/cpu-transist}

	{\tiny Données : \cpudb}
    \end {center}

    % Limite de la loi de Moore~: on se rapproche également des dimensions
    % atomiques

\end {frame}

\begin {frame} {Motivations -- Performances}
    Conclusion~:
    
    \begin {itemize}
	\item on ne peut plus tellement réduire le temps de cycle
	\item on peut encore mettre davantage de transistors sur la
	    même puce

	\item d'où : parallélisme sur une même puce \\
	    \implique processeurs multi-c{\oe}urs

    \end {itemize}

    Pour aller plus loin (i.e. plus vite) \implique davantage de parallélisme
    \\
    \implique architectures multi-processeurs

\end {frame}

\begin {frame} {Motivations -- Scalabilité}
    «~\emph {scalability}~»~: augmenter les performances en cas
    de besoin

    \vspace* {3mm}

    Exemples~:

    \begin {itemize}
	\item serveurs de noms sur Osiris
	\item ferme de serveurs pour un moteur de recherche
	\item etc.
    \end {itemize}

    \begin {center}
	\includegraphics [width=.5\textwidth] {\inc/motiv-scal}
    \end {center}

\end {frame}

\begin {frame} {Motivations -- Fiabilité}

    Exemple~: ordinateurs de marque «~Tandem~» (1975--1985)

    \begin {itemize}
	\item société spécialisée dans les ordinateurs à tolérance
	    de panne
	\item un ordinateur Tandem = deux ordinateurs se surveillant
	    en permanence

    \end {itemize}

\end {frame}

\begin {frame} {Motivations -- Clarté}
    Un programme parallèle peut être plus clair :

    \begin {itemize}
	\item structuration, décomposition en petits modules
	\item expression du parallélisme inhérent au problème
	    \begin {itemize}
		\item tâches indépendantes
		\item traitement d'événements multi-sources
	    \end {itemize}
    \end {itemize}

    Un programme parallèle \emph {peut} être plus clair...
    mais pas forcément !


\end {frame}

\begin {frame} {Motivations -- Difficultés}
    Un programme parallèle~:

    \begin {itemize}
	\item n'est pas forcément déterministe \\
	    \implique 2 exécutions différentes ne donnent pas
	    forcément le même résultat
	\item n'est pas toujours facile à programmer \\
	    \implique problèmes de synchronisation difficiles
	\item n'est pas facile à déboguer \\
	    \implique conséquence des deux premiers points
    \end {itemize}

\end {frame}

\begin {frame} {Motivations -- Difficultés}
    Un programme parallèle n'est pas forcément déterministe

    \vspace* {3mm}

    Exemple sur 2 processeurs avec une mémoire partagée :

    \vspace* {3mm}

    \begin {minipage} {.58\textwidth}
	\begin {center}
	    \begin {tabular} {l|l}
		\multicolumn {1} {c|} {P$_1$} &
		    \multicolumn {1} {c} {P$_2$} \\ \hline
		(1) \code {a=1} & (3) \code {a=3} \\
		(2) \code {b=2} & \\
	    \end {tabular}
	\end {center}
	Arbre des exécutions possibles~:
	\begin {itemize}
	    \item n{\oe}uds : états de la mémoire
	    \item arcs : instructions
	\end {itemize}
    \end {minipage}
    \begin {minipage} {.40\textwidth}
	\includegraphics [width=\textwidth] {\inc/arbre-exec}
    \end {minipage}

    \vspace* {3mm}

    2 états terminaux distincts \implique programme non déterministe
\end {frame}

\begin {frame} {Motivations -- Synthèse}
    En synthèse, le parallélisme :
    \begin {itemize}
	\item est indispensable pour obtenir de meilleures
	    performances

	    \begin {itemize}
		\item l'augmentation des performances du silicium
		    ne suffit pas toujours...

		\item ... et l'augmentation des performances du silicium
		    passe par davantage de parallélisme
		    \\
		    \implique tous les ordinateurs actuels contiennent
		    du parallélisme

	    \end {itemize}

	\item est une voie naturelle d'expression de certains problèmes

    \end {itemize}

    \vspace* {3mm}

    Ce cours a pour objectif de vous apprendre à déjouer les erreurs
    de programmation et pour vous donner des méthodes

\end {frame}

%%%%%%%%%%%%%%%%%%%%%%%%%%%%%%%%%%%%%%%%%%%%%%%%%%%%%%%%%%%%%%%%%%%%%%%%%%%%%%
% Notions d'architectures parallèles
%%%%%%%%%%%%%%%%%%%%%%%%%%%%%%%%%%%%%%%%%%%%%%%%%%%%%%%%%%%%%%%%%%%%%%%%%%%%%%

\titreB {Notions d'architectures parallèles}

\begin {frame} {Notion d'architectures parallèles}
    Stabilisation des ordinateurs séquentiels autour de~:

    \begin {itemize}
	\item Architecture de Von Neumann (EDVAC, 1945)\\
	    Mémoire commune pour les instructions et les données
	\item Architecture Harvard (Mark I de l'U Harvard, 1944) \\
	    Mémoire séparée pour les instructions et les données
    \end {itemize}

\end {frame}

\begin {frame} {Notion d'architectures parallèles}
    Plusieurs niveaux de parallélisme possibles~:

    \begin {itemize}
	\item job~: capacity computing

	\item programme~: plusieurs CPU pour un problème unique

	\item instruction~: pipeline, unités fonctionnelles multiples,
	    architectures superscalaires, etc

	\item bit~: conception interne des unités fonctionnelles (ex: ALU)
    \end {itemize}
    \implique implémentations différentes du parallélisme
    \\
    \implique foisonnement d'architectures parallèles

    \vspace* {3mm}

    Tous les ordinateurs actuels comportent du parallélisme
\end {frame}

\begin {frame} {Classification de Flynn}
    Classification de Flynn (1972)~:

    \begin {quote}
	\begin {tabular} {|l|l|c|c|} \cline {3-4}
	    \multicolumn {2} {c|} {~} &
	    		\multicolumn {2} {c|} {Flot de données} \\ \cline {3-4}
	    \multicolumn {2} {c|} {~} &
			Simple & Multiple \\ \hline
	    Flot & Simple &
	    		\textbf {SISD} & \textbf {SIMD} \\ \cline {2-4}
	    d'instructions & Multiple &
			\textbf {MISD} & \textbf {MIMD} \\ \hline
	\end {tabular}
    \end {quote}

    \begin {itemize}
	\item Flot d'instructions~: les unités de traitement
	    exécutent toutes la même instruction, ou des instructions
	    différentes

	\item Flot de données~: chaque unité de traitement travaille
	    sur la même donnée, ou sur des données différentes

    \end {itemize}
    \vspace* {3mm}
    Beaucoup d'autres tentatives de classification depuis \\
    \implique peu convaincantes
\end {frame}

\begin {frame} {Classification de Flynn -- SISD}
    Single Instruction Stream, Single Data Stream

    \begin {itemize}
	\item une instruction unique...
	\item ... qui travaille sur une donnée unique...
	\item ... c'est l'ordinateur séquentiel classique !
    \end {itemize}
\end {frame}

\begin {frame} {Classification de Flynn -- SIMD}
    Single Instruction Stream, Multiple Data Stream

    \vspace* {3mm}

    Appliquer la même opération sur un ensemble de données~:
    \begin {itemize}
	\item calcul matriciel \\
	    Ex: en Fortran 90~: \code {A(1,1:10) = B(:) + C(1:20:2)}
	\item traitement d'images
	\item traitement audio
	\item etc.
    \end {itemize}

    \vspace* {2mm}

    Mise en {\oe}uvre~:
    \begin {itemize}
	\item MasPar MP-1 (1990), MP-2 (1992)
	\item Connection Machine CM-1 (64k proc, 1985), CM-2 (1987)
	\item etc.
    \end {itemize}

\end {frame}

\begin {frame} {Classification de Flynn -- SIMD}
    Plus récemment~: extensions vectorielles pour architecture x86

    \ctableau {\fB} {|c|l|c|} {
	\rca \textbf {Nom} & \textbf {Signification} & \textbf {Date} \\
	\rcb MMX & MultiMedia eXtension & 1997 \\
	\rca SSE & Streaming SIMD Extension & 1999 \\
	\rcb SSE2 & & 2001 \\
	\rca SSE3 & & 2004 \\
	\rcb SSSE3 & Supplemental Streaming SIMD Extension & 2006 \\
	\rca SSE4 & & 2007 \\
	\rcb AVE & Advanced Vector Extensions & 2011 \\
    }

    \vspace* {3mm}
    \implique évolutions successives (et non changement radical)
    
\end {frame}

\begin {frame} {Classification de Flynn -- SIMD}
    Exemple : extension SSE sur x86

    \begin {itemize}
	\item 8 registres de 128 bits (XMM0 à XMM7)
	\item chaque registre = 4 flottants sur 32 bits
	\item opérations sur des registres = opérations sur 4 flottants
	\item ex~: l'instruction ADDPS additionne 2 registres :

	    \begin {center}
		\includegraphics [width=.6\textwidth] {\inc/intel-sse}
	    \end {center}

	    \implique addition des 4 valeurs de XMM1 avec les 4 de XMM5
	    en une seule instruction du processeur
	    \\
	    (ADDPS = Add Packed Single precision)

    \end {itemize}
\end {frame}

\begin {frame} {Classification de Flynn -- MISD}
    Multiple Instruction Stream, Single Data Stream

    \vspace* {3mm}

    Plusieurs instructions sur la même donnée~?

    \begin {itemize}
	\item pas d'exemple réellement mis en {\oe}uvre
	\item cas d'école~: les \emph {pipelines} internes de processeurs

	    \begin {itemize}
		\item souvent vu comme une approximation de MISD

		\item pas vraiment MISD, car la donnée change entre
		    les unités fonctionnelles

		\item mais intéressant à étudier \implique alors,
		    pourquoi pas ici ?

	    \end {itemize}
    \end {itemize}

\end {frame}

\begin {frame} {Zoom sur le \emph {pipeline}}
    Exemple~: pipeline du processeur Intel 486 (1989)

    \vspace* {3mm}

    Cinq unités fonctionnelles travaillent en parallèle~:

    \begin {itemize}
	\item FI~: Fetch Instructions \\
	    {\small \implique lit une ligne (16 octets) depuis le cache
	    d'instructions}

	\item D1~: Main Instruction Decode \\
	    {\small \implique première partie du décodage d'une instruction
	    et préparation des actions de D2}

	\item D2~: Secondary Instruction Decode \\
	    {\small \implique fin du décodage d'instruction et calcul des
	    adresses des opérandes en mémoire}

	\item EX~: Execute \\
	    {\small \implique exécution d'une instruction (une instruction
	    complexe peut requérir plusieurs cycles de EX)}

	\item WB~: Write Back \\
	    {\small \implique écrit le résultat dans un registre ou
	    dans le cache}

    \end {itemize}

\end {frame}

\begin {frame} {Zoom sur le \emph {pipeline}}
    Exécution d'une instruction~: chaque étage du \emph {pipeline}
    prend un cycle d'horloge

    \begin {center}
	\includegraphics [width=.7\textwidth] {\inc/pipe-486a}
    \end {center}

    \vspace* {3mm}

    Lorsque le \emph {pipeline} est plein, une nouvelle instruction
    finit de s'exécuter à chaque cycle

    \begin {center}
	\includegraphics [width=.7\textwidth] {\inc/pipe-486b}
    \end {center}

\end {frame}

\begin {frame} {Zoom sur le \emph {pipeline}}
    Dans la réalité~:

    \begin {itemize}
	\item la lecture d'une ligne du cache n'est pas systématiquement
	    nécessaire
	\item certaines instructions nécessitent plusieurs EX \\
	    (ex: \code {ADD mem,reg} \implique plusieurs accès à la mémoire)

    \end {itemize}

    \begin {center}
	\includegraphics [width=.7\textwidth] {\inc/pipe-486c}
    \end {center}

\end {frame}

\begin {frame} {Zoom sur le \emph {pipeline}}
    Exemples de \emph {pipelines}~:

    \begin {itemize}
	\item Intel 486 (1989)~: cf précédemment
	\item Intel Pentium Pro (1995)~: 3 instructions par cycle \\
	    \implique 3 \emph {pipelines} en parallèle
	\item ARM A8~: 2 \emph {pipelines} de 13 étages chacun \\
	    (4 fetch + 4 decode + 5 ALU)
	\item Intel Core i7 (2008)~: 14 étages

    \end {itemize}

\end {frame}

\begin {frame} {Zoom sur le \emph {pipeline}}
    Enjeux des \emph {pipelines}~: conserver le \emph {pipeline}
    plein

    \vspace* {5mm}

    Exemples d'optimisations (Pentium Pro, 1995)~:

    \begin {itemize}
	\item \textbf {exécution spéculative}~: simuler l'exécution
	    d'instructions, et sauver les résultats en fonction
	    d'un test précédent
	\item \textbf {prédiction de branchement}~: prédire le cas
	    courant \\
	    \implique exemple~: test de contrôle d'une boucle \code {while}
	\item \textbf {\emph {out of order execution}}~: changer l'ordre
	    des instructions dans le \emph {pipeline} pour optimiser son
	    rendement
	    \\
	    \implique en conservant la sémantique du programme

    \end {itemize}

    \implique optimisations internes au processeur (en matériel)
\end {frame}

\begin {frame} {Classification de Flynn -- MIMD}
    Multiple Instruction Stream, Multiple Data Stream

    \vspace* {3mm}

    Regroupe la plupart des architectures parallèles~:

    \begin {itemize}
	\item stations de travail en réseau (clusters, grilles de calcul)
	\item ordinateurs massivement parallèles
	\item serveurs multi-processeurs
	\item etc.
    \end {itemize}

    \implique besoin d'une «~sous-classification~»

    \vspace* {3mm}

    Deux modèles extrêmes~:

    \begin {itemize}
	\item architectures à mémoire distribuée
	\item architectures à mémoire partagée (ou mémoire globale)
    \end {itemize}

\end {frame}


\begin {frame} {Classification de Flynn -- MIMD}
    Architectures à mémoire distribuée~:

    \begin {center}
	\includegraphics [width=.6\textwidth] {\inc/mem-dist}
    \end {center}

    \begin {itemize}
	\item chaque processeur dispose de sa propre mémoire
	\item le réseau d'interconnexion peut être~:
	    \begin {itemize}
		\item un réseau spécialisé pour machine parallèle \\
		    (cross-bar, grille, full-mesh, hypercube, Myrinet, etc.)
		\item un réseau généraliste de type Ethernet \\
		    \implique grilles de calcul, cluster de calcul
	    \end {itemize}
	\item communication par passage de messages
    \end {itemize}

\end {frame}

\begin {frame} {Classification de Flynn -- MIMD}
    Architectures à mémoire partagée~:

    \begin {center}
	\includegraphics [width=.6\textwidth] {\inc/mem-glob}
    \end {center}

    \begin {itemize}
	\item la mémoire est partagée par tous les processeurs
	    \begin {itemize}
		\item un seul espace d'adressage pour tous les processeurs
		\item éventuellement plusieurs bancs mémoire (performances)
	    \end {itemize}
	\item le réseau d'interconnexion peut être le bus de la machine
	\item communication par l'intermédiaire de la mémoire
	\item modèle non extensible au delà de quelques dizaines de
	    processeurs
	\item ordinateurs SMP = \emph {Symmetric Multi-Processing}
    \end {itemize}

\end {frame}

\begin {frame} {Classification de Flynn -- MIMD}
    Architecture hybride~: mémoire partagée distribuée

    \begin {itemize}
	\item DSM~: \emph {Distributed shared memory}
	\item chaque processeur a sa propre mémoire \\
	    \implique mémoire physiquement distribuée
	\item mécanisme (logiciels / matériels) pour donner
	    au programme l'illusion d'une mémoire unique \\
	    \implique mémoire logiquement partagée

	\item temps d'accès non uniforme \\
	    NUMA~: \emph {Non-Uniform Memory Access}

	    (par opposition aux architectures à mémoire partagée qui
	    ont un temps d'accès uniforme~: UMA)
    \end {itemize}

\end {frame}

\begin {frame} {Classification de Flynn -- MIMD}
    Objet de ce cours~: architectures à mémoire partagée

    \vfill

    Architectures à mémoire distribuée~: cf cours de Systèmes
    Distribués au semestre 6

\end {frame}

\begin {frame} {Classification de Flynn -- MIMD}
    Composants d'un ordinateur à mémoire partagée~:

    \begin {center}
	\includegraphics [width=.6\textwidth] {\inc/arch-shm}
    \end {center}

    \begin {itemize}
	\item arbitre de bus~:
	    \begin {itemize}
		\item arbitre l'accès au bus \\
		    \implique un seul processeur a accès au bus
		    à un instant donné
		\item peut être embarqué dans les processeurs \\
		    \implique arbitrage distribué
	    \end {itemize}
	\item cache~:
	    \begin {itemize}
		\item chaque processeur a son propre cache
		\item chaque cache «~écoute~» le bus et invalide
		    une donnée si un autre processeur modifie la
		    case correspondante
	    \end {itemize}
    \end {itemize}
\end {frame}

\begin {frame} {Classification de Flynn -- MIMD}
    Et les processeurs multi-c{\oe}urs~?

    \vspace* {3mm}
    Intel a introduit deux termes~:
    \begin {itemize}
	\item \emph {hyperthreading} (Pentium 4, 2000)
	\item \emph {multicore} (à partir du Pentium Extreme Edition, 2005)
    \end {itemize}

    \vspace* {3mm}

    \implique technologies permises par l'augmentation du nombre de
    transistors sur la puce (Loi de Moore)

\end {frame}


\begin {frame} {Classification de Flynn -- MIMD}
    Hyperthreading~:

    \vspace* {3mm}

    \begin {minipage} {.40\textwidth}
	\includegraphics [width=\textwidth] {\inc/intel-ht}
    \end {minipage}
    \begin {minipage} {.58\textwidth}
	\begin {itemize}
	    \item deux (ou plus) ensembles de registres contiennent l'état
		du processeur «~virtuel~»

	    \item les unités fonctionnelles (UF = additionneur, pipeline, etc.)
		sont communes

	    \item illusion de deux (ou plus) processeurs sur le même
		carré de silicium

	\end {itemize}

    \end {minipage}

    \vspace* {3mm}

    Si une instruction du processeur virtuel 1 requiert l'UF
    «~additionneur~», qui est utilisé par le processeur virtuel 2,
    le 1 attend.

\end {frame}

\begin {frame} {Classification de Flynn -- MIMD}
    Multic{\oe}urs~:

    \vspace* {3mm}

    \begin {minipage} {.40\textwidth}
	\includegraphics [width=\textwidth] {\inc/intel-mc}
    \end {minipage}
    \begin {minipage} {.58\textwidth}
	\begin {itemize}
	    \item unités fonctionnelles (regroupées dans le
		«~moteur d'exécution~») propres à chaque
		c{\oe}ur

	    \item certains composants restent uniques sur le
		processeur physique

	\end {itemize}

    \end {minipage}

    \vspace* {3mm}

    Moins bonne utilisation du silicium (des unités fonctionnelles
    peuvent être inutilisées à certains moments), mais meilleures
    performances

\end {frame}

\begin {frame} {Classification de Flynn -- MIMD}
    Exemple du processeur Intel Core i7 (2011)~:

    \begin {center}
	\includegraphics [width=.9\textwidth] {\inc/intel-i7}

	\centerline {\tiny D'après «~Intel 64 and IA-32 Architectures Software Developer's Manual, vol 1~»}
    \end {center}

    \begin {itemize}
	\item 4 c{\oe}urs sur le processeur physique
	\item chaque c{\oe}ur est hyperthreadé avec 2 processeurs virtuels
    \end {itemize}

\end {frame}

\begin {frame} {Classification de Flynn -- MIMD}
    Cas courant~: un ordinateur parallèle, c'est~:

    \begin {itemize}
	\item une carte mère supportant plusieurs processeurs physiques
	\item chaque processeur physique comporte plusieurs processeurs
	    virtuels (hyperthreadés et/ou multic{\oe}urs)
    \end {itemize}

    De plus, on peut combiner ce type d'ordinateurs avec un réseau
    d'interconnexion pour réaliser une grille de calcul
\end {frame}

\begin {frame} {Synthèse}
    Les architectures parallèles présentent une grande diversité

    \vfill

    Le cours de systèmes concurrents est centré sur les architectures
    disposant d'une mémoire partagée, quelque soit le contexte
    (multic{\oe}urs, processeurs distincts, etc.)

\end {frame}


%%%%%%%%%%%%%%%%%%%%%%%%%%%%%%%%%%%%%%%%%%%%%%%%%%%%%%%%%%%%%%%%%%%%%%%%%%%%%%
% Quantification
%%%%%%%%%%%%%%%%%%%%%%%%%%%%%%%%%%%%%%%%%%%%%%%%%%%%%%%%%%%%%%%%%%%%%%%%%%%%%%

\titreB {Quantification}

\begin {frame} {Quantification}
    Comment quantifier les aspects liés au parallélisme~?

    \begin {itemize}
	\item grain
	\item latence
	\item accélération (\emph {speedup\/})
	\item rendement
    \end {itemize}
\end {frame}

\begin {frame} {Quantification -- Grain}
    Grain = taille des actions s'exécutant en parallèle \\
    \implique limitées par des communications ou des synchronisations

    \begin {itemize}
	\item grain fin (1 à 10 instructions)

	    \begin {itemize}
		\item Ex: \code {A(1,1:10) = B(:) + C(1:20:2)}
		\item Les 10 cases de \code {A} peuvent être calculées en
		    parallèle
	    \end {itemize}
	    \implique facile à gérer (matériel)

	\item gros grain (au delà d'un millier d'instructions environ)

	    \begin {itemize}
		\item Ex: \code {send\_page\_web (request)}
		\item Traitement de plusieurs requêtes Web en parallèle
	    \end {itemize}
	    \implique gestion par le programmeur

	% \item grain ultra-fin (< 1 instruction)~: \emph {pipeline}

    \end {itemize}

\end {frame}

\begin {frame} {Quantification -- Latence}
    Latence de communication / de synchronisation~: temps nécessaire pour
    réaliser une communication (de taille nulle) ou une synchronisation
    entre processeurs

    \vspace* {3mm}

    La latence est une caractéristique de l'architecture matérielle~:

    \begin {itemize}
	\item les architectures à mémoire distribuée ont en général
	    une latence de communication importante
	    \\
	    (une synchronisation consiste en une communication)

	\item les architectures à mémoire centralisée ont en général
	    une faible latence de synchronisation
	    \\
	    communication = accès mémoire \implique latence minime

    \end {itemize}

\end {frame}

\begin {frame} {Quantification -- Grain et latence}
    Grain et latence sont liés~:

    \begin {itemize}
	\item plus la latence est élevée :
	    \begin {itemize}
		\item plus on cherche à rendre les tâches indépendantes
		\item moins on communique de données / moins on synchronise
	    \end {itemize}
	    \implique plus le grain est gros

	\item à l'inverse, moins la latence est élevée~:

	    \begin {itemize}
		\item moins ça coûte cher de synchroniser les actions
	    \end {itemize}
	    \implique plus on peut se permettre d'affiner le grain

    \end {itemize}
\end {frame}

\begin {frame} {Quantification -- Grain et latence}

    Le grain est à la fois~:

    \begin {itemize}
	\item une caractéristique de l'architecture matérielle
	    \\
	    \implique dépend de la latence

	\item une caractéristique du problème
	    \\
	    \implique structuration en actions indépendantes

    \end {itemize}

\end {frame}

\begin {frame} {Quantification -- Accélération / \emph {Speedup}}
    Comment quantifier le gain de performances résultant de l'utilisation
    du parallélisme~?

    \vspace* {3mm}

    \begin {quote}
	Accélération (\emph {Speedup\/}) : $S(n)=T_s / T_p(n)$
    \end {quote}

    Explication~:
    \begin {itemize}
	\item $T_s$ = temps de la version séquentielle du programme
	\item $T_p (n)$ = temps de la version parallèle du programme
	    sur $n$ processeurs

    \end {itemize}

    \vspace* {3mm}
    \small
    Note~: le \emph {speedup} peut s'appliquer à d'autres
    domaines.
    \\
    Exemple~: $S=T_m / T_c$ mesure l'accélération entre $T_m$ (temps
    mis pour parcourir une certaine distance en marchant) et $T_c$ (temps
    mis pour cette même distance en courant).
    \\
    $S = 2$ \implique je vais 2 fois plus vite en courant qu'en marchant

\end {frame}

\begin {frame} {Quantification -- Accélération / \emph {Speedup}}
    Exemple~:
    soit un programme composé de 3 actions ($a_1$, $a_2$, $a_3$) de 10
    ms chacune

    \begin {itemize}
	\item $T_s = 30$ ms
	\item sur 2 processeurs, $T_p (2) = 20$ ms
	    \begin {itemize}
		\item processeur $P_1$ exécute $a_1$ et $a_3$
		\item processeur $P_2$ exécute $a_2$
	    \end {itemize}
	\item donc, $S(2) = 30/20 = 1,5$
	\item sur 4 processeurs, $T_p(3) = 10$ ms, donc $S(3) = 30/10 = 3$
    \end {itemize}

    \vspace* {3mm}

    Dans cet exemple, on suppose qu'on n'a pas d'\emph {overhead} de
    synchronisation entre les actions (rarement le cas...)

\end {frame}

\begin {frame} {Quantification -- Accélération / \emph {Speedup}}
    \begin {minipage} {.53\textwidth}
	\small
	\begin {itemize}
	    \item $S(n) = n$ : \emph {speedup} linéaire, cas optimal
	    \item $S(n) \in [1..n]$ : cas standard
	    \item $S(n) < 1$ : version parallèle plus
		lente que version séquentielle
		\\
		\implique problème !
	    \item $S(n) > n$ : exceptionnel !
		\begin {itemize}
		    \item algorithmes séquentiel et parallèle
			différents

		    \item effet de cache
		    \item ...
		\end {itemize}

	\end {itemize}
    \end {minipage}
    \hfill
    \begin {minipage} {.45\textwidth}
	\includegraphics [width=\textwidth] {\inc/speedup}
    \end {minipage}
\end {frame}

\begin {frame} {Quantification -- Loi d'Amdahl}
    Loi d'Amdahl~: borne supérieure de l'accélération

    \begin {quote}
	Si un programme comporte une fraction $f$ (avec $0 \leq f \leq
	1$) de son temps d'exécution séquentiel qui ne peut être
	parallélisée, alors $\forall n > 0, S(n) \leq 1/f$

    \end {quote}
\end {frame}

\begin {frame} {Quantification -- Loi d'Amdahl}

    Exemple~:
    \begin {center}
	\includegraphics [width=.6\textwidth] {\inc/amdahl}
    \end {center}

    D'où $S(4) = 5/2 = 2,5$ \\
    La loi d'Amdahl prédit que $S(n) \leq 5, \forall n$

\end {frame}

\begin {frame} {Quantification -- Efficacité}
    Est-ce qu'il est intéressant d'utiliser un ordinateur parallèle~?

    \vspace* {3mm}

    \begin {quote}
	Efficacité (ou rendement) : $E(n)=\frac {T_s} {n T_p(n)}$
    \end {quote}

    \implique taux d'utilisation des processeurs

    \vspace* {3mm}

    Exemple précédent~: $E(4) = 5 / (4 \times 2) = 5/8 = 0,625$

    \small
    \begin {itemize}
	\item normalement, $E(n) \in [0..1]$
	\item plus $E(n)$ est proche de 1, meilleure est l'utilisation
	    des processeurs
	\item plus $E(n)$ est proche de 0, moins les processeurs sont
	    utilisés
	\item $E(n) > 1$ : \emph {speedup} superlinéaire
    \end {itemize}

\end {frame}

%%%%%%%%%%%%%%%%%%%%%%%%%%%%%%%%%%%%%%%%%%%%%%%%%%%%%%%%%%%%%%%%%%%%%%%%%%%%%%
% Parallélisme caché
%%%%%%%%%%%%%%%%%%%%%%%%%%%%%%%%%%%%%%%%%%%%%%%%%%%%%%%%%%%%%%%%%%%%%%%%%%%%%%

\titreB {Parallélisme caché}

\begin {frame} {Parallélisme caché}
    Le matériel \emph {cache} souvent le parallélisme~:

    \begin {itemize}
	\item architectures superscalaires
	\item bancs mémoire
	\item \emph {pipeline}
	\item multiples unités fonctionnelles pour l'exécution d'une
	    instruction
    \end {itemize}

    Cacher le parallélisme \implique préserver la sémantique
    «~séquentielle~» exprimée dans le programme

\end {frame}

\begin {frame} {Parallélisme apparent}
    Si on souhaite exploiter du parallélisme à plus haut niveau,
    il faut l'expliciter~:

    \begin {itemize}
	\item avec le système~: mémoire partagée, processus, threads, etc.
	\item avec un langage~: Fortran 90, OpenMP, etc.
    \end {itemize}

    ... ou alors recourir à la parallélisation automatique...
\end {frame}

%%%%%%%%%%%%%%%%%%%%%%%%%%%%%%%%%%%%%%%%%%%%%%%%%%%%%%%%%%%%%%%%%%%%%%%%%%%%%%
% Parallélisme virtuel
%%%%%%%%%%%%%%%%%%%%%%%%%%%%%%%%%%%%%%%%%%%%%%%%%%%%%%%%%%%%%%%%%%%%%%%%%%%%%%

\titreB {Parallélisme réel/virtuel}

\begin {frame} {Parallélisme réel/virtuel}
    Deux sortes de parallélisme~:
    \begin {itemize}
	\item réel~: offert par le matériel
	\item virtuel~: offert par le système d'exploitation \\
	    (multiples processus, multiples threads)
    \end {itemize}

    \vspace* {3mm}

    \implique problèmes de concurrence identiques dans les
    deux cas \\
    Condition~: partage de mémoire entre fils d'exécution distincts

\end {frame}
