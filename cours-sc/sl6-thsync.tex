%
% Historique
%   2014/08/30 : pda : conception
%

\def\inc{inc6-thsync}

\titreA {Autres mécanismes de synchronisation de threads}

%%%%%%%%%%%%%%%%%%%%%%%%%%%%%%%%%%%%%%%%%%%%%%%%%%%%%%%%%%%%%%%%%%%%%%%%%%%%%%
% Introduction
%%%%%%%%%%%%%%%%%%%%%%%%%%%%%%%%%%%%%%%%%%%%%%%%%%%%%%%%%%%%%%%%%%%%%%%%%%%%%%

\titreB {Introduction}

\begin {frame} {Introduction}
    Mécanismes de synchronisation de threads déjà rencontrés~:

    \begin {enumerate}
	\item \emph {mutex}~: mécanisme de base
	\item \emph {spin}~: analogue au \emph {mutex}, mais avec
	    attente active
	\item sémaphore POSIX (non spécifique aux threads)
    \end {enumerate}

    Les threads offrent deux autres mécanismes~:
    
    \begin {enumerate}
	\addtocounter {enumi} {3}
	\item conditions (\emph {condition variables\/})
	\item verrous lecteur/écrivain
    \end {enumerate}

\end {frame}

%%%%%%%%%%%%%%%%%%%%%%%%%%%%%%%%%%%%%%%%%%%%%%%%%%%%%%%%%%%%%%%%%%%%%%%%%%%%%%
% Conditions
%%%%%%%%%%%%%%%%%%%%%%%%%%%%%%%%%%%%%%%%%%%%%%%%%%%%%%%%%%%%%%%%%%%%%%%%%%%%%%

\titreB {Conditions}

\begin {frame} {Conditions}
    Problème de synchronisation fréquent~:

    \begin {itemize}
	\item un thread attend un événement ou une condition
	\item la condition est matérialisée par une variable
	\item un autre thread signale que l'événement est arrivé
	    ou que la condition est remplie
    \end {itemize}
\end {frame}

\begin {frame} {Conditions}
    Exemple de solution~:

    \begin {itemize}
	\item le premier thread attend que la condition soit réalisée~:

	    \begin {tabbing}
		\hspace* {10mm} \= \hspace* {5mm} \= \kill
		\> cond $\leftarrow$ 0 \\
		\> tant que cond = 0 \\
		\> \> ;
	    \end {tabbing}

	\item lorsque la condition est réalisée, le deuxième thread 
	    fait~:

	    \begin {tabbing}
		\hspace* {10mm} \= \hspace* {5mm} \= \kill
		\> cond $\leftarrow$ 1 \\
	    \end {tabbing}

    \end {itemize}

    \vspace* {3mm}

    \implique cette solution utilise l'attente active !

    \vspace* {3mm}

    Exercice~: implémenter cela avec des \emph {mutex} POSIX pour
    réaliser une attente passive...

\end {frame}

\begin {frame} [fragile] {Conditions}
    Solution simple~: utiliser les conditions POSIX

    \lstinputlisting [basicstyle=\scriptsize\lstmonstyle, firstline=4, lastline=21] {\inc/cond.c}
\end{frame}

\begin {frame} {Conditions}
    \begin {itemize}
	\item une condition est forcément associée à un \emph {mutex} \\
	    \implique l'accès à la variable est protégé par le \emph {mutex}

	\item le \emph {mutex} n'est pas \textit {explicitement}
	    déverrouillé lors de l'attente

	    \begin {itemize}
		\item déverrouillage implicite pendant
		    \code {pthread\_cond\_wait}
		\item re-verrouillage implicite à la sortie de
		    \code {pthread\_cond\_wait}
		\item \implique évite les sections critiques

	    \end {itemize}

	\item le réveil peut être provoqué par un autre événément \\
	    \implique toujours vérifier la raison du réveil \\
	    \implique cf \code {while (var == 0)} dans l'exemple

    \end {itemize}
\end {frame}

\begin {frame} {Conditions}
    \begin {itemize}
	\item \code {\small int pthread\_cond\_init (pthread\_cond\_t *, pthread\_condattr\_t *)}
	    \\
	    ou utilisation de \code {\small {PTHREAD\_COND\_INITIALIZER}}
	\item \code {\small int pthread\_cond\_destroy (pthread\_cond\_t *)}
	\item \code {\small int pthread\_cond\_wait (pthread\_cond\_t *, pthread\_mutex\_t *)}
	\item \code {\small int pthread\_cond\_timedwait (pthread\_cond\_t *, pthread\_mutex\_t *, struct timespec *habs)}

	    Avec \code {\small struct timespec \{ time\_t tv\_sec; long tv\_nsec; \}}
	\item \code {\small int pthread\_cond\_signal (pthread\_cond\_t *)}
	\item \code {\small int pthread\_cond\_broadcast (pthread\_cond\_t *)}
    \end {itemize}
\end {frame}


%%%%%%%%%%%%%%%%%%%%%%%%%%%%%%%%%%%%%%%%%%%%%%%%%%%%%%%%%%%%%%%%%%%%%%%%%%%%%%
% Lecteurs/écrivains
%%%%%%%%%%%%%%%%%%%%%%%%%%%%%%%%%%%%%%%%%%%%%%%%%%%%%%%%%%%%%%%%%%%%%%%%%%%%%%

\titreB {Verrous lecteurs/écrivains}

\begin {frame} [fragile] {Lecteurs/écrivains}
    Problème classique de synchronisation lecteurs/écrivains~: \\
    \implique solution simple avec les threads POSIX

    \lstinputlisting [basicstyle=\scriptsize\lstmonstyle, firstline=4] {\inc/rw.c}

    \begin {itemize}
	\item implémentation : priorité aux écrivains
    \end {itemize}
\end{frame}

\begin {frame} {Verrous lecteurs/écrivains}
    \begin {itemize}
	\item \code {\small int pthread\_rwlock\_init (pthread\_rwlock\_t *, pthread\_rwlockattr\_t *)}
	\item \code {\small int pthread\_rwlock\_destroy (pthread\_rwlock\_t *)}
	\item \code {\small int pthread\_rwlock\_rdlock (pthread\_rwlock\_t *, pthread\_mutex\_t *)}
	\item \code {\small int pthread\_rwlock\_wrlock (pthread\_rwlock\_t *, pthread\_mutex\_t *)}
	\item \code {\small int pthread\_rwlock\_tryrdlock (pthread\_rwlock\_t *, pthread\_mutex\_t *)}
	\item \code {\small int pthread\_rwlock\_trywrlock (pthread\_rwlock\_t *, pthread\_mutex\_t *)}
	\item versions avec \code {timedrdlock} et \code {timedwrlock}
	\item \code {\small int pthread\_rwlock\_unlock (pthread\_rwlock\_t *)}
    \end {itemize}
\end {frame}
