\RequirePackage [l2tabu, orthodox] {nag}

\documentclass {article}

    \usepackage [utf8] {tdpda}

\begin {document}

\entete
    {Systèmes d'Exploitation}
    {Licence d'informatique -- Semestre 4}
    {2015 -- 2016 (printemps)
}
    {Conseils pour réussir vos TP et projets}

%%%%%%%%%%%%%%%%%%%%%%%%%%%%%%%%%%%%%%%%%%%%%%%%%%%%%%%%%%%%%%%%%%%%%%%%%%%%%%

\section {Introduction}

L'UE de Systèmes d'Exploitation nécessite que vous mettiez en pratique
vos connaissances par la rédaction de nombreux programmes en langage C,
dont une bonne partie sera soumise à évaluation. Vous devez bien sûr
appliquer les méthodes et consignes apprises dans l'UE «~Techniques
de Développement~» (L2S3). En complément, ce petit guide a pour but
de vous indiquer quelques règles basiques et de bon sens pour vous
permettre d'obtenir la meilleure note possible\footnote {\copyright
Pierre David. Disponible sur \url {http://github.com/pdav/ens}.
Ce texte est placé sous licence « Creative Commons Attribution
-- Pas d’Utilisation Commerciale 4.0 International ».
Pour accéder à une copie de cette licence,
merci de vous rendre à l'adresse suivante
\url {http://creativecommons.org/licenses/by-nc/4.0/}}.

Bon nombre d'organisations (entreprises, logiciels libres)
ont rédigé des «~standards de programmation~», composés de
règles plus ou moins arbitraires dont le but est de faciliter
l'écriture et la maintenance des logiciels.  On trouve de
tels guides de style de programmation en ligne\footnote {\url
{http://www.maultech.com/chrislott/resources/cstyle/}}
comme en particulier ceux de Google\footnote {\url
{https://code.google.com/p/google-styleguide/}} pour les différents
langages utilisés dans cette entreprise, ou encore
l'indispensable standard de programmation sécurisée\footnote {\url
{https://www.securecoding.cert.org/confluence/display/seccode/CERT+C+Coding+Standard}}
du CERT.  Le présent document n'ambitionne pas de remplacer de tels
guides, mais de les compléter par des règles de base.

%%%%%%%%%%%%%%%%%%%%%%%%%%%%%%%%%%%%%%%%%%%%%%%%%%%%%%%%%%%%%%%%%%%%%%%%%%%%%%

\section {La forme}

Que ce soit lors de vos études ou dans votre vie professionnelle
future, les programmes que vous écrivez seront inévitablement
lus par d'autres. Dès lors, la lisibilité de vos programmes
est cruciale\footnote {Il n'y a qu'un seul cas où la lisibilité
n'est pas requise, bien au contraire, c'est le concours IOCCC (\url
{http://www.ioccc.org/}). Ne prenez pas ça comme modèle pour vos
propres programmes...}. Dans le cadre de vos études, en particulier,
vous écrivez des programmes destinés à être évalués, c'est-à-dire
à être lus et compris par une autre personne que vous : la lisibilité
doit donc être une de vos priorités au même niveau que la correction
de vos programmes et le respect de l'énoncé.


\subsection {Soignez la cohérence du style et l'indentation}

On peut identifier plusieurs styles de placement des accolades dans
les programmes en C, comme par exemple~:

\begin {quote}
\begin {minipage} {.25\textwidth}
\begin {lstlisting}
if (...) {
    ...
} else {
    ...
}
\end{lstlisting}
\end {minipage}
\hfill
\begin {minipage} {.25\textwidth}
\begin {lstlisting}
if (...)
{
    ...
}
else
{
    ...
}
\end{lstlisting}
\end {minipage}
\hfill
\begin {minipage} {.25\textwidth}
\begin {lstlisting}
if (...) {
	...
    } else {
	...
}
\end{lstlisting}
\end {minipage}
\end {quote}

Ces styles de placement constituent le premier guide de
lecture. Choisissez le style que vous préférez, mais restez cohérent
tout au long d'un programme. De même, lorsque vous faites une addition
à un programme existant, utilisez le style général du programme et
n'introduisez pas d'incohérence.

L'indentation (le décalage pour visualiser l'imbrication des blocs) est
de toute première importance pour permettre au lecteur de comprendre
la structure de votre programme. Au premier coup d'oeil, une mauvaise
indentation donnera un sentiment d'amateurisme et n'incitera guère à
prolonger la lecture.

Quelle largeur d'indentation utiliser~? C'est à vous de choisir, mais 4
espaces constituent un bon compromis entre une bonne visualisation de la
structure de blocs et des lignes pas trop longues (voir \ref {lignes80}
ci-après).

Astuces~: le programme \texttt {indent}\footnote {\url
{http://www.gnu.org/software/indent/}} disponible sur turing, est
remarquablement configurable et vous permet de reformatter votre programme
en prenant en compte le style de placement des accolades et l'indentation
que vous aurez choisis.  N'hésitez pas à lire la documentation pour
le configurer (voir~\ref {indentpro}, page~\pageref {indentpro})
et à l'utiliser. Vous pouvez également utiliser un environnement
de développement intégré (IDE) qui vous aidera à maintenir une
indentation homogène.


\subsection {Ne changez pas la longueur des tabulations}

Le caractère «~tabulation~» a une signification héritée des machines
à écrire~: il provoque l'avancée du chariot jusqu'au prochain taquet
de tabulation\footnote {\url
{http://fr.wikipedia.org/wiki/Taquet_de_tabulation}}. Les premières
imprimantes ont par la suite utilisé ce caractère avec des tabulations
fixes, de longueur égale à 8 caractères. Cette convention est restée
jusqu'à nos jours, partagée entre terminaux, émulateurs de terminaux
et imprimantes.

Certains éditeurs permettent de changer la longueur de
tabulation\footnote {Par exemple, \texttt {vi} permet d'utiliser la
commande \texttt {set tabstop=4} (ou \texttt {set ts=4}). Il ne faut
pas utiliser cette commande, et laisser la valeur 8 par défaut. À
la place, vous pouvez utiliser \texttt {set shiftwidth=4} (ou \texttt
{set sw=4}) et utiliser les touches \texttt {CTRL-T} et \texttt {CTRL-D}
pour indenter ou «~dés-indenter~».}\footnote {Les utilisateurs de
\texttt {Geany} doivent configurer l'indentation comme suit~: menu
Éditer $\rightarrow$ Préférences, puis onglet Éditeur $\rightarrow$
Indentation~: Largeur=«~4~», Type=«~Tabulations et espaces~», Largeur
des tabulations=«~8~» et Indentation par la touche Tab=«~oui~».}~:
c'est une mauvaise pratique à éviter absolument pour au moins deux
raisons~:

\begin {itemize}
    \item la personne qui va vous relire n'a pas forcément les mêmes
	réglages que vous~;
    \item votre programme peut être imprimé, et l'imprimante utilise
	des tabulations de longueur 8.
\end {itemize}
Dans ces deux cas, votre programme apparaîtra alors au lecteur avec
des décalages fantaisistes et la lisibilité s'en ressentira. Une bonne
manière de vérifier votre indentation est d'afficher votre programme à
l'écran (avec \texttt {more} par exemple)~: ce que vous voyez correspond
à ce qui sera imprimé.


\subsection {Les lignes ne doivent pas dépasser 80 colonnes}
    \label {lignes80}

Votre programme ne doit pas contenir de ligne plus longue que 80 colonnes.
Pourquoi cette limite alors que vous avez un grand écran qui vous permet
d'afficher des lignes de 2365 colonnes ? La raison est simple~:
la personne qui vous relit n'a pas forcément un écran aussi grand que le
vôtre, ou peut préférer utiliser une police de caractères plus grosse.
Elle peut également vouloir imprimer votre programme, ce qui peut avoir
comme effet de tronquer les lignes supérieures à 80 colonnes.  Il est
donc important que votre programme respecte cette contrainte même si
vous devez pour cela découper des instructions sur plusieurs lignes.


\subsection {Documentez vos intentions}

La documentation doit être rédigée en même temps que votre programme.
Elle ne doit pas faire doublon avec le code mais elle doit expliquer
en termes de haut niveau votre intention, en supposant que le lecteur
n'a pas de difficulté avec le langage de programmation (évitez donc
le classique commentaire de type \verb|/* incrémentation de i */|).

Pour le formalisme, vous pourrez utiliser avec profit celui de \texttt
{doxygen}\footnote {\url {http://www.doxygen.org}}, comme par exemple~:

\begin {quote}
\begin {lstlisting}
/**
 * @brief Construit une liste des nombres premiers
 *
 * Cette fonction initialise une liste vide, puis parcourt tous les
 * entiers de `n` a `m`, determine pour chacun si c'est un nombre premier
 * et, si c'est le cas, l'ajoute a la liste. Pour supprimer la liste,
 * il est conseille d'utiliser la fonction `detruire_liste`.
 *
 * @param n limite inferieure (borne incluse) des nombres premiers
 * @param m limite superieure (borne incluse) des nombres premiers
 * @result une nouvelle liste
 */

struct element *liste_premiers (int n, int m)
{
    ...
}
\end{lstlisting}
\end {quote}

Comme pour n'importe quel texte, vérifiez bien sûr l'orthographe de
vos commentaires.

La documentation est une partie essentielle pour la compréhension de
votre programme. Vous souhaitez être compris, n'est-ce pas ?


\subsection {Découpez vos fichiers, mais pas trop}

Une bonne pratique de programmation consiste à découper un programme en
plusieurs fichiers pour constituer des ensembles cohérents de fonctions,
de classes, etc. Cependant, ne poussez pas cette idée à l'extrême :
pour de petits programmes de quelques dizaines à quelques centaines de
lignes~: ce n'est pas forcément nécessaire et peut risquer même de
nuire à la lisibilité, obligeant votre lecteur à jongler sans cesse
entre plusieurs fichiers.


\subsection {Copyright}

Inutile de placer dans vos exercices de TP ou dans vos projets des
mentions de copyright longues comme un jour sans pain. Outre le fait
que ça gâche beaucoup de papier si c'est imprimé, vos créations
restent entre vous et la personne qui évalue votre travail et ne sont
pas destinées à publication.


% \subsection {Français ou anglais ?}
% 
% Dans le cadre de vos études, vous devriez rédiger  le français.
% 
%     noms de procedures, de var, de champ, de macros, commentaires ?

%%%%%%%%%%%%%%%%%%%%%%%%%%%%%%%%%%%%%%%%%%%%%%%%%%%%%%%%%%%%%%%%%%%%%%%%%%%%%%

\section {Le langage C}

\subsection {Programmez portable}

Une des qualités du langage C est de permettre la rédaction de
programmes portables, c'est-à-dire donnant des résultats identiques
sur plusieurs systèmes et versions de systèmes différents. Cependant,
programmer portable n'est n'est sans doute pas la recommandation la plus
facile à suivre. Elle peut être frustrante car vous devrez sans doute
vous limiter dans l'utilisation des caractéristiques qui facilitent
la vie, mais qui ne sont disponibles que sur un nombre limité de
systèmes. Et elle nécessite de l'expérience car vous devez avoir la
connaissance de plusieurs systèmes ou environnements.

Si vous en avez la possibilité, observez le comportement de vos
programmes sur plusieurs systèmes différents (Linux, *BSD, MacOS, etc.).
Dans tous les cas, le point minimum à respecter est que vos programmes,
sauf contre-ordre, doivent pouvoir fonctionner correctement sur la machine
turing mise à votre disposition\footnote {Vous pouvez vous connecter
sur turing depuis les salles de TP, mais également à distance depuis
chez vous ou ailleurs via \texttt {ssh}. Vous pouvez en outre transférer
vos fichiers à l'aide de \texttt {scp} ou \texttt {sftp}.}.


\subsection {Quel standard choisir ?}

Le langage C a une longue histoire qui passe par plusieurs versions,
dont certaines sont normalisées~:

\begin {itemize}
    \item la version originale, dite K\&R (pour Kernighan et Ritchie,
	les deux concepteurs)~: cette version est aujourd'hui largement
	obsolète~;

    \item C89 (ou C90)~: la première version normalisée (en 1989 par
	l'ANSI et en 1990 par l'ISO), elle apporte les prototypes de
	fonctions~;

    \item C99~: normalisé par l'ANSI en 2000, apporte beaucoup
	d'améliorations comme les fonctions \texttt {inline}, les
	variables pouvant être placées dans le code et non plus
	seulement en début de bloc, les tableaux de taille variable ou
	les commentaires avec \verb|//|. Nombre de ces améliorations
	étaient déjà disponibles comme extensions (non portables)
	de compilateurs comme GCC~;

    \item C11~: dernière version, normalisée par l'ISO en 2011, n'est
	pas encore très répandue. Si vous voulez programmer portable,
	attendez encore quelques temps avant de l'utiliser.

\end {itemize}

Pour compliquer les choses, bon nombre de compilateurs supportent des
extensions spécifiques qu'il vaut mieux ne pas utiliser pour des raisons
évidentes de portabilité.

Quel standard choisir~? Dans le cadre de vos études, hors besoin
spécifique, le mieux est d'utiliser le compilateur par défaut
sur turing, dans la version de langage par défaut fournie par ce
compilateur (c'est à dire C89 avec extensions spécifiques dont
certaines sont compatibles avec C99). Si vous utilisez une version de
langage différente, n'oubliez pas de le préciser.


\subsection {Laissez le compilateur optimiser}

La plupart des compilateurs savent à présent bien optimiser le code
généré. Cela vous permet de programmer en privilégiant la lisibilité
plutôt qu'une optimisation hasardeuse, et vous pouvez ainsi vous
concentrer sur des algorithmes plus efficaces.


\subsection {Lisez les messages du compilateur}

Le compilateur vous donne de précieuses indications («~warning~»)
lorsqu'il compile. Ces messages sont à prendre au sérieux car ils
indiquent presque à coup sûr un problème dans votre programme. Ne
les ignorez pas, vous gagnerez ainsi un temps précieux !

Si vous utilisez GCC, vous devriez toujours spécifier deux options~:
\begin {itemize}
    \item \texttt {-Wall}~: détecte des problèmes potentiels qu'il
	faut presque toujours corriger. Encore mieux, utilisez
	\texttt {-Wextra} en complément pour détecter davantage
	de problèmes~;
    \item \texttt {-Werror}~: refuse la compilation en présence de
	«~warning~»
\end {itemize}

Ces deux options sont importantes. Ne négligez pas la deuxième~: elle
n'est pas là pour vous embêter, mais pour vous inciter à corriger votre
programme. Ce faisant, vous gagnerez du temps car plus un problème est
détecté tôt et plus il est facile à corriger.

\subsection {Pas de code dans des fichiers d'inclusion}

Les fichiers suffixés par «~\texttt {.h}~» sont des fichiers inclus
au moment de la compilation, leur contenu est compilé comme le reste
du fichier source. Ils servent à mettre des définitions (de macros,
de types, de prototypes de fonctions) qui peuvent servir dans plusieurs
autres fichiers. Ne mettez jamais de code exécutable (i.e. code d'une
fonction par exemple) dans un fichier «~\texttt {.h}~».


\subsection {Utilisez les bons termes}

Le langage C permet parfois plusieurs syntaxes pour atteindre le même
objectif. Par exemple, les deux boucles suivantes génèrent le même
code~:

\begin {quote}
\begin {minipage} {.4\textwidth}
\begin {lstlisting}
for (i = 0 ; i < N ; i++)
    printf ("%d\n", i) ;
\end{lstlisting}
\end {minipage}
\hfill
\begin {minipage} {.4\textwidth}
\begin {lstlisting}
i = 0 ;
while (i < N)
{
    printf ("%d\n", i) ;
    i++ ;
}
\end{lstlisting}
\end {minipage}
\end {quote}

Pourtant, elles ne sont pas équivalentes~: dans le cas d'une variable
parcourant un intervalle entre deux bornes, la boucle \texttt {for}
est beaucoup plus lisible. Dans la plupart des autres cas, cependant,
la boucle \texttt {while} sera mieux adaptée.

De la même manière, les formes \texttt {tab[i]} et \texttt {*(tab+i)}
génèrent le même code, mais ne sont pas équivalentes en termes
de lisibilité.


\subsection {Ne dépassez pas les bornes}

Le débordement de tampon est l'origine d'un des problèmes de sécurité
les plus répandus. Grâce au débordement de tampon, un pirate peut
injecter, par un moyen ou un autre, des données à un programme en vue
de saturer un espace mémoire fini (tableau, zone mémoire allouée avec
\texttt {malloc}, etc.) pour écraser la mémoire environnante et modifier
le comportement prévu du programme.  Vous devez donc prendre l'habitude
de vérifier que vous n'accédez jamais à la mémoire en dehors des
zones prévues, ce qui condamne l'utilisation des fonctions comme
\texttt {gets}, \texttt {strcpy}, \texttt {strcat}, \texttt {sprintf},
etc. au profit de fonctions plus modernes prenant en paramètre la taille
maximum des zones à modifier (comme \texttt {fgets}, \texttt {strncpy},
\texttt {strncat}, \texttt {snprintf}, etc.)


\subsection {Privilégiez la simplicité}

Entre deux solutions équivalentes, choisissez systématiquement la plus
simple si elle n'est pas moins efficace.

Attention toutefois aux boucles cachées~: ainsi, par exemple~:

\begin {quote}
\begin {lstlisting}
nb_a = 0 ;
for (i = 0 ; i < strlen (chaine) ; i++)
{
    if (chaine [i] == 'a')
	nb_a++ ;
}
\end{lstlisting}
\end {quote}

Cette boucle a une complexité en $O(n^2)$ où $n$ est la longueur de la
chaîne, car la fonction \texttt {strlen} réalise un parcours itératif
pour en déterminer la longueur à chaque itération. Il faut donc
calculer la longueur une fois pour toutes avant la boucle pour avoir
une complexité en $O(n)$.

%%%%%%%%%%%%%%%%%%%%%%%%%%%%%%%%%%%%%%%%%%%%%%%%%%%%%%%%%%%%%%%%%%%%%%%%%%%%%%

\section {Les tests}

\subsection {Un prérequis indispensable}

Rendre pour l'évaluation un programme non testé, c'est prendre un très
grand risque... Les tests constituent donc un prérequis indispensable
pour vos programmes. En fait, vous devriez commencer par rédiger des
scénarios de test avant même d'écrire la première ligne de votre
programme. Cela vous permettra également de mieux vous imprégner du
cahier des charges du programme à rédiger.

\subsection {Utilisez de grands jeux de données}

Si vous le pouvez, utilisez des jeux de données significatifs pour vos
tests. Par exemple, supposons que vous deviez rédiger un programme qui
compte le nombre de lignes. Testez-le non pas avec une ligne ou deux,
mais avec quelques milliers voire quelques millions. Cela vous permettra
de détecter plus rapidement les problèmes, que ce soit de performances
ou simplement de résistance.


\subsection {Automatisez vos jeux de tests}

Tester peut devenir fastidieux car il faut rejouer les nombreux tests
que vous avez conçus après chaque compilation pour vérifier que les
dernières modifications n'ont pas généré de régression. Il est
donc grandement recommandé d'automatiser les tests, via par exemple
un simple script shell qui enchaîne les tests et vérifie la présence
(ou l'absence) de motif dans la sortie.


\subsection {Mesurez la couverture de vos jeux de tests}

Au fur et à mesure que vous réalisez vos tests, la question est
de savoir s'ils sont suffisants (c'est-à-dire s'ils permettent de
parcourir toutes les lignes de votre programme) ou s'ils sont redondants
(c'est-à-dire si deux scénarios testent les mêmes lignes). Pour cela,
le compilateur GCC vous propose un outil intéressant~:

\begin {enumerate}
    \item compilez votre programme avec l'option \texttt {-coverage}
    \item passez tous vos jeux de tests
    \item utilisez l'utilitaire \texttt {gcov} pour générer des sources
	annotés avec le nombre d'utilisation de chaque ligne.

\end {enumerate}

Grâce au résultat de \texttt {gcov}, vous pouvez repérer les lignes
non exécutées, qui indiquent que la couverture des jeux de tests n'est
pas complète. Toute ligne non exécutée recèle potentiellement des
erreurs.  En recherchant la couverture maximale de vos jeux de tests,
vous améliorerez donc significativement la qualité de vos programmes.


\subsection {Utilisez systématiquement \texttt {valgrind}}

Les erreurs de mémoire sont fréquentes en C~: fuite de mémoire,
débordement de tampon, utilisation de mémoire déjà libérée,
etc. Pour vous aider à traquer ces erreurs, même si elles ne sont
pas visibles (c'est-à-dire que votre programme semble fonctionner
normalement), testez systématiquement avec l'utilitaire \texttt
{valgrind}.


%%%%%%%%%%%%%%%%%%%%%%%%%%%%%%%%%%%%%%%%%%%%%%%%%%%%%%%%%%%%%%%%%%%%%%%%%%%%%%

\section {Le rapport}

\subsection {Soignez la présentation et l'orthographe}

Cela devrait aller de soi, mais un rapport doit être préparé avec
soin, tant sur le plan de la présentation que de l'orthographe. Utilisez
systématiquement le correcteur intégré dans votre traitement de texte,
ou le programme \texttt {aspell} si vous utilisez \LaTeX.

Au delà de quelques pages (une petite dizaine par exemple), votre
rapport doit comporter un sommaire. Bien évidemment, les pages doivent
être numérotées.


\subsection {Expliquez les points importants, pas les détails}

Un rapport est un exercice difficile : vous êtes dans le feu de l'action,
peut-être même encore en train de déboguer votre programme, et vous
devez prendre de la hauteur pour rédiger un document de synthèse.

Un rapport doit permettre à une personne qui ne connaît pas votre
programme de se faire une idée de son fonctionnement. Il doit expliquer
les points importants et non les détails (pas de code, par exemple).


%%%%%%%%%%%%%%%%%%%%%%%%%%%%%%%%%%%%%%%%%%%%%%%%%%%%%%%%%%%%%%%%%%%%%%%%%%%%%%

% Plagiat
% Google : no

% \section {La remise}
% 
% \subsection {Respectez l'énoncé}
% 
% tar-gz

%%%%%%%%%%%%%%%%%%%%%%%%%%%%%%%%%%%%%%%%%%%%%%%%%%%%%%%%%%%%%%%%%%%%%%%%%%%%%%

\appendix

\section {Annexe}

\subsection {Fichier de configuration pour \texttt {indent}}
    \label {indentpro}

Le fichier ci-après est un exemple de fichier \verb|~/.indent.pro|
servant à configurer les nombreuses options de l'utilitaire \texttt
{indent}. Une fois installé, il suffit après de lancer \texttt
{indent} sur vos sources C pour les voir ré-indentés suivant ce qui
est spécifié ici.

\begin {quote}
    \lstinputlisting {indent.pro}
\end {quote}

\end {document}
