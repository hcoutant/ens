\def\inc{inc6-pipe}

\titreA {Gestion des tubes}

%%%%%%%%%%%%%%%%%%%%%%%%%%%%%%%%%%%%%%%%%%%%%%%%%%%%%%%%%%%%%%%%%%%%%%%%%%%%%%
% Introduction
%%%%%%%%%%%%%%%%%%%%%%%%%%%%%%%%%%%%%%%%%%%%%%%%%%%%%%%%%%%%%%%%%%%%%%%%%%%%%%

\titreB {Introduction}

\begin {frame} {Introduction}
    Les tubes sont une des innovations majeures d'Unix

    \begin {itemize}
	\item exemple en Shell : \code {\$ ls -l | wc -l}

	    \begin {center}
		\includegraphics [width=.8\linewidth] {\inc/principe}
	    \end {center}

	\item le processus \code {ls} écrit dans le tube
	    \begin {itemize}
		\item utilisation de la primitive \code {write}
		\item si \code {ls} écrit trop vite (\implique tube plein),
			\code {write} attend
	    \end {itemize}
	\item le processus \code {wc} lit dans le tube
	    \begin {itemize}
		\item utilisation de la primitive \code {read}
		\item si \code {wc} lit trop vite (\implique tube vide),
		    \code {read} attend
	    \end {itemize}
	\item les deux processus tournent en parallèle
	    \begin {itemize}
		\item synchronisation implicite
	    \end {itemize}
	\item pas de limitation sur la quantité de données transférée
    \end {itemize}
\end {frame}

%%%%%%%%%%%%%%%%%%%%%%%%%%%%%%%%%%%%%%%%%%%%%%%%%%%%%%%%%%%%%%%%%%%%%%%%%%%%%%
% Création des tubes
%%%%%%%%%%%%%%%%%%%%%%%%%%%%%%%%%%%%%%%%%%%%%%%%%%%%%%%%%%%%%%%%%%%%%%%%%%%%%%

\titreB {Création des tubes}

\begin {frame} {Création des tubes}
    \prototype {
	\code {int pipe (int tube [2])}
    }

    \begin {itemize}
	\item création du tube et de deux descripteurs
	\item après appel à \code {pipe} :
	    \begin {center}
		\includegraphics [width=.7\linewidth] {\inc/creation-0}
	    \end {center}
	\item lecture via \code {tube [0]}
	\item écriture via \code {tube [1]}
	\item utilité : avec \code {fork}...
    \end {itemize}
\end {frame}


\begin {frame} {Exemple [1/6]}
    \begin {center}
	\includegraphics [width=\linewidth] {\inc/creation-1}
    \end {center}
\end {frame}

\begin {frame} {Exemple [2/6]}
    \begin {center}
	\includegraphics [width=\linewidth] {\inc/creation-2}
    \end {center}
\end {frame}

\begin {frame} {Exemple [3/6]}
    \begin {center}
	\includegraphics [width=\linewidth] {\inc/creation-3}
    \end {center}
\end {frame}

\begin {frame} {Exemple [4/6]}
    \begin {center}
	\includegraphics [width=\linewidth] {\inc/creation-4}
    \end {center}
\end {frame}

\begin {frame} {Exemple [5/6]}
    \begin {center}
	\includegraphics [width=\linewidth] {\inc/creation-5}
    \end {center}
\end {frame}

\begin {frame} {Exemple [6/6]}
    \begin {center}
	\includegraphics [width=\linewidth] {\inc/creation-6}
    \end {center}
\end {frame}

%%%%%%%%%%%%%%%%%%%%%%%%%%%%%%%%%%%%%%%%%%%%%%%%%%%%%%%%%%%%%%%%%%%%%%%%%%%%%%
% Règles de fonctionnement
%%%%%%%%%%%%%%%%%%%%%%%%%%%%%%%%%%%%%%%%%%%%%%%%%%%%%%%%%%%%%%%%%%%%%%%%%%%%%%

\titreB {Règles de fonctionnement}

\begin {frame} {Règles de fonctionnement [1/2]}
    Règles particulières des tubes :
    \begin {itemize}
	\item Primitive \code {read} bloquante
	    \begin {itemize}
		\item \code {read} renvoie 0 quand tube vide et plus
		    aucun écrivain
	    \end {itemize}
	\item \code {read} renvoie ce qui est disponible dans le tube
	    \begin {itemize}
		\item vraisemblablement moins que ce qui est demandé
		\item \implique être prêt à ce que \code
		    {read} renvoie moins que demandé
	    \end {itemize}
	\item Écriture (\code {write}) dans un tube sans lecteur
	    \implique problème !
	    \begin {itemize}
		\item renvoyer \code {-1} ne suffit pas
		    \begin {itemize}
			\item programmes mal écrits ne testent pas
			    les erreurs...
		    \end {itemize}
		\item envoi du signal \code {SIGPIPE} \implique
		    terminaison du processus
		\item exemple : \code {\$ find / | ./a.out}
		    \begin {itemize}
			\item arrêt « prématuré » de \code {a.out} \implique
			    arrêt automatique de \code {find}
			    sans continuer à générer des données inutiles
		    \end {itemize}
	    \end {itemize}
    \end {itemize}
\end {frame}

\begin {frame} {Règles de fonctionnement [2/2]}
    Règles particulières des tubes :
    \begin {itemize}
	\item Il peut y avoir plusieurs lecteurs et plusieurs écrivains
	    \begin {itemize}
		\item situation « normale » (exemple : juste après \code {fork})
		\item attention à la détection de « fin de fichier »
		    \begin {itemize}
			\item plus aucun écrivain...
			\item \implique fermer les descripteurs dès
			    qu'ils ne sont plus utilisés
		    \end {itemize}
	    \end {itemize}

	\item Écritures simultanées par plusieurs écrivains
	    \begin {itemize}
		\item taille $\leq$ \code {PIPE\_BUF} : pas de mélange
		    entre écrivains
		\item taille $>$ \code {PIPE\_BUF} : mélange possible
		    entre écrivains
		\item \code {PIPE\_BUF} = 512 (FreeBSD) ou 4096 (Linux)
	    \end {itemize}

    \end {itemize}
\end {frame}

%%%%%%%%%%%%%%%%%%%%%%%%%%%%%%%%%%%%%%%%%%%%%%%%%%%%%%%%%%%%%%%%%%%%%%%%%%%%%%
% Tubes nommés
%%%%%%%%%%%%%%%%%%%%%%%%%%%%%%%%%%%%%%%%%%%%%%%%%%%%%%%%%%%%%%%%%%%%%%%%%%%%%%

\titreB {Tubes nommés}

\begin {frame} {Tubes nommés}
    \prototype {\code {int mkfifo (const char *path, mode\_t mode)}}

    \begin {itemize}
	\item Tubes créés par \code {pipe} : tubes anonymes
	    \begin {itemize}
		\item doivent être créés par un ancêtre commun aux
		    processus
		\item héritage des descripteurs d'ouverture
	    \end {itemize}
	\item Tubes nommés : nom de fichier dans l'arborescence
	\item Nouveau type de fichier : «~\textit {fifo}~»
	    \begin {itemize}
		\item avec \code {stat} : \code {S\_IFIFO} et
		    \code {S\_ISFIFO()}
	    \end {itemize}
	\item Création avec \code {mkfifo}
	    \begin {itemize}
		\item accès ultérieur avec \code {open}, \code {read},
		    \code {write}, et \code {close}
		\item \implique comme avec n'importe quel fichier régulier
	    \end {itemize}
	\item Règles de fonctionnement : cf tubes anonymes
	    \begin {itemize}
		\item ... après démarrage d'un lecteur et d'un écrivain
		\item \code {read} ou \code {write} bloqué en attendant
		    l'autre partie
	    \end {itemize}
    \end {itemize}
\end {frame}

%%%%%%%%%%%%%%%%%%%%%%%%%%%%%%%%%%%%%%%%%%%%%%%%%%%%%%%%%%%%%%%%%%%%%%%%%%%%%%
% Redirections
%%%%%%%%%%%%%%%%%%%%%%%%%%%%%%%%%%%%%%%%%%%%%%%%%%%%%%%%%%%%%%%%%%%%%%%%%%%%%%

\titreB {Redirections}

\begin {frame} {Redirections}

    \begin {itemize}
	\item Comment faire la redirection avec un tube ?
	    \begin {itemize}
		\item exemple : \code {\$ ls -l | wc -l}
	    \end {itemize}

	\item Principe similaire aux redirections classiques
	    \begin {enumerate}
		\item créer un tube anonyme avant le \code {fork}
		\item dans le processus écrivain (ici \code {ls}) :
		    \begin {enumerate}
			\item \code {dup2 (tube [1], 1)}
			\item \code {close (tube [0])}
			\item \code {close (tube [1])}
		    \end {enumerate}
		\item dans le processus lecteur (ici \code {wc}) :
		    \begin {enumerate}
			\item \code {dup2 (tube [0], 0)}
			\item \code {close (tube [0])}
			\item \code {close (tube [1])}
		    \end {enumerate}
	    \end {enumerate}
    \end {itemize}
\end {frame}
