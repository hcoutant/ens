%%%%%%%%%%%%%%%%%%%%%%%%%%%%%%%%%%%%%%%%%%%%%%%%%%%%%%%%%%%%%%%%%%%%%%%%%%%%%%%
% TUBES
%
% Historique
%   1993/01/31 : pda : création
%   1995/05/02 : pda : ajout des signaux POSIX
%   1996/02/28 : pda : séparation des tubes
%   2013/09/23 : vincent : ajout exercices 4 et 5 (pipes)
%%%%%%%%%%%%%%%%%%%%%%%%%%%%%%%%%%%%%%%%%%%%%%%%%%%%%%%%%%%%%%%%%%%%%%%%%%%%%%%

\td {Système~: Tubes}

% \but
% 
% L'objet de ce TD est l'étude des tubes.


\question
    \label {q:tube1}

Écrivez un programme composé de deux processus~: le père lit des
données sur l'entrée standard et les passe par un {\em tube} au fils
qui les affiche sur sa sortie standard. Le père attend ensuite la
terminaison du fils.


\question

Généralisez l'exercice précédent à $n$ processus~:  le premier passe les
données depuis l'entrée standard au second, qui les passe au troisième,
et ainsi de suite jusqu'au $n-1$-ème qui les passe au $n$-ème, qui les
écrit sur sa sortie standard.


\question

On désire connaître la capacité d'un tube.  Pour cela, on propose
d'envoyer des données, en les comptant, dans un tube qu'aucun lecteur ne
consulte (ouvert en lecture, mais jamais lu par un processus).  Au bout
d'un certain nombre d'octets, l'écrivain se bloque en attendant que le
tube se vide.  Si un signal survient dans cet état, on peut afficher le
nombre d'octets placés dans le tube, c'est à dire sa capacité.

Faites un programe pour afficher la taille d'un tube avec la méthode
ci-dessus.


\question

Écrivez un programme qui crée deux processus fils, redirige les
entrées/sorties et appelle la fonction execlp, pour faire l'équivalent
de ce que fait le shell lorsqu'on tape la commande~:

\begin {quote}
    \texttt {ps eaux | grep "\^{ }<nom>" | wc -l}
\end {quote}

Le nom sera donné par l'utilisateur en argument du programme, ou
prendra la valeur de la variable d'environnement {\tt USER} par défaut.

\question

Reprenez l'exercice~\ref {q:minishell} en y ajoutant les pipes.

\question

Reprenez l'exercice~\ref {q:tube1} en le séparant en deux
programmes distincts et en utilisant un tube nommé.  Le premier
programme crée le tube avec {\tt mkfifo}, et y place les données lues
sur l'entrée standard.  Le deuxième programme ouvre le tube, lit les
données qui s'y trouvent et les affiche sur sa sortie standard.


\question

On désire mettre en place un système de communication inter-processus
sous le contrôle d'un gérant de communication.  Le gérant de
communication reçoit des requêtes des processus par un (unique) tube
nommé (appellé $T$). Le nom de ce tube est connu par tous les processus
voulant communiquer.

Lorsqu'un processus $P_i$ veut communiquer avec un autre, il crée un
tube nommé (appelé $T_i$) qui lui est propre, puis envoie au gérant (par
l'intermédiaire du tube $T$) un message spécifiant le nom de $T_i$.
C'est l'{\em abonnement}.

Le gérant ouvre alors $T_i$, lui associe un numéro interne ($\geq 0$),
et renvoie ce numéro interne au processus $P_i$ par l'intermédiaire de
$T_i$.

Par la suite, lorsque le processus $P_i$ veut envoyer un message au
processus $P_j$, le processus $P_i$ écrit un message dans le tube $T$,
le gérant lit ce message, l'analyse pour voir si c'est possible (i.e.
si $P_j$ est déjà abonné) et l'envoie à $P_j$ par l'intermédiaire du
tube $T_j$. Le gérant envoie un message $P_i$ pour lui signaler si
l'opération a réussi ou non.

Le processus $P_i$ doit pouvoir obtenir la liste de tous les processus
abonnés. Dans ce cas, il envoie un message dans $T$, le gérant répond
alors par $T_i$ en lui envoyant la liste.

Le processus $P_i$ doit aussi pouvoir se désabonner.

\begin {enumerate}
    \item Spécifiez le format des messages (c'est-à-dire le {\em
	protocole}) circulant dans $T$ et dans les tubes $T_i$.

    \item Programmez le gérant.

\end {enumerate}


\question

On s'intéresse maintenant aux processus clients (les $P_i$). Ils
doivent avoir une interface simplifiée (interface de programmation)
pour dialoguer avec le gérant.

Cette interface suppose que toutes les données relatives à une
connexion sont décrites par un type {\tt comm} dont vous devez
définir le contenu.

Les fonctions disponibles sont~:

\begin {itemize}

    \item \verb:comm initialiser (char tube []):

	Cette fonction amorce une communication avec le gérant spécifié
	par le tube de nom {\tt tube} (elle crée et ouvre le tube propre
	au processus, et elle le tube du gérant).

    \item \verb:int liste (comm gerant, int abonnes [], int maxab):

	Cette fonction retourne dans le tableau {\tt abonne} la liste
	des abonnes au gérant de communication spécifié par {\tt
	gerant}.  Le nombre maximum\footnote {S'il y a plus d'abonnés,
	leurs numéros sont perdus.} de numéros pouvant être placés dans
	ce tableau est spécifié par {\tt maxab}. Cette fonction retourne
	le nombre d'abonnés placés dans le tableau, ou -1 pour signifier
	une erreur.
    
    \item \verb:int envoyer (comm gerant, int abonne, char *message, int lg):

	Cette fonction envoie à l'abonné {\tt abonne}, par
	l'intermédiaire du gérant, un message spécifié par les {\tt lg}
	octets inclus dans {\tt message}.  Le résultat est un code
	signifiant si l'opération s'est bien déroulée.

    \item \verb:int recevoir (struct comm *comm, char *message, int *lgmax):

	Cette fonction attend un message. Le message (au plus {\tt
	lgmax} octets) est placé dans la zone spécifiée par {\tt
	message}. La longueur du message reçu est placée en retour dans
	{\tt lgmax}. Le résultat de cette fonction est l'identificateur
	de l'émetteur, ou -1 si une erreur est intervenue.

\end {itemize}

On vous demande de :

\begin {enumerate}

    \item définir le type {\tt comm}~;

    \item programmer une application simple à l'aide des fonctions
	décrites ci-dessus~: à chaque fois qu'un processus s'abonne, il
	demande à tous les autres abonnés leur numéro de processus, et
	affiche le numéro minimum~;
    
    \item programmer les fonctions ci-dessus.

\end {enumerate}



