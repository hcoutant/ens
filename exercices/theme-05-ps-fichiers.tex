%%%%%%%%%%%%%%%%%%%%%%%%%%%%%%%%%%%%%%%%%%%%%%%%%%%%%%%%%%%%%%%%%%%%%%%%%%%%%%%
% PRIMITIVES DE GESTION DE FICHIERS
%
% Historique
%   1993/01/31 : pda : création
%   1996/09/10 : pda : réunion en un seul thème
%   1997/08/26 : pda : ajout de l'exercice des erreurs
%   1997/08/26 : pda : ajout de l'exercice "ls"
%%%%%%%%%%%%%%%%%%%%%%%%%%%%%%%%%%%%%%%%%%%%%%%%%%%%%%%%%%%%%%%%%%%%%%%%%%%%%%%

\td {Système~: Gestion de fichiers}

% \but
% 
% L'objet de ce TD est l'étude des primitives système, et plus
% particulièrement des primitives de gestion des fichiers.


\question

Nommez quelques primitives système. Est-ce que {\tt fopen} est une
primitive système~?

Quelles sont les différences et ressemblances entre
primitives système et fonctions de bibliothèque~?
Comment justifier ces différences~?

Illustrez ces différences et ressemblances sur les
fonctions et primitives d'entrée~/~sortie.


\question

Pourquoi la fonction ci-dessous ne peut pas fonctionner ?  Détaillez
toutes les fautes.  On ne demande pas de corriger cette fonction.

\begin {quote}
\small
\begin {verbatim}
int faux (char *nom)
{
    FILE *fp ;
    int c ;

    fp = open (nom, "r") ;
    read (fp, &c, 1) ;
    fclose (fp) ;
    return c ;
}
\end{verbatim}
\end {quote}


\question
    \label {cp}

Écrivez un programme qui recopie un fichier {\tt toto} vers un fichier
{\tt titi} à créer, à l'aide des primitives système.
Vous ne chercherez
pas à créer le nouveau fichier avec les permissions du fichier original.


\question

Écrivez la fonction {\tt getchar} qui renvoie un caractère lu
sur l'entrée standard, ou la constante {\tt EOF} en fin de
fichier.

Quelle peut être la valeur numérique de la constante EOF~?


\question

Écrivez une version bufferisée de {\tt getchar}.


\question

On croit souvent que les primitives système étant de plus bas niveau,
elles sont plus efficaces que les fonctions de bibliothèque
équivalentes. On désire confirmer ou infirmer cette proposition par
l'expérimentation.

Pour cela, on demande de rédiger deux programmes pour copier l'entrée
standard sur la sortie standard. Le premier utilisera les fonctions
de bibliothèque \texttt {getchar} et \texttt {putchar}.  Le deuxième
utilisera les primitives système \texttt {read} et \texttt {write}
et prendra en argument la taille du buffer utilisé pour la copie.
Si cette taille égale 1, la copie sera effectuée caractère par
caractère.

Vous utiliserez la commande Unix \texttt {time} pour comparer les
temps d'exécution, en considérant la somme des temps CPU en mode
utilisateur et en mode système\footnote {Le temps «~réel~» correspond
au temps réellement écoulé~: on nt'utilisera pas cette valeur car
elle dépend de la charge du système.}.

En prenant comme tailles de buffer les puissances successives de 2
(2$^0$, 2$^1$, 2$^2$, 2$^3$, etc.), à partir de quelle taille de
buffer est-il plus intéressant d'utiliser les primitives système
que les fonctions de bibliothèque~?


\question

Écrivez un programme qui affiche en clair le type du fichier demandé
(répertoire, fichier ordinaire, etc.), ainsi que ses permissions
(lecture, écriture et exécution, sous la même forme que la commande {\tt
ls} avec l'option {\tt -l}).


\question

On désire implémenter une nouvelle version de la librairie standard
d'entrées/sorties à l'aide des primitives système.

\begin {enumerate}
    \item Donnez une définition du type {\tt FICHIER}.  N'oubliez de
	prévoir la bufferisation des entrées/sorties.

    \item Programmez la fonction {\tt my\_open}, analogue à {\tt fopen}.

    \item Reprenez l'exercice précédent pour programmer {\tt my\_getc},
	analogue à {\tt getc}.

    \item Programmez la fonction {\tt my\_putc}, analogue à {\tt putc}.

    \item Programmez la fonction {\tt my\_close}, analogue à {\tt
	fclose}.

\end {enumerate}



\question

Écrivez une commande qui prend en paramètre un nom de répertoire, et
qui affiche tous les objets contenus dans ce répertoire. On prendra les
mêmes conventions de restriction d'affichage que la commande {\tt ls}
(pas d'affichage des noms commençant par un point).


\question

Reprenez le programme de l'exercice~\ref {cp} pour recopier toute une
arborescence.


\question

Reprenez le programme de l'exercice précédent pour restaurer dans les
copies les dates d'accès et modification ainsi que les permissions des
fichiers originaux.


\question

Reprenez la fonction développée lors de l'exercice~\ref {getpath} pour écrire
un programme {\tt which} afin de chercher où une commande est trouvée.
Par exemple, ``{\tt which~ls}'' doit donner~: {\tt /bin/ls}.


\question

Le shell {\tt ksh} dispose d'une variable {\tt CDPATH}.  Celle-ci
spécifie une certain nombre de répertoires de recherche.  Lorsqu'on
utilise {\tt cd} avec un argument (nom de chemin relatif), celui est
cherché dans les différents répertoires indiqués par {\tt CDPATH} et le
changement de répertoire est effectué s'il est trouvé.

Programmez une commande {\tt chdir}.

Pourquoi cette commande ne peut pas fonctionner~?


