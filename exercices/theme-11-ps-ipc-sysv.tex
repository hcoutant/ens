%%%%%%%%%%%%%%%%%%%%%%%%%%%%%%%%%%%%%%%%%%%%%%%%%%%%%%%%%%%%%%%%%%%%%%%%%%%%%%%
% IPC SYSTEM V
%
% Historique
%   1993/01/31 : pda : création
%   1996/04/09 : pda : séparation
%   1996/09/10 : pda : réunion en un seul thème
%   2007/05/11 : pda : ajout exercice
%%%%%%%%%%%%%%%%%%%%%%%%%%%%%%%%%%%%%%%%%%%%%%%%%%%%%%%%%%%%%%%%%%%%%%%%%%%%%%%

\td {Système~: IPC System V}

% \but
% 
% L'objet de ce TD est l'étude des primitives de communication dites IPC
% System~V.


\question

Écrivez un programme composé de deux processus. Le père crée
une file de messages anonyme, puis y envoie des messages composés
de lignes lues sur l'entrée standard. Lorsque la fin de fichier
est détectée, il envoie une chaîne vide, puis attend la terminaison
du processus fils. Le processus fils reçoit les messages, puis
comptabilise le nombre d'octets reçus.


\question

On désire traiter des requêtes fournies sous forme de messages, chaque
requête étant composée d'une priorité (de 1 à 3, 1 étant la plus
prioritaire) et d'une donnée de type \verb|char []|.

La file de messages est identifiée par une clef formée à l'aide d'un nom
de fichier ({\tt toto} par exemple) et de la lettre E.

Écrivez le programme serveur qui crée la file de messages, puis traite
les messages selon leur priorité.  Chaque message est traité à l'aide
d'une fonction {\tt traite} que l'on supposera écrite.  Le programme
doit se terminer et effacer la file de messages lorsqu'il reçoit un
message contenant une donnée de taille nulle.

Écrivez à présent le programme client qui reçoit zéro (pour envoyer un
message de taille nulle) ou deux paramètres (priorité et chaîne de
caractères représentant la donnée).


\question

On suppose qu'il existe une file de messages de clef $c$. On désire
avoir un système qui permette à $n$ processus d'émettre des messages à
destination du processus $i$ ($1 \leq i \leq n$).

Programmez les fonctions~:

\begin {quote}
\begin {verbatim}
    int ouvrir (key_t clef)
    int envoyer (int msqid, int moi, int lui,
                        void *message, size_t taille)
    int recevoir (int msqid, int moi, int *lui,
                        void *message, size_t *taille)
\end{verbatim}
\end {quote}

La fonction {\tt ouvrir} doit permettre l'accès à la file. La fonction
{\tt envoyer} envoie, par l'intermédiaire de la file, un message à
destination du processus {\tt lui}, composé de {\tt taille} octets
situés à l'adresse {\tt message}. La fonction {\tt recevoir} attend un
message de l'un quelconque des processus, place à l'adresse {\tt lui} le
numéro de l'émetteur, et place à l'adresse {\tt message} le message
reçu. Le paramètre {\tt taille} précise la place maximum utilisable pour
le message en entreé, et la place réellement utilisée en sortie.


\question

On désire communiquer des grandes quantités de données entre deux
processus.  Pour cela, on propose d'utiliser un segment de mémoire
partagée de taille {\tt MAX} octets.  Lorsque l'émetteur désire envoyer
$n$ octets, il les place au début du segment, puis il envoie un
message contenant $n$ au récepteur.  Lorsque le récepteur a fini de
lire le message, il renvoie un message a l'émetteur pour signifier que
la place est libre. Programmez les deux processus.


\question

% DEPENDANCE VIS-A-VIS DE LA QUESTION PRECEDENTE
Reprenez l'exercice précédent en utilisant un segment de mémoire
partagée et un groupe de sémaphores~: expliquez la méthode de
synchronisation et programmez les deux processus.


\question

Écrivez un programme composé de deux processus~:  le premier
initialise un segment de mémoire partagée et un groupe de deux
sémaphores, crée le fils, puis lit des données entières dans le segment
et les affiche.  Le fils calcule des données avec une fonction {\tt
calcul} (qui renvoie 0 lorsque toutes les données ont été calculées) et
les transmet au fur et à mesure au père par l'intermédiaire du segment
de mémoire partagée.  Pour écrire une donnée, le fils fait un P sur le
premier sémaphore, écrit la donnée, puis fait un V sur le deuxième
sémaphore.  De manière symétrique, le père fait un P sur le deuxième
sémaphore, lit la donnée, puis fait un V sur le premier sémaphore.


\question

% DEPENDANCE VIS-A-VIS DE LA QUESTION PRECEDENTE
En vous basant sur l'exercice précédent, écrivez deux programmes.
Le premier s'utilise de la manière suivante~:

\begin {quote}
    \small
    \verb|producteur| \textit {n}
\end {quote}

où \textit {n} est une valeur entière. La valeur est placée dans
le segment de mémoire partagée après la dernière valeur déjà produite.
S'il n'y a plus d'emplacement libre, le programme attend qu'un
emplacement se libère par le deuxième programme.

Le deuxième programme, \texttt {consommateur}, ne prend aucun
argument et renvoie la première valeur stockée dans le segment de
mémoire partagée. Si aucune valeur n'est disponible, le programme
attend qu'un producteur y stocke une valeur.

Vous noterez que l'indice de la première valeur stockée doit être
partagé par tous les consommateurs. De même, l'indice du premier
emplacement libre doit être partagé par tous les producteurs.  Ces
deux indices doivent donc être placés dans le segment de mémoire
partagée, et constituent des variables critiques qui doivent être
protégées par des sémaphores d'exclusion mutuelle.


\question

On souhaite implémenter un mécanisme comparable aux files de messages à
l'aide de la mémoire partagée et un groupe de trois sémaphores.  Le
segment de mémoire partagée contient (voir figure) un en-tête, puis les
messages eux-mêmes.
La zone des messages est gérée comme un buffer circulaire~:  lorsqu'un
message ne peut être mis en entier à la fin du segment, sa suite est
placée au début de la zone des messages.
L'en-tête contient la taille de la zone des messages, les références au
premier message et au dernier message.

% \fig {msg} {Segment de mémoire partagée}
\fig {msg}

Par convention, le groupe de sémaphores a une clef égale à la
clef du segment de mémoire partagée + 1. Il contient~:

\begin {itemize}
    \item un sémaphore représentant la place disponible dans le
	segment~:  il est initialisé à la taille de la zone des messages
	en nombre d'octets~;
    \item un deuxième sémaphore pour réaliser l'exclusion mutuelle lors
	de la manipulation des structures de données~: il est initialisé
	à 1~;
    \item un troisième sémaphore pour représenter le nombre de messages
	dans la file~: il est initialisé à 0~;
\end {itemize}


Écrivez les fonctions~:

\begin {itemize}
    \item \verb|int initialiser_file (key_t clef)| \\
	cette fonction, appelée une seule fois, crée et initialise un
	segment de mémoire partagée pour 10000 octets de messages, puis
	crée le groupe de sémaphores et l'initialise et retourne -1 en
	cas d'erreur~;

    \item \verb|int envoyer (key_t clef, const char *message)| \\
	cette fonction envoie le message spécifié, et retourne -1 en cas
	d'erreur~;

    \item \verb|int recevoir (key_t clef, char *message)| \\
	cette fonction attend un message, le recopie dans la zone
	spécifiée par l'adresse fournie en paramètre (que l'on suppose
	déjà allouée), et retourne -1 en cas d'erreur.

\end {itemize}

