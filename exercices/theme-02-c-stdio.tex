%%%%%%%%%%%%%%%%%%%%%%%%%%%%%%%%%%%%%%%%%%%%%%%%%%%%%%%%%%%%%%%%%%%%%%%%%%%%%%%
% LANGAGE C : FONCTIONS D'ENTREES/SORTIES
%
% Historique
%   1995/02/16 : pda : création
%   1997/08/26 : pda : ajout de l'exercice archiver/desarchiver
%%%%%%%%%%%%%%%%%%%%%%%%%%%%%%%%%%%%%%%%%%%%%%%%%%%%%%%%%%%%%%%%%%%%%%%%%%%%%%%


\td {Langage~C~: Fonctions d'entrées/sorties}


% \but
% 
% Cette séance est destinée à introduire les fonctions d'entrées/sorties
% de la bibliothèque standard.


\question

Écrivez un programme qui recopie un fichier {\tt toto} vers un fichier
{\tt titi} à créer, à l'aide des fonctions {\tt fopen}, {\tt putc}, {\tt
getc} et {\tt fclose}.


\question

On désire construire un programme pour réaliser un carnet d'adresses
rudimentaire.  Le carnet est placé dans un fichier, composé de
fiches représentées par des structures du langage~C~:

\begin {quote}
\small
\begin {verbatim}
#define MAXNOM 50
#define MAXTEL 10

struct fiche
{
    int  occupe ;               /* vrai la fiche est utilisee */
    char nom [MAXNOM] ;         /* le nom de la personne */
    char telephone [MAXTEL] ;   /* son numero de telephone */
} ;

typedef struct fiche fiche ;
\end{verbatim}
\end {quote}

Lorsqu'une fiche est détruite, le champ {\tt occupe} est initialisé à la
valeur 0 ({\em faux}). Cette fiche pourra être réutilisée par la suite.

L'accès au carnet est réalisé par une suite de petits programmes
accomplissant chacun une action bien précise~:

\begin {itemize}
    \item {\tt chercher} \\
	Ce programme prend un argument, le nom à chercher. Si le nom est
	trouvé, le numéro de la fiche est affiché sur la sortie
	standard. Dans le cas contraire, rien n'est affiché.
    \item {\tt afficher} \\
	Ce programme prend un argument, le numéro de la fiche à
	afficher. Si ce numéro n'existe pas (argument vide), aucune
	fiche n'est affichée.
    \item {\tt ajouter} \\
	Ce programme prend deux arguments, un nom et un numéro de
	téléphone. La fiche est ajoutée dans le fichier, soit en
	récupérant une fiche inoccupée, soit en l'ajoutant à la fin du
	fichier.
    \item {\tt detruire} \\
	Ce programme prend un argument, le numéro de la fiche à
	détruire.
\end {itemize}


\question

La commande Unix {\tt tail} affiche, par défaut, les 10 dernières lignes
d'un fichier texte. L'option {\tt -n} permet de changer ce nombre de
lignes. Programmez votre propre version de cette commande {\tt tail}

Pour analyser les arguments, il faut utiliser la fonction {\tt getopt}.
Pour repérer les $n$ dernières lignes d'un fichier, une méthode suggérée
consiste à lire les $m$ derniers octets du fichier et à compter le
nombre de caractères \verb|\n| en partant de la fin.  Si ce nombre n'est
pas suffisant, on recommence.  Lorsque le nombre $n$ de \verb|\n| est
atteint, on affiche tous les caractères jusqu'à la fin du fichier.


\question

Le programme {\tt tar} est un utilitaire d'archivage :  il recopie toute
une arborescence et stocke tous les fichiers bout à bout dans un seul
fichier (l'{\em archive}).  Ceci peut être utile pour réaliser des
sauvegardes (la cartouche ou la bande est alors vue par le système comme
le fichier archive), pour transmettre des fichiers multiples par le
réseau, par courrier électronique ou bien avec {\tt ftp}.

On désire programmer deux utilitaires d'archivage comparables à {\tt tar}.

\begin {enumerate}

    \item la syntaxe de l'archiveur est :

	\begin {quote}
	    \verb|archiver archive f1 f2 ... fn|
	\end {quote}

	Le fichier {\tt archive} est le fichier dans lequel les fichiers
	{\tt f1} à {\tt fn} sont recopiés et mis bout à bout.  Chaque
	fichier y est précédé de son nom et de sa taille en octets.

	Note : quelles précautions doit-on prendre pour qu'un fichier
	archivé sur une machine puisse être désarchivé sur une autre~?

    \item la syntaxe du désarchiveur est :

	\begin {quote}
	    \verb|desarchiver archive|
	\end {quote}

	Cette commande analyse le fichier {\tt archive} et en extrait
	tous les fichiers.

\end {enumerate}

