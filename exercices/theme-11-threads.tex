%%%%%%%%%%%%%%%%%%%%%%%%%%%%%%%%%%%%%%%%%%%%%%%%%%%%%%%%%%%%%%%%%%%%%%%%%%%%%%%
% THREADS POSIX
%
% Historique
%   2013/08/28 : pda : création
%%%%%%%%%%%%%%%%%%%%%%%%%%%%%%%%%%%%%%%%%%%%%%%%%%%%%%%%%%%%%%%%%%%%%%%%%%%%%%%

\td {Threads POSIX}

% \but
% 
% L'objet de ce TD est l'étude des threads POSIX

\question	% erreur classique

Que fait le programme ci-après ?

\begin {quote}
\begin {verbatim}
#define MAX     10

void *fonction (void *arg)
{
    printf ("numéro du thread = %d\n", * (int *) arg) ;
    return NULL ;
}

void erreur (char *msg)
{
    perror (msg) ; exit (1) ;
}

int main (int argc, char *argv [])
{
    pthread_t tid [MAX] ; int i ;
    for (i = 0 ; i < MAX ; i++)
        if (pthread_create (&tid [i], NULL, &fonction, (void *) &i) == -1)
            erreur ("pthread_create") ;
    for (i = 0 ; i < MAX ; i++)
        if (pthread_join (tid [i], NULL) == -1)
            erreur ("pthread_join") ;
}
\end{verbatim}
\end {quote}


\question	% threads indépendants

Soit un programme qui prend en argument 2 valeurs $n$ et $p$ et démarre
$n$ threads.  Chaque thread reçoit les valeurs $n$ et $p$ ainsi que
le numéro $t \in [1,n]$ du thread et calcule $u_t = \sum_{i=1}^{p}
((t-1)p + i)$. Le thread principal affiche la somme des valeurs $u_t$
calculées.

Quelle est la valeur calculée par ce programme~? Rédigez-le.

% . Le premier calcule la somme des entiers de $1$ à $p$, le deuxième
% calcule la somme des entiers de $p+1$ à $2 p$, et ainsi de suite jusqu'au
% thread numéro $n$ (qui calcule la somme des entiers de $(n-1) p +1$
% jusqu'à $n p$.


\question	% threads indépendants

Écrivez un programme composé de $n$ nouveaux threads. Chacun de ces
threads incrémente un compteur qui lui est propre et s'arrête à
la valeur 10 en affichant un message. À chaque incrément, le thread
doit attendre un temps aléatoire (vous utiliserez \texttt {nanosleep}
et vous limiterez la durée à une seconde) et afficherez la valeur
du compteur préfixée par le numéro du thread ($\in [1,n]$). Le
programme ne doit se terminer que lorsque les $n$ threads ont terminé,
et doit afficher le numéro du thread qui s'est terminé en dernier.


\question	% mise en évidence du problème de variable critique
		% note : en TP uniquement...

Écrivez un programme composé de 4 nouveaux threads. Chaque thread
incrémente un million de fois une variable globale \texttt {compteur}
de type \texttt {long int}. Une fois les threads terminés, affichez
la valeur du compteur.

Que constatez-vous ? Que proposez-vous pour résoudre le problème ?


\question

Expliquez, pour chacune des deux portions de code suivantes, pourquoi
elle ne peut pas être exécutée de manière fiable par deux threads.

\hfill
\begin {minipage} [t] {0.45\linewidth}
    \small
    \begin {verbatim}
struct listelem *head ;

void insert_list (struct listelem *e)
{
    e->next = head ;
    head = e ;
}
\end{verbatim}
\end {minipage}
\hfill
\begin {minipage} [t] {0.45\linewidth}
    \small
    \begin {verbatim}
double compte_en_banque ;

void recevoir (double montant)
{
    compte_en_banque += montant ;
}
\end{verbatim}
\end {minipage}


\question

On désire réaliser des sémaphores à utiliser avec les threads POSIX.
Pour cela, on définit les fonctions suivantes~:

\begin {quote}
\begin {verbatim}
semaphore_t semaphore_create (int val) ;
void semaphore_P (semaphore_t sem) ;
void semaphore_V (semaphore_t sem) ;
\end{verbatim}
\end {quote}

Donnez la définition du type \verb|semaphore_t|, et rédigez ces fonctions
en utilisant les mutex.
