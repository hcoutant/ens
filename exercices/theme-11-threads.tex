%%%%%%%%%%%%%%%%%%%%%%%%%%%%%%%%%%%%%%%%%%%%%%%%%%%%%%%%%%%%%%%%%%%%%%%%%%%%%%%
% THREADS POSIX
%
% Historique
%   2013/08/28 : pda : création
%%%%%%%%%%%%%%%%%%%%%%%%%%%%%%%%%%%%%%%%%%%%%%%%%%%%%%%%%%%%%%%%%%%%%%%%%%%%%%%

\td {Threads POSIX}

% \but
% 
% L'objet de ce TD est l'étude des threads POSIX

\question	% threads indépendants

Écrivez un programme composé de $p$ threads pour calculer la somme
des entiers de 1 à $n$ (avec $n > p$. Le premier thread calcule la
somme des entiers de $1$ à $p-1$, le deuxième calcule la somme des
entiers de $p$ à $2p-1$, et ainsi de suite jusqu'au dernier 

\question	% threads indépendants

Écrivez un programme composé de $n$ nouveaux threads. Chacun de ces
threads incrémente un compteur qui lui est propre et s'arrête à
la valeur 10 en affichant un message. À chaque incrément, le thread
doit attendre un temps aléatoire (vous utiliserez \texttt {nanosleep}
et vous limiterez la durée à une seconde) et afficherez la valeur
du compteur préfixée par le numéro du thread ($\in [1...n]$). Le
programme ne doit se terminer que lorsque les $n$ threads ont terminé,
et doit afficher le numéro du thread qui s'est terminé en dernier.

\question	% tri rapide en parallèle

tri rapide 


\question

Expliquez, pour chacune des deux portions de code suivantes, pourquoi
elle ne peut pas être exécutée de manière fiable par deux threads.

\hfill
\begin {minipage} [t] {0.45\linewidth}
    \small
    \begin {verbatim}
struct listelem *head ;

void insert_list (struct listelem *e)
{
    e->next = head ;
    head = e ;
}
\end{verbatim}
\end {minipage}
\hfill
\begin {minipage} [t] {0.45\linewidth}
    \small
    \begin {verbatim}
double compte_en_banque ;

void recevoir (double montant)
{
    compte_en_banque += montant ;
}
\end{verbatim}
\end {minipage}


\question

On désire réaliser des sémaphores à utiliser avec les threads POSIX.
Pour cela, on définit les fonctions suivantes~:

\begin {quote}
\begin {verbatim}
semaphore_t semaphore_create (int val) ;
void semaphore_P (semaphore_t sem) ;
void semaphore_V (semaphore_t sem) ;
\end{verbatim}
\end {quote}

Donnez la définition du type \verb|semaphore_t|, et rédigez ces fonctions
en utilisant les mutex.
