%%%%%%%%%%%%%%%%%%%%%%%%%%%%%%%%%%%%%%%%%%%%%%%%%%%%%%%%%%%%%%%%%%%%%%%%%%%%%%%
% LANGAGE C : POINTEURS ET CALCULS D'ADRESSE
%
% Historique
%   1995/10/05 : pda : création
%%%%%%%%%%%%%%%%%%%%%%%%%%%%%%%%%%%%%%%%%%%%%%%%%%%%%%%%%%%%%%%%%%%%%%%%%%%%%%%


\td {Langage~C~: Pointeurs}


% \but
% 
% Cette séance est une introduction au maniement des pointeurs,
% et notamment l'arithmétique sur les pointeurs.


\question

À l'aide d'un papier et d'un crayon, reconstituez le déroulement
du programme suivant. Déduisez-en l'affichage sur la sortie stantard,
ainsi que les endroits auxquels le programme provoque une erreur.

\begin {quote}
\small
\begin {verbatim}
#define PRabp(n)    printf ("Ligne %d: " \
        "a: 0x%x %d, b: 0x%x %d, p: 0x%x 0x%x %d\n", \
        n, &a, a, &b,b,&p, p, *p)
#define PRtp(n)     printf( "Ligne %d: " \
        "t: 0x%x %d %d %d, p: 0x%x 0x%x %d\n", \
        n, t, t[0], t[1], t[2],&p, p, *p)

int main (int argc, char *argv [])
{
    int a, b ;
    int t [3] = { 4, 5, 6 } ;
    int *p ;

    p=&b ; a=0 ; b=2 ;       PRabp (1) ;
    *p=4 ;                   PRabp (2) ;
    p++ ;                    PRabp (3) ;
    (*p)++ ;                 PRabp (4) ;
    p=0 ;                    PRabp (5) ;
    p=t ;                    PRtp (6) ;
    p[0] = 10 ; p[1] = 11 ;  PRtp (7) ;
    p++ ;                    PRtp (8) ;
    *p = 15 ; *(p+1) = 16 ;  PRtp (9) ;
    p++ ;                    PRtp (10) ;
    p[0] = 20 ; p[1] = 21 ;  PRtp (11) ;
    return 0 ;
}
\end{verbatim}
\end {quote}


\question
%%%% Dependances: fgets, atoi, pointeurs, malloc

Écrivez un programme qui lit une liste de nombres entiers (avec les
fonctions {\tt fgets} et {\tt atoi}) jusqu'à la fin de fichier, les
enregistre dans une liste en les triant au fur et à mesure, puis affiche
les nombres de la liste (avec la fonction {\tt printf}).


\question

Modifiez le programme de l'exercice précédent pour mettre dans la
liste (qui doit toujours être triée) des chaînes de caractères lues au
clavier.


\question
%%%% Dependances: structures, pointeurs, arithmetique sur les pointeurs

Lorsqu'une structure est déclarée, chaque champ est placé à une adresse
relative au début de la structure.  Soit une structure {\tt s} et une
champ {\tt c}, expliquez comment obtenir la valeur numérique de cette
adresse relative.


\question
%%%% Dependances: fgets, puts, chaînes, strlen, strncmp, pointeurs
%%%% Dependances: arithmetique sur les pointeurs

Écrivez un programme qui lit deux chaînes, puis cherche si la deuxième
chaîne fait partie de la première, et affiche un message en conséquence
(voir exercice~\ref {strbrk}).  Cette fois-ci, vous utiliserez les
fonctions de librairie {\tt strlen} et {\tt strncmp}.


\question
%%%% Dependances: fonction, pointeurs, passage par adresse

Écrivez la fonction \verb:char *mon_strchr (char *chaine, int car):
sans utiliser de fonction de librairie (et surtout pas {\tt strchr}).


\question
%%%% Dependances: fonction, pointeurs, passage par adresse, sizeof,
%%%% Dependances: pile d'execution

Pourquoi les fonctions ci-après ne peuvent pas fonctionner~?
%Pour chacune des fonctions suivantes, il vous est demandé~:
%
%\begin {enumerate}
%    \addtolength {\itemsep} {-2mm}
%    \item de déterminer si la fonction est valide ou non~;
%    \item si elle est valide, d'expliquer son fonctionnement~;
%    \item si elle n'est pas valide, de préciser la ou les raisons~;
%\end {enumerate}

\begin {quote}
    \small

% l'adresse retournee est dans la pile
\begin {verbatim}
int *f1 (int val)
{
    int *p ;
    p = val == 0 ? NULL : &val ;
    return p ;
}
\end{verbatim}

% l'adresse retournee est dans la pile
\begin {verbatim}
char *f2 (void)
{
    char tab [3 + 1] ;
    strcpy (tab, "abc") ;
    return tab ;
}
\end{verbatim}

% l'argument de printf necessite un "*p"
\begin {verbatim}
static char *texte [] = { "une", "suite", "de", "lignes", NULL, } ;

void f3 (void)
{
    char **p ;
    for (p = texte ; *p ; p++)
        printf ("%s\n", p) ;
}
\end{verbatim}

% tab ne peut être incrémenté : c'est un tableau
\begin {verbatim}
char tab [] = "abcdef\n" ;

void f4 (void)
{
    int i ;
    for (i = 0 ; *(tab + i) ; i++)
        putchar (*(tab + i)) ;
    for ( ; *tab ; tab++)
        putchar (*tab) ;
}
\end{verbatim}


% deux problèmes :
%   - sizeof (chaine) = la taille du pointeur
%   - la chaine est stockée en mémoire, et chaine est un pointeur :
%	au deuxième appel, la chaîne sera la chaîne modifiée lors
%	du premier appel.
\begin {verbatim}
void f5 (char tab [], int pas)
{
    char *chaine = "abcdef" ;
    int i ;
    for (i = 0 ; i < sizeof (chaine) ; i += pas)
        chaine [i] = '-' ;
    strcpy (tab, chaine) ;
}
\end{verbatim}
\end {quote}


\question
%%%% Dependances: printf, scanf, float, tableaux, passage de parametre par adr
    \label {sortfloat}

On désire écrire un programme pour lire une série de nombres flottants et
les placer dans un tableau (on supposera qu'il n'y a pas plus de 100
valeurs), trier ce tableau et enfin l'afficher sur la sortie standard.
Pour cela, on propose la décomposition en fonctions~:

\begin {enumerate}
    \item \verb|int lire_tableau (double *tableau, int max_elem, int *nb_elem)| \\
	qui reçoit un tableau en paramètre contenant au plus {\tt
	max\_elem} éléments, lit les valeurs et les range dans le
	tableau, puis renvoie le nombre d'élements lus dans {\tt
	nb\_elem}.  La valeur de retour de cette fonction doit être 0 si
	la lecture s'est bien passée, -1 sinon.


    \item \verb|void trier_tableau (double *tableau, int nb_elem)| \\
	qui trie les {\tt nb\_elem} éléments du tableau {\tt tableau}.

    \item \verb|void afficher_tableau (double *tableau, int nb_elem)| \\
	qui affiche les {\tt nb\_elem} éléments du tableau {\tt tableau}.

\end {enumerate}

Programmez ces fonctions ainsi que {\tt main}.


\question
%%%% Dependances: tableaux d'entiers, malloc

Écrivez un programme qui lit une série d'entiers sur l'entrée
standard, les range dans un tableau (on supposera qu'on ne lit pas plus
de 100 nombres), puis trie ce tableau et enfin l'affiche sur la sortie
standard. Vous n'avez pas le droit d'utiliser l'opérateur {\tt[}...{\tt]}.



\question
%%%% Dependances: pointeurs, typedef, fonctions

Écrivez la déclaration d'un tableau de 50 pointeurs sur des structures
contenant chacune un pointeur sur un caractère, un tableau de 10
entiers et un pointeur sur cette même structure. Donnez une définition
de ce type avec {\tt typedef}.


\question
%%%% Dependances: fonctions, char *, malloc

La fonction {\tt fgets} a un défaut~: l'espace dans lequel les caractères
lus sont rangés doit être alloué par la fonction appelante. Écrivez
une nouvelle fonction~:

\begin {quote}
    \verb|char *mon_fgets (void)|
\end {quote}

qui lit des caractères sur l'entrée standard
(on supposera qu'on ne lit pas plus de 100
caractères) et renvoie l'adresse d'une zone mémoire où votre
fonction les aura rangés.


\question
%%%% Dependances: fonctions, passage par adresse, pointeurs

On désire remplacer la fonction de lecture de tableau de
l'exercice~\ref {sortfloat} en éliminant la contrainte d'un nombre
d'entrées maximum. Pour cela, on change le prototype en~:

\begin {quote}
    \verb|int lire_tableau (double **tableau, int *nb_elem)|
\end {quote}

Le paramètre {\tt tableau} reçoit en sortie un tableau avec le nombre
suffisant d'entrées pour mémoriser tous les éléments lus.  Vous
commencerez par allouer (avec {\tt calloc}) un tableau de $n = 100$
entrées.  Lorsque $n$ éléments sont saisis, allouez un nouveau tableau
de taille $2n$ (sans utiliser {\tt realloc}), recopiez-y les $n$
éléments et désallouez le premier tableau.


\question
%%%% Dependances: fonctions

Comment peut-on déduire que le programmeur qui a écrit la fonction
ci-dessous est débutant~?

\begin {quote}
\small
\begin {verbatim}
int main (void)
{
    char *ligne ;

    ligne = malloc (80) ;
    while (gets (ligne) != NULL)
        puts (ligne) ;
}
\end{verbatim}
\end {quote}



\question
    \label {arbre}

On désire manipuler des expressions (composées des quatre opérations
classiques manipulant exclusivement des entiers) représentées sous forme
d'arbres.  Pour cela, on définit une structure~:

\begin {quote}
\small
\begin {verbatim}
enum code { noeud, feuille } ;

struct noeud
{
    enum code code ;
    union
    {
        int valeur ;
        struct
        {
            char operation ;
            struct noeud *fils_gauche ;
            struct noeud *fils_droit ;
        } s ;
    } u ;
} ;
\end{verbatim}
\end {quote}

Un n{\oe}ud de l'arbre est identifié par le code {\tt noeud} et contient
l'opération et ses deux opérandes.  Une feuille est identifiée par le
code {\tt feuille} et contient la valeur entière.


\begin {enumerate}
    \item Étant donnée l'expression spécifiée en notation préfixée~:

	\begin {quote}
	    \tt (* (* 3 (+ 1 2)) (+ 4 5))
	\end {quote}

	\begin {itemize}
	    \item dessinez l'arbre correspondant~;
	    \item donnez l'expression équivalente en notation algébrique.
	\end {itemize}
    
    \item Écrivez une fonction~:

\begin {quote}
\begin {verbatim}
struct noeud *lire_prefixe (void)
\end{verbatim}
\end {quote}

	qui lit sur l'entrée standard une expression en notation préfixée
	et la mémorise sous forme d'un arbre dont la racine est la
	valeur de retour.

    \item Écrivez une fonction~:

\begin {quote}
\begin {verbatim}
void ecrire_prefixe (struct noeud *arbre)
\end{verbatim}
\end {quote}

	qui affiche sur la sortie standard l'arbre en notation préfixée.
    
    \item Écrivez une fonction~:

\begin {quote}
\begin {verbatim}
int profondeur_arbre (struct noeud *arbre)
\end{verbatim}
\end {quote}

	qui renvoie la profondeur maximum de l'arbre.

    \item Écrivez une fonction~:

\begin {quote}
\begin {verbatim}
int evaluer_arbre (struct noeud *arbre)
\end{verbatim}
\end {quote}

	qui renvoie la valeur de l'expression mémorisée dans l'arbre.
	Pour choisir et réaliser une opération, vous utiliserez le choix
	multiple ({\tt switch}) pour sélectionner la fonction {\tt
	ajouter}, {\tt soustraire}, {\tt multiplier} ou {\tt diviser}.
	Ces fonctions prennent les deux opérandes comme arguments et
	renvoient le résultat de l'opération.

    \item Écrivez une fonction~:

\begin {quote}
\begin {verbatim}
void ecrire_algebrique (struct noeud *arbre)
\end{verbatim}
\end {quote}

	qui affiche l'expression mémorisée dans l'arbre comme une
	expression en notation algébrique (infixée). Par exemple,
	l'expression ci-dessus peut être écrite~:
	\verb:((3*(1+2))*(4+5)):.

\end {enumerate}


\question
%%%% Dependances:

Reprendre la fonction d'évaluation d'une expression mémorisée dans un
arbre (exercice~\ref {arbre}). Cette fois-ci, 
l'évaluation d'une opération doit être réalisée à l'aide d'un tableau de
pointeurs sur des fonctions.


\question
%%%% Dependances:

On définit une liste doublement chaînée par~:

\begin {quote}
\small
\begin {verbatim}
struct element
{
    int clef ;                   /* clef de recherche */
    char *valeur ;               /* valeur associee */
    struct element *suivant ;    /* suivant dans la liste */
    struct element *precedent ;  /* precedent dans la liste */
} ;

struct element tete ;
\end{verbatim}
\end {quote}

La variable {\tt tete} est un élément ne contenant aucune clef et aucune
valeur. Le champ {\tt suivant} pointe sur le premier élément de la
liste, le champ {\tt precedent} pointe sur le dernier. La liste doit
toujours être triée suivant la clef.

Écrivez les fonctions~:

\begin {itemize}
    \item \verb:void initialiser_liste (void): \\
	initialise la structure de liste~;
    \item \verb:void ajouter_element (int clef, char *valeur): \\
	ajoute un élément. La valeur doit être recopiée~;
    \item \verb:char *lire_valeur (int clef): \\
	renvoie un pointeur sur la valeur (sans la recopier)~;
    \item \verb:void retirer_element (int clef): \\
	retire l'élément identifié par sa clef.
\end {itemize}



\question
%%%% Dependances: tableaux de chaînes, malloc
    \label {getpath}

La variable Shell {\tt PATH} contient une liste de répertoires séparés
par des caractères ``{\tt :}''.  Par exemple~:
\verb|/bin:/usr/bin:/usr/local/bin|.

Écrivez une fonction qui prenne en paramètre une chaîne et sépare dans
un tableau de chaînes les différents constituants.

Écrivez un programme qui lit le contenu de la variable {\tt PATH}
(avec la fonction {\tt getenv}), sépare les différents constituants à
l'aide de la fonction que vous venez de programmer, et qui affiche
ces différents constituants.


\question
%%%% Dependances: fonctions, pointeurs, malloc, void *

On désire réaliser un module de gestion de pile générique (c'est à dire
qui soit capable de gérer des éléments de type quelconque). Pour cela,
on définit les fonctions suivantes~:

\begin {itemize}
    \item \verb:PILE *initialiser_pile (int taille):\\
	initialise les structures de données nécessaire. Le paramètre
	{\tt taille} est la taille d'un élément de la pile.
    \item \verb:void empiler (PILE *pile, void *valeur):\\
	empile l'élément situé à l'adresse {\tt valeur}~;
    \item \verb:int pile_vide (PILE *pile):\\
	teste si la pile est vide~;
    \item \verb:void depiler (PILE *pile, void *valeur):\\
	dépile l'élément au somment et le recopie à l'adresse spécifiée
	par {\tt valeur}.
\end {itemize}

Définissez le type {\tt PILE} (comme une liste doublement chaînée),
et programmez ces fonctions.


\question

On désire réaliser une fonction de tri polymorphique, c'est-à-dire qui
soit capable de trier des objets de n'importe quel type. Cette fonction
a la syntaxe suivante~:

\begin {quote}
\begin {verbatim}
void quicksort (void *tab [], int debut, int fin,
                int (*comp) (void *, void *))
\end{verbatim}
\end {quote}

Cette fonction utilise l'algorithme du tri rapide ({\em quick sort})
pour trier un les éléments d'indice compris entre {\tt debut} et {\tt
fin} (bornes incluses) d'un tableau d'éléments référencés par un
pointeur générique.  Pour comparer deux éléments, {\tt quicksort}
utilise une fonction prenant en paramètre deux pointeurs génériques et
renvoyant -1 si le premier argument est inférieur au deuxième, 0 s'ils
sont égaux, et 1 si le premier est supérieur au deuxième.

\begin {enumerate}
    \item écrivez un programme qui lit une suite de nombres flottants,
	utilise {\tt quicksort} pour les trier, et enfin affiche le
	tableau trié (on suppose qu'il ne peut pas y avoir plus de 10000
	nombres)~;

    \item écrivez un programme analogue pour trier des chaînes de
	caractères (on suppose qu'il ne peut pas y avoir plus de 10000
	chaînes, et chaque chaîne est limitée à 1000 caractères)~;

    \item écrivez enfin la fonction {\tt quicksort}.

\end {enumerate}


\question

Les sciences physiques sont consommatrices de puissance de calcul.  Les
problèmes traités font souvent apparaître des {\em matrices creuses},
c'est-à-dire des matrices peuplées essentiellement d'éléments nuls.

Par exemple, une matrice $10\thinspace000 \times 10\thinspace000$
représentée sous la forme d'un tableau bidimensionnel occupe
100\thinspace000\thinspace000 fois la taille d'une valeur flottante,
soit environ 400~Mo si 32 bits sont utilisés (ce qui correspond à la
simple précision sur la plupart des processeurs actuels).  Si cette
matrice contient seulement deux nombres non nuls par ligne, il n'y a que
80\thinspace000 octets d'information ``utile'', d'où une perte énorme.

Pour résoudre ce problème, on propose d'utiliser un codage différent
pour les matrices creuses~:  on représente un élément non nul par un
triplet $(i, j, v)$ où $i$ et $j$ sont les numéros de ligne et de
colonne et $v$ est la valeur de cet élément.  Ces éléments sont placés
dans une liste chaînée. Pour simplifier l'implémentation, on ne
cherchera pas à minimiser les temps de calculs.

Programmez les fonctions~:

\begin {quote}
\begin {verbatim}
typedef struct element *matrice ;

matrice ajouter_matrice (matrice M1, matrice M2, int n)
matrice multiplier_matrice (matrice M1, matrice M2, int n)
\end{verbatim}
\end {quote}

Note~: on suppose que les matrices sont de dimension $n \times n$.


