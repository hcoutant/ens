%%%%%%%%%%%%%%%%%%%%%%%%%%%%%%%%%%%%%%%%%%%%%%%%%%%%%%%%%%%%%%%%%%%%%%%%%%%%%%%
% PRIMITIVES DE GESTION DE PROCESSUS
%
% Historique
%   1993/01/31 : pda : création
%   1996/01/06 : pda : séparation en deux TD
%   1996/09/10 : pda : réunion en un seul thème
%%%%%%%%%%%%%%%%%%%%%%%%%%%%%%%%%%%%%%%%%%%%%%%%%%%%%%%%%%%%%%%%%%%%%%%%%%%%%%%

\td {Système~: Gestion des processus}

% \but
% 
% L'objet de ce TD est l'étude des primitives système de
% gestion des processus.


\question

Écrivez un programme générant un processus fils avec la primitive
système {\tt fork}.

\begin {itemize}
    \item le processus fils doit afficher son numéro ({\em pid}) ainsi
	que le numéro du père à l'aide des primitives système {\tt
	getpid} et {\tt getppid}, puis sort (primitive {\tt exit}) avec
	un code de retour égal au dernier chiffre du {\em pid}.

    \item le processus père, quant à lui, affiche le {\em pid} du fils,
	puis attend sa terminaison (primitive {\tt wait}) et affiche son
	code de retour.

\end {itemize}


\question

Écrivez un programme qui lance {\em n} processus fils dans une
première étape puis, dans une deuxième étape, attend leur terminaison
à l'aide de la primitive {\tt wait}.
\`A chaque fois qu'un processus se termine,
le père affiche son numéro ({\em pid}) et son code de retour.
Vous passerez $n$ en argument à votre programme.


\question

Écrivez un programme ayant la syntaxe suivante~:

\vspace* {-3mm}
\begin {quote}
{\tt matproc $n$ $m$}
\end {quote}

L'action de ce programme doit être de générer $n$ processus, chacun
d'entre-eux devant générer $n$ processus à son tour, et ainsi de suite
jusqu'à $m$ niveaux.

Combien de processus sont générés au total ?

\question
% ajouté par Vincent, 12/9/13

Écrivez un programme qui~:
\begin {enumerate}
    \item ouvre un fichier dont le nom est passé en argument (en
	lecture)~;
    \item créé un processus fils~;
    \item puis les deux processus (père et fils) lisent les caractères
	l'un après l'autre dans le fichier en utilisant la primitive système
	{\tt read}, et les affichent à l'écran.
\end {enumerate}
Expérimentez ce programme avec un fichier texte de taille assez grande.
Vérifiez que le bon nombre de caractères a été lu, ou non.

Que se passe-t-il lorsque le fichier est ouvert {\em après} l'appel à
{\tt fork}~?


\question

Écrivez un programme qui~:

\begin {enumerate}
    \item affiche d'abord l'heure, sous forme de secondes et de
	micro-secondes telles que retournées par \texttt {gettimeofday}~;

    \item appelle ensuite (avec une des primitives \texttt {exec})
	la commande \texttt {ls} avec l'option \texttt {-l} sur un
	répertoire passé en paramètre~;

    \item affiche enfin l'heure (comme précédemment) ainsi que le
	temps écoulé pendant l'appel de \texttt {ls}.

\end {enumerate}


\question

Écrivez un programme qui~:

\begin {enumerate}
    \item lance la commande {\tt ls} sur un répertoire passé en
        paramètre~;
    \item redirige la sortie standard de {\tt ls} sur {\tt /dev/null}~;
    \item affiche les temps (réel et processeur) en secondes
	pris par la commande {\tt ls} (primitive système {\tt times}).
\end {enumerate}


\question

Écrivez un programme admettant les arguments suivants~:

\begin {quote}
    \texttt {parexec \textit {n cmd arg ...}}
\end {quote}

Votre programme doit appeller $n$ fois (avec \texttt {fork} et \texttt
{execvp}) la commande spécifiée en paramètre en lui passant les
arguments (en nombre non connu) fournis, en ajoutant le rang d'appel $i$
($0 \leq i < n$) comme argument supplémentaire à la fin.

Ainsi, l'appel~:

\begin {quote}
    \texttt {parexec 5 sh tester toto}
\end {quote}

doit provoquer l'exécution de~:

\begin {itemize}
    \item \texttt {sh tester toto 0}
    \item \texttt {sh tester toto 1}
    \item ...
    \item \texttt {sh tester toto 4}
\end {itemize}

Les $n$ commandes doivent être lancées en parallèle, et le code de
retour de \texttt {parexec} doit être égal à 0 si toutes les commandes
réussissent (i.e. renvoient un code de retour nul), ou 1 si au moins
l'une des commandes échoue (i.e. renvoie un code de retour non nul).


\question

Rédigez un programme qui prend en argument deux nombres $n$ et $m$.

\begin {itemize}
    \item Dans une première étape, votre programme doit générer
	$n$ processus fils. Chaque processus fils $i$ (avec $1 \leq
	i \leq n$) doit générer un nombre aléatoire $a_i$ (avec la
	fonction de bibliothèque \texttt {rand}) compris entre 1 et $m$,
	bornes incluses, attendre (avec la fonction de bibliothèque
	\texttt {sleep}) $a_i$ secondes, puis se terminer avec un code de
	retour égal à $a_i$.

    \item Dans une deuxième étape, votre programme doit attendre la
	terminaison des $n$ processus fils et récupérer les différentes
	valeurs $a_i$ retournées.

    \item Dans une troisième étape, votre programme doit générer un
	nouveau processus pour exécuter la commande \texttt {expr}
	(\texttt {/usr/bin/expr}) en lui passant en arguments $a_1$,
	puis \texttt {+}, puis $a_2$ , puis \texttt {+}, etc jusqu'à
	$a_n$, soit $2n-1$ arguments en plus de \texttt {expr} (ex:
	\verb*|expr 1 + 7 + 2 + 3|), et en redirigeant la sortie
	standard de cette commande vers un fichier de nom \texttt {toto}.

    \item Enfin, dans une quatrième et dernière étape, votre programme
	attend la terminaison de ce dernier processus, puis lit le
	fichier \texttt {toto} et affiche son contenu.

\end {itemize}

Bien sûr, vous n'utiliserez dans votre programme que des primitives
systèmes (à l'exception de \texttt {rand} et \texttt {sleep}, ou des
fonctions de manipulation de chaînes de caractères comme \texttt {atoi}
ou \texttt {snprintf}).


\question

Écrivez un programme C équivalent au script shell suivant, qui prend
en argument un nom d'utilisateur :

\begin {quote}
    \verb:ps eaux > toto ; grep "^$1 " < toto > /dev/null && echo "$1 est connecté":
\end {quote}

Votre programme devra :

\begin {itemize}
    \item réaliser effectivement les exécutions de \texttt {ps} et
	\texttt {grep} en utilisant la primitive \texttt {execlp} ;

    \item mettre en place les redirections des entrées/sorties
	nécessaires grâce à la primitive \texttt {dup} (ou \texttt
	{dup2}) ;

    \item réaliser l'affichage final avec la primitive \texttt {write}.

\end {itemize}


\question

Écrivez un programme qui~:

\begin {enumerate}
    \item attende une ligne (une commande) sur l'entrée standard (vous
	pouvez utiliser la fonction de librairie {\tt fgets})~;

    \item la décompose en mots séparés par des espaces~;

    \item recherche le premier mot dans le {\tt PATH}~;

    \item exécute cette commande par le biais de la primitive système
	{\tt execv}~;
    
    \item revienne au point 1.

\end {enumerate}

Félicitations, vous avez écrit un {\em shell}~!


\question

Reprenez le programme de l'exercice précédent et ajoutez les
redirections dans des fichiers (à l'aide des mots $<$, $>$ ou les
deux).


\question
\label{q:minishell}

Reprenez le programme de l'exercice précédent et ajoutez un traitement
spécial dans le cas où la commande est terminée par le mot \verb:&:.

