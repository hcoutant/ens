%%%%%%%%%%%%%%%%%%%%%%%%%%%%%%%%%%%%%%%%%%%%%%%%%%%%%%%%%%%%%%%%%%%%%%%%%%%%%%%
% ENVIRONNEMENT D'UN PROCESSUS, SIGNAUX V7, SIGNAUX POSIX
%
% Historique
%   1996/02/27 : pda : création
%   1996/09/10 : pda : séparation de l'environnement
%   1997/08/26 : pda : ajout de l'exercice pwd
%%%%%%%%%%%%%%%%%%%%%%%%%%%%%%%%%%%%%%%%%%%%%%%%%%%%%%%%%%%%%%%%%%%%%%%%%%%%%%%

\td {Système~: Environnement d'un processus}

% \but
% 
% L'objet de ce TD est l'approfondissement de notions souvent
% rencontrées dans le contexte des processus.


\question

La variable globale {\tt environ} est déclarée comme~:

\begin {quote}
    \verb|extern char **environ|
\end {quote}

Cette variable est automatiquement définie. Elle référence le premier
élément d'un tableau de chaînes de la forme {\em var{\tt=}valeur}. Ce
tableau est terminé par le pointeur {\tt NULL}.

Dessinez cette structure de données.

Écrivez la fonction de librairie {\tt getenv}.


\question

La fonction de librairie {\tt system} prend en argument une commande
(contenant éven\-tuel\-lement des redirections, ou même composée de
plusieurs commandes reliées par un tube), l'exécute et renvoie
le code de retour de la commande si elle a été lancée, ou -1
sinon.

Vous utiliserez le Shell de Bourne ({\tt /bin/sh}) avec l'option {\tt
-c} pour exécuter la commande.


\question

Il peut arriver qu'un programme désire savoir si l'entrée ou la
sortie standard a été redirigée.

Donnez des exemples de programmes dont le comportement dépend de la
redirection de l'entrée ou de la sortie standard.

Écrivez la fonction {\tt isatty} qui prend en argument un descripteur
de fichier, et renvoie 1 si c'est un terminal (ou plus généralement un
périphérique en mode caractère) ou 0 sinon.


\question

Les champs {\tt st\_dev} et {\tt st\_ino} de la structure {\tt stat}
identifient de manière unique un fichier dans le système. Écrivez une
fonction {\tt ttyname} qui prend en paramètre un descripteur de fichier,
et renvoie un pointeur sur une chaîne (statique) contenant le nom
complet du fichier (cherché dans {\tt /dev}) correspondant.


\question

Écrivez une nouvelle version de la fonction {\tt getcwd}.  Pour cela,
on cherchera le numéro d'inode du répertoire courant.  Puis, on
cherchera dans le répertoire parent le nom correspondant à l'inode du
répertoire courant.  En répétant cette opération jusqu'au répertoire
d'inode numéro 2 (la racine du système de fichiers), on peut
reconstituer le nom du répertoire courant.

On notera que cette méthode ne tient compte que du système de fichiers
courant. On n'essayera pas de s'affranchir de cette limitation.


% getlogin
% 
% sleep
% 
% simuler des tubes (popen) avec des fichiers

