%%%%%%%%%%%%%%%%%%%%%%%%%%%%%%%%%%%%%%%%%%%%%%%%%%%%%%%%%%%%%%%%%%%%%%%%%%%%%%%
% LANGAGE C : PROGRAMMATION AVANCEE
%
% Historique
%   1993/02/03 : pda : création
%   1995/10/06 : pda : adaptation pour 95/96
%   1996/01/03 : pda : ajout de l'exercice malloc/free
%   1996/09/10 : pda : réunion en un seul thème
%   1997/08/26 : pda : ajout des questions pour l'analyse des makefiles
%%%%%%%%%%%%%%%%%%%%%%%%%%%%%%%%%%%%%%%%%%%%%%%%%%%%%%%%%%%%%%%%%%%%%%%%%%%%%%%


\td {Langage~C~: Programmation avancée}

% \but
% 
% L'objet de ce TD est d'appréhender quelques points avancés du
% langage~C~: compilation séparée, programmation portable, 
% fonctions à nombres d'arguments variables, etc.


\question
%%%% Dependances: stdarg

Écrivez en C une fonction {\tt afficher\_couples}~:

\begin {quote}
    \verb|void afficher_couples (int nbcouples, ...)|
\end {quote}

qui prend un nombre de couples, puis une suite de couples
(chaîne, entier) à afficher sur la sortie standard. Votre fonction
doit prendre un nombre variable d'arguments (voir \texttt {stdarg.h})


\question

On désire écrire un programme pour calculer la circonférence et la
surface d'un cercle étant donné son rayon.  Le programme doit être
composé de plusieurs fichiers~:

\begin {quote}
    \renewcommand {\arraystretch} {1}
    \begin {tabular} {ll}
	\tt principal.c & contient la fonction {\tt main} \\
	\tt circonference.c & contient la fonction {\tt circonference} \\
	\tt surface.c & contient la fonction {\tt surface} \\
	\tt pi.h & contient la définition de la valeur approchée de $\pi$
    \end {tabular}
\end {quote}


\begin {enumerate}
    \item
	Écrivez le contenu des différents fichiers, et
	donnez la commande minimum de compilation du programme.

    \item
	Décomposez cette commande de manière à avoir une commande pour
	la compilation de chaque fichier, puis une commande pour
	l'édition de liens.  \\
	Si vous modifiez le fichier {\tt surface.c}, donnez les
	commandes minimum pour recompiler le programme.  Même question
	pour la modification du fichier {\tt pi.h}.

    \item
	Représentez les dépendances de l'exercice précédent sous forme
	de graphe étiqueté.  Écrivez le fichier {\tt Makefile}
	correspondant.

	Ajoutez une cible fictive {\tt clean} dans le {\tt Makefile}
	pour supprimer tous les fichiers intermédiaires ainsi que le
	fichier résultat.

\end {enumerate}


\question

On désire connaître le nombre d'occurrences de chaque mot lu sur
l'entrée standard.  Par exemple, le texte ``{\em il fait beau, n'est il
pas ?}'' est constitué de :

\begin {quote}
    \renewcommand {\arraystretch} {0.9}
    \tt
    \begin {tabular} {ll}
	il   & : 2 fois \\
	fait & : 2 fois \\
	beau & : 1 fois \\
	n    & : 1 fois \\
	est  & : 1 fois \\
	pas  & : 1 fois 
    \end {tabular}
\end {quote}

Pour cela, on définit les fonctions~:

\begin {itemize}
    \item \verb:void initialiser_mots (void): \\
	qui initialise la structure de données associée aux mots~;

    \item \verb:struct mot *trouver_mot (char texte []): \\
	qui trouve un mot, et le crée s'il n'est pas trouvé~;

    \item \verb:void mettre_a_jour_mot (struct mot *mot): \\
	qui met à jour (incrémente) le compteur associé au mot~;

    \item \verb:void afficher_mots (void): \\
	qui affiche le résultat final.

\end {itemize}


\begin {enumerate}
    \item Écrivez le programme en supposant que les fonctions
	ci-dessus existent déjà et sont dans un autre module (fichier).
	Vous placerez dans un fichier {\tt mot.h} les déclarations des
	prototypes de ces fonctions.  Y-a-t'il besoin de placer la
	définition de la structure {\tt mot} dans ce fichier~?

    \item Écrivez les fonctions ci-dessus en prenant comme structure
	de données une liste simplement chaînée~:

\begin {quote}
\small
\begin {verbatim}
struct mot
{
    char *texte ;          /* texte associe au mot */
    int   nb_occurrences ; /* nb d'occurrences du mot */
    struct mot *suivant ;  /* liste chainee */
} ;
struct mot *tete ;
\end{verbatim}
\end {quote}

    \item Écrivez une nouvelle version des fonctions ci-dessus en
	utilisant une table de hachage.

\end {enumerate}


\question

\`A l'aide des mécanismes de compilation conditionnelle offerts par le
préprocesseur, reprenez l'exercice précédent et intégrez les deux
structures de données pour représenter les mots (liste chaînée et table
de hachage) en un seul fichier. La personne qui compile votre programme
doit pouvoir sélectionner l'une ou l'autre structure en définissant un
seul symbole ou en ne le définissant pas.


\question

Sur Linux, la page de manuel {\em p} du chapitre {\em n} est localisée
dans le fichier de nom
{\tt /usr/share/man/man{\em n}.gz/{\em p}.{\em n}}. Cette page est
compressée, et la commande pour la formatter est~:
\verb:gunzip | nroff -man: en supposant que le fichier est présenté sur
l'entrée standard.

Sur HP-UX, la page de manuel {\em p} du chapitre {\em n} est localisée
dans le fichier de nom
{\tt /usr/share/man/man{\em n}.Z/{\em p}.{\em n}}. Cette page est
compressée, et la commande pour la formatter est~:
\verb:zcat | nroff -man: en supposant que le fichier est présenté sur
l'entrée standard.

Sur SunOS, la page de manuel {\em p} du chapitre {\em n} est localisée
dans le fichier de nom
{\tt /usr/man/man{\em n}/{\em p}.{\em n}}. Cette page n'est pas
compressée, et la commande pour la formatter est~:
\verb:nroff -man: toujours en supposant que le fichier est présenté sur
l'entrée standard.

Écrivez une commande {\tt man}. Votre programme doit être portable
sur Linux, HP-UX ou sur SunOS, en utilisant le symbole {\tt linux},
{\tt hpux} ou {\tt sun} du préprocesseur. Vous utiliserez la fonction
{\tt popen} pour formatter la page de manuel.

Rappel~: la vraie commande  {\tt man} admet 1 ou 2 paramètres~:

\begin {quote}
    {\tt man} [ {\em n} ] {\em p}
\end {quote}

Si le chapitre {\em n} n'est pas fourni, la page correspondant à {\em p}
est cherchée dans l'ensemble des chapitres. On supposera que le chapitre
est un entier entre 1 et 8.


%\question
%
%Analysez les fichiers {\tt Makefile} ci-après (voir pages~\pageref
%{makefile} à~\pageref {makefile-fin}), et répondez aux questions
%suivantes~:
%
%\begin {enumerate}
%    \item Si on appelle {\tt make} à partir du répertoire {\tt demo/lib},
%	quel est le fichier créé~? Même question pour les répertoires
%	{\tt demo/src} et {\tt demo/doc}.
%
%    \item Expliquez l'enchaînement des appels à {\tt make} lors de
%	l'appel initial à {\tt make} dans le répertoire {\tt demo}.
%
%    \item Dans le fichier {\tt demo/Makefile}, pourquoi les {\tt
%	antislashes} (\verb:\:) sont-ils nécessaires dans la cible {\tt
%	subdirs}~?
%
%    \item Expliquez l'action effectuée lors de la construction de la
%	cible {\tt pcc.o} si {\tt make} est appelé depuis le répertoire
%	{\tt demo/src}. Même question si {\tt make} est appelé depuis le
%	répertoire {\tt demo}.
%
%    \item Comment fait-on pour adapter l'ensemble du programme à une
%	autre version de système d'exploitation~? Où sont concentrées
%	les modifications à apporter~? Pourquoi~?
%
%    \item Expliquez les actions effectuées lors de la construction de la
%	cible {\tt manuel.ps}.
%
%    \item Pourquoi le fichier {\tt demo.o} (répertoire {\tt demo/src}) est
%	construit avec une méthode différente des autres fichiers~?
%
%\end {enumerate}
%
%
%\cleardoublepage
%    \label {makefile}
%
%\subsection* {Fichier demo/Makefile}
%
%{\small
%\listing {mf1}
%}
%
%\clearpage
%\subsection* {Fichier demo/lib/Makefile}
%
%{\small
%\listing {mf2}
%}
%
%\clearpage
%\subsection* {Fichier demo/src/Makefile}
%
%{\small
%\listing {mf3}
%}
%
%\subsection* {Fichier demo/doc/Makefile}
%
%{\small
%\listing {mf4}
%}
%\label {makefile-fin}
