%%%%%%%%%%%%%%%%%%%%%%%%%%%%%%%%%%%%%%%%%%%%%%%%%%%%%%%%%%%%%%%%%%%%%%%%%%%%%%%
% SOCKETS
%
% Historique
%   2008/01/14 : pda : création
%%%%%%%%%%%%%%%%%%%%%%%%%%%%%%%%%%%%%%%%%%%%%%%%%%%%%%%%%%%%%%%%%%%%%%%%%%%%%%%

\td {Réseau~: Sockets}

% \but
% 
% L'objet de ce TD est l'étude des primitives systèmes de programmation
% réseau (sockets) et des fonctions de bibliothèque associées.

\question

Écrivez un programme qui prend en paramètre un nom de hôte (tel que
\texttt {mailhost.u-strasbg.fr} par exemple), une adresse IPv4 ou
une adresse IPv6. Pour chaque adresse trouvée par la fonction \texttt
{getaddrinfo}, votre programme doit afficher la famille d'adresses
(IPv4 ou IPv6) et l'adresse avec la fonction \texttt {inet\_ntop}.


\question

Modifiez votre programme précédent pour ne sélectionner qu'une
famille d'adresses (IPv4 ou IPv6).



\question

Écrivez un client UDP pour le protocole Echo (service «~\texttt
{echo/udp}~»).

Écrivez un serveur UDP pour le protocole Echo (service «~\texttt
{echo/udp}~»).

\question

Même exercice que précédemment en mode TCP.


\question

Écrivez à l'aide du protocole TCP un client et le serveur qui
échangent des valeurs. On adoptera le protocole décrit par la
structure suivante~:

\begin {quote}
\begin {verbatim}
struct proto {
    uint16_t nint ;          /* nombre de valeurs émises */
    int32_t valeurs [MAX] ;  /* les valeurs elles-mêmes */
} ;
\end{verbatim}
\end {quote}

Le client accepte comme premier argument un nom de hôte ou une
adresse IP (v4 ou v6), et comme arguments suivants la liste des
valeurs à transmettre. Les arguments suivants seront les valeurs.
Pour chaque connexion de client, le serveur doit lire la liste de
valeurs, et renvoyer au client les valeurs élevées au carré.

Les données doivent être échangées, même si les deux ordinateurs
ont des formats de stockage des entiers différents.


\question

Modifiez le programme de l'exercice précédent~: si la connexion
vient d'une adresse IPv4, le serveur renvoie les nombres élevés à
la puissance 4. Si la connexion vient d'une adresse IPv6, le serveur
renvoie les nombres élevés à la puissance 6.


\question

Programmez un système de discussion multi-utilisateurs (\textit
{chat\/}). Chaque client se connecte sur le serveur. Toute ligne
saisie par l'utilisateur est envoyée par le client vers le serveur.
Toute ligne reçue depuis le serveur est affichée à l'écran. Lorsque
le serveur reçoit une ligne d'un client, il la retransmet à tous
les autres clients.


\question

L'objet de cet exercice est de rédiger un programme pour distribuer
des jetons à des clients IP. On représente un jeton par un entier
de 1 à $n$. Le serveur accepte des connexions des clients.

Si le client demande un jeton, le serveur lui attribue le premier
jeton disponible et mémorise l'adresse IP du client et la date
d'attribution. Ces informations (jetons, adresse IP associée et
date d'obtention) doivent être mémorisées dans un segment de mémoire
partagée.

Si un client demande à qui est attribué un jeton, le serveur répond
avec l'adresse IP du client, ou un code en cas de jeton non associé.

Enfin, un jeton est considéré comme abandonné au bout d'un certain
délai. En conséquence, un client peut demander à rafraîchir
l'association d'un jeton avant l'expiration du délai. Pour ce faire,
il envoie un message au serveur avec le numéro du jeton à rafraîchir.
Si l'adresse IP concorde, le serveur doit réinitialiser la date
d'obtention du client.

\begin {enumerate}
    \item spécifiez le protocole que le serveur et les clients doivent
	utiliser, de telle sorte que vous puissiez utiliser le programme
	\texttt {telnet} comme client primitif~;
    \item rédigez le serveur, qui doit prendre en paramètre le
	nombre de jetons initialement disponibles~;
    \item rédigez le client d'obtention de jeton~;
    \item rédigez le client d'interrogation de jeton~;
    \item rédigez le client de rafraîchissement de jeton.
\end {enumerate}
