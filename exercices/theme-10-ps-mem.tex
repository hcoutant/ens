%%%%%%%%%%%%%%%%%%%%%%%%%%%%%%%%%%%%%%%%%%%%%%%%%%%%%%%%%%%%%%%%%%%%%%%%%%%%%%%
% GESTION MEMOIRE
%
% Historique
%   2015/03/29 : pda : création
%%%%%%%%%%%%%%%%%%%%%%%%%%%%%%%%%%%%%%%%%%%%%%%%%%%%%%%%%%%%%%%%%%%%%%%%%%%%%%%

\td {Système~: Gestion mémoire}


\question

Quelles sont, sur \texttt {turing}, les tailles des types suivants~:
\begin {itemize}
    \item \texttt {int}
    \item \texttt {char}
    \item \texttt {char *}
    \item \verb|struct { char c ; char *pc ; int i ; }|
    \item \verb|struct { char c ; char *pc ; int i ; } *|
\end {itemize}

Expliquez ce que vous constatez pour la taille de la structure.


\question

Donnez la définition d'une macro \verb|ADRESSE_RELATIVE_DE(s,c)|
permettant d'obtenir l'adresse du champ \texttt {c} d'une structure
\texttt {s} relativement au début de la structure. L'adresse relative
renvoyée par votre macro doit être exprimée en octets.


\question

Pour détecter les corruptions mémoire, on propose d'utiliser la
technique dite du «~canari~»~: à chaque fois qu'une zone mémoire est
allouée, \texttt {malloc} initialise un espace avant et un espace après
avec des valeurs convenues\footnote {On prendra par exemple les 8 octets
\texttt {0x01}, \texttt {0x23} ... \texttt {0xef}.}
(fixes).  Lorsqu'on libère la zone, \texttt {free} vérifie que ces
valeurs convenues sont toujours présentes. Si elles ne le sont pas,
\texttt {free} signale l'erreur.

Écrivez les fonctions \texttt {mon\_malloc} et \texttt {mon\_free}
qui fonctionnent comme décrit ci-dessus. Vos fonctions appelleront
les vraies \texttt {malloc} et \texttt {free} pour allouer et libérer
effectivement la mémoire.


\question

À l'aide de la primitive système \texttt {mmap}, écrivez un programme
qui prend en argument un nom de fichier et compte le nombre de lignes
qui s'y trouvent.
