\def\inc{inc2-1-intro}

\titreA {Introduction}

%%%%%%%%%%%%%%%%%%%%%%%%%%%%%%%%%%%%%%%%%%%%%%%%%%%%%%%%%%%%%%%%%%%%%%%%%%%%%%
% Support matériel
%%%%%%%%%%%%%%%%%%%%%%%%%%%%%%%%%%%%%%%%%%%%%%%%%%%%%%%%%%%%%%%%%%%%%%%%%%%%%%

\titreB {Support matériel}

\begin {frame} {Rappel : le noyau}
    Définition :

    \begin {quote}
	Le noyau est constitué de l'ensemble minimum des primitives
	systèmes nécessaires pour atteindre les deux objectifs
	fondamentaux :

	\begin {itemize}
	    \item partager équitablement les ressources
	    \item garantir la sécurité des données
	\end {itemize}

    \end {quote}
\end {frame}

\begin {frame} {Support matériel}
    Support matériel indispensable pour assumer le rôle de gardien
    des ressources :

    \begin {enumerate}
	\item processeur avec (minimum) deux modes d'exécution

	    \begin {itemize}
		\item l'accès direct aux ressources est réservé au seul
		    noyau
		\item sinon : n'importe quel programme peut accéder
		    aux ressources sans contrôle
	    \end {itemize}

	\item mécanisme d'interruptions matérielles

	    \begin {itemize}
		\item permet de bénéficier du parallélisme des
		    périphériques
		\item sinon : noyau doit surveiller l'achèvement
		    des requêtes

	    \end {itemize}

	\item horloge capable d'interrompre périodiquement le processeur

	    \begin {itemize}
		\item permet au noyau de reprendre le contrôle
		\item sinon : système non préemptif
	    \end {itemize}

	\item mécanisme de traduction/protection mémoire

	    \begin {itemize}
		\item assure l'indépendance des espaces mémoire
		\item sinon : pas de protection des processus les uns par
		    rapport aux autres
	    \end {itemize}

    \end {enumerate}

\end {frame}

\begin {frame} {Support matériel -- Modes d'exécution}
    Mode d'exécution caractérisé par un bit dans le registre d'état (SR)
    du processeur
    \begin {itemize}
	\item mode non privilégié :
	    \begin {itemize}
		\item mode d'exécution pour le code utilisateur
		\item certaines instructions sont interdites
		\item tentative d'exécution d'une instruction privilégiée
		    \implique
		    exception (interruption interne au processeur)
		\item même les programmes de l'administrateur sont
		    exécutés en mode non privilégié (code utilisateur)
	    \end {itemize}
	\item mode privilégié :
	    \begin {itemize}
		\item mode d'exécution pour le noyau
		\item accès à toutes les instructions du processeur
	    \end {itemize}
    \end {itemize}
\end {frame}

\begin {frame} {Support matériel -- Modes d'exécution}
    Exemples d'instructions privilégiées (à adapter en fonction du
    type de processeur) :

    \ctableau {\fC} {|l|l|} {
	IN, OUT & instructions d'E/S : accès direct aux périphériques \\
	écriture dans SR & pourrait modifier le mode d'exécution courant \\
	RESET & réinitialise le processeur \\
	RTI & retour d'interruption \\
    }
\end {frame}


%%%%%%%%%%%%%%%%%%%%%%%%%%%%%%%%%%%%%%%%%%%%%%%%%%%%%%%%%%%%%%%%%%%%%%%%%%%%%%
% Noyau
%%%%%%%%%%%%%%%%%%%%%%%%%%%%%%%%%%%%%%%%%%%%%%%%%%%%%%%%%%%%%%%%%%%%%%%%%%%%%%

\titreB {Le noyau}

\begin {frame} {Le noyau}
    Le noyau est chargé en mémoire au démarrage de l'ordinateur
    \begin {itemize}
	\item seul programme à utiliser le mode « privilégié »
	\item initialise le processeur, la mémoire, les périphériques
	\item « répond » aux requêtes des processus \\
	    \implique le noyau implémente les primitives système
    \end {itemize}

    \vspace* {3mm}
    Le noyau est un programme autonome :
    \begin {itemize}
	\item pas d'utilisation de bibliothèques externes
	\item édition de liens \implique un binaire unique et autonome
	\item ajout dynamique de modules
	    \begin {itemize}
		\item ajout de nouvelles fonctions
		    \begin {itemize}
			\item pilotes, systèmes de fichiers, etc.
			\item mise à jour de tables (ex: table des pilotes)
		    \end {itemize}
		\item édition de liens complémentaire au chargement
	    \end {itemize}
    \end {itemize}
\end {frame}

\begin {frame} {Le noyau}
    Trois « accès » possibles au noyau :

    \vspace* {2mm}

    \begin {minipage} [c] {.35\linewidth}
	\includegraphics [width=1\linewidth] {\inc/acces}
    \end {minipage}
    \hfill
    \begin {minipage} [c] {.64\linewidth}
	% \fC
	\begin {itemize}
	    \item interruption matérielle
		\begin {itemize}
		    \item générée par un périphérique \\
			exemple : fin d'entrée/sortie
		    \item générée par l'horloge \\
			exemple : fin de quantum
		\end {itemize}
	    \item exception provoquée par le processus
		\begin {itemize}
		    \item violation de segment
		    \item division par 0
		    \item etc.
		\end {itemize}
	    \item appel d'une primitive système
		\begin {itemize}
		    \item instruction spéciale du processeur
		    \item considéré comme une exception
		\end {itemize}
	\end {itemize}
    \end {minipage}

    \vspace* {3mm}

    Traitement similaire des 3 types d'accès
\end {frame}

\begin {frame} {Rappel : traitement des interruptions}
    Actions du processeur suite à une interruption/exception~:
    \begin {enumerate}
	\item lorsque l'événement se produit, PC pointe dans le code du
	    processus, SP dans la pile du processus et SR indique qu'on
	    est en mode «~non privilégié~» (par exemple)

	\item le processeur sauvegarde ces registres

	\item le processeur modifie ensuite ces registres :
	    \begin {itemize}
		\item SR :
		    \begin {itemize}
			\item passage en mode « privilégié »
			\item blocage (masquage) éventuel des interruptions
		    \end {itemize}

		\item PC : initialisé à partir du vecteur d'interruption
		    ou d'exception

		\item SP : pointe sur la pile noyau
	    \end {itemize}

    \end {enumerate}

    \vspace* {2mm}

    \implique tout ceci est effectué par le matériel
\end {frame}

\begin {frame} {Rappel : traitement des interruptions}
    Vecteur d'interruptions :

    \begin {itemize}
	\item tableau d'adresses de fonctions internes au noyau
	\item placé à une adresse fixée pour le processeur
	\item indexé par le numéro de l'interruption
	    \begin {itemize}
		\item exemple : interruptions clavier,
		    interruptions disque, etc
	    \end {itemize}
	\item initialisé par le noyau au démarrage du système
    \end {itemize}

    \vspace* {3mm}

    Vecteur similaire pour les exceptions

    \begin {itemize}
	\item l'instruction d'appel aux primitives système est
	    considérée comme une exception

	    \begin {itemize}
		\item rappel : instruction non privilégiée
	    \end {itemize}
    \end {itemize}

\end {frame}

\begin {frame} {Rappel : traitement des interruptions}
    \begin {center}
	\includegraphics [width=.9\textwidth] {\inc/ps-except}
    \end {center}
\end {frame}

\begin {frame} {Rappel : traitement des interruptions}
    Une fois le contexte (PC, SP, SR) initialisé, le processeur exécute
    le code du noyau~:

    \begin {enumerate}
	\item (en assembleur) sauvegarde du reste du contexte CPU
	    (registres généraux, etc.)

	\item (en assembleur) mise en place d'un contexte de pile
	    pour un appel de procédure en langage de haut niveau (ex: C)

	\item (en assembleur) branchement à une adresse

	\item (en C) vérification de la raison de l'interruption

	    \begin {itemize}
		\item interrogation des contrôleurs de périphériques
		    pour identifier l'origine de l'interruption
	    \end {itemize}

	\item (en C) action correspondant à l'interruption

    \end {enumerate}
\end {frame}

\begin {frame} {Rappel : traitement des interruptions}
    Au retour~:

    \begin {itemize}
	\item actions logicielles symétriques en fin d'exception (en
	    C puis en assembleur)

	\item actions (en matériel) symétriques à la prise en compte
	    de l'exception : instruction spéciale (IRET pour x86,
	    RTE pour 68000)

    \end {itemize}
\end {frame}

\begin {frame} {Appel des primitives systèmes}
    En réalité, les primitives sont des fonctions de bibliothèque...
    \begin {center}
	\includegraphics [width=.9\textwidth] {\inc/trap-lib}
    \end {center}
\end {frame}

\begin {frame} {Appel des primitives systèmes}
    Les appels à des primitives sont des fonctions de bibliothèque :
    \begin {itemize}
	\item compilateur : un appel à une primitive
	    système est traité comme un appel à n'importe quelle
	    fonction C

	    \begin {itemize}
		\item aucune spécificité des primitives systèmes
	    \end {itemize}

	\item bibliothèque standard du C : ajout d'une fonction

	    \begin {itemize}
		\item codée en assembleur
		    \begin {itemize}
			\item squelette identique pour quasiment
			    toutes les primitives
		    \end {itemize}
		\item instruction spéciale du processeur (ex: \code {trap})
		\item retour en fonction de la retenue (bit « carry »)

		    \ctableau {\fD} {|c|c|} {
			\textbf {carry} & \textbf {registre r0 en retour}
			    \\
			0 & valeur de retour de la primitive \\
			1 & valeur à mettre dans \code {errno} \\
		    }
	    \end {itemize}
    \end {itemize}
\end {frame}

%%%%%%%%%%%%%%%%%%%%%%%%%%%%%%%%%%%%%%%%%%%%%%%%%%%%%%%%%%%%%%%%%%%%%%%%%%%%%%
% Démarrage
%%%%%%%%%%%%%%%%%%%%%%%%%%%%%%%%%%%%%%%%%%%%%%%%%%%%%%%%%%%%%%%%%%%%%%%%%%%%%%

\titreB {Démarrage du noyau}

\begin {frame} {Démarrage du noyau}
    Le noyau est chargé en mémoire au démarrage de l'ordinateur

    \vspace* {3mm}

    Plusieurs phases :
    \begin {enumerate}
	\addtocounter {enumi} {-1}
	\item interrupteur sur « on » (ou processeur réinitialisé) :
	    \begin {itemize}
		\item BIOS (ou équivalent, programme en ROM ou Flash)
		    démarre
		\item il initialise la mémoire et les périphériques
		\item il cherche les disques dont le secteur 0 contient
		    un numéro magique identifiant un secteur bootable
		    \implique choix
	    \end {itemize}
	\item le BIOS charge le secteur 0 du disque sélectionné
	    \begin {itemize}
		\item ce programme doit tenir dans un secteur \implique petit !
		\item c'est le chargeur primaire
		\item insuffisant pour lire le noyau (dépend du système
		    de fichiers) \implique besoin d'un chargeur secondaire
		\item ce programme charge un autre programme qui peut être
		    plus grand \implique emplacement conventionnel
	    \end {itemize}
    \end {enumerate}
\end {frame}

\begin {frame} {Démarrage du noyau}
    \begin {enumerate}
	\addtocounter {enumi} {1}
	\item le chargeur primaire charge le chargeur secondaire
	    \begin {itemize}
		\item ce programme est plus grand
		\item il est capable de lire le système de fichiers
		\item \implique capable de lire les blocs du noyau
		    éparpillés dans le secteur de fichiers
	    \end {itemize}

	\item le chargeur secondaire analyse le système de fichiers
	    pour lire les blocs du noyau
	    \begin {itemize}
		\item une fois le noyau chargé, le contrôle est
		    transféré \\
		    (PC $\leftarrow$ adresse du noyau en mémoire)
	    \end {itemize}
    \end {enumerate}

    Méthode alternative : mémoriser l'emplacement des blocs du
    noyau sur le disque pour éliminer le chargeur secondaire
\end {frame}

\begin {frame} {Démarrage du noyau}
    Le noyau est chargé en mémoire au démarrage de l'ordinateur
    \begin {center}
	\includegraphics [width=.9\textwidth] {\inc/boot}
    \end {center}
\end {frame}

\begin {frame} {Démarrage du noyau}
    Et après le chargement du noyau ?

    \begin {itemize}
	\item le noyau crée un descripteur de processus pour le pid 0
	    \begin {itemize}
		\item à partir de ce moment, le noyau exécute un processus
	    \end {itemize}
	\item ce processus (descripteur) est ensuite dupliqué \implique pid = 1
	\item le nouveau processus fait \code {execl ("/sbin/init", ...)}
	    \begin {itemize}
		\item version interne de \code {execl}
		\item chargement du code de \code {/sbin/init} en mémoire
		\item « retour » de primitive système \implique exécution
		    de \code {main}
		\item \code {/sbin/init} (ou \code {systemd}) exécute
		    de nouvelles primitives systèmes
		    \begin {itemize}
			\item lancement des scripts de démarrage
			\item lancement des démons
		    \end {itemize}
	    \end {itemize}
	\item c'est parti...
    \end {itemize}
\end {frame}

