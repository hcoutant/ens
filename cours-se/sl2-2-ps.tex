\def\inc{inc2-2-ps}

\titreA {Processus}

%%%%%%%%%%%%%%%%%%%%%%%%%%%%%%%%%%%%%%%%%%%%%%%%%%%%%%%%%%%%%%%%%%%%%%%%%%%%%%
% Constituants d'un processus
%%%%%%%%%%%%%%%%%%%%%%%%%%%%%%%%%%%%%%%%%%%%%%%%%%%%%%%%%%%%%%%%%%%%%%%%%%%%%%

\titreB {Constituants d'un processus}

\begin {frame} {Constituants d'un processus}
    Rappel :

    \begin {quote}
	un processus est décrit par :
	\begin {itemize}
	    \fB
	    \item un espace mémoire pour le programme et les données
	    \item des attributs
	    \item un contexte matériel
		\begin {itemize}
		    \fC
		    \item registres du processeur
		    \item traduction d'adresses
		\end {itemize}
	\end {itemize}
    \end {quote}

    { \fB (voir \code {task\_struct} sur
	\url {http://www.tldp.org/LDP/tlk/ds/ds.html} )}

    \vspace* {2mm}

    À chaque fois :
    \begin {itemize}
	\fB
	\item qu'un processus est retiré du processeur
	    \begin {itemize}
		\fC
		\item son contexte est sauvegardé (espace mémoire, registres
		    du processeur, etc.)
	    \end {itemize}
	\item qu'un processus est mis sur le processeur
	    \begin {itemize}
		\fC
		\item son contexte est restauré
	    \end {itemize}
    \end {itemize}
\end {frame}

\begin {frame} {Attributs d'un processus}
    Un processus possède des attributs (visibles ou non) :

    \begin {itemize}
	\fB
	\item état (prêt à tourner, en attente, etc.)
	\item identificateur de processus (pid), de processus parent (ppid)
	\item propriétaire (uid), groupes (tableau de gid), uid effectif, etc.
	\item référence au fichier en cours d'exécution
	\item ouvertures de fichiers
	\item répertoire courant
	\item terminal de contrôle
	\item localisation en mémoire
	\item consommation de temps CPU
	\item information d'ordonnancement, priorité (nice)
	\item actions associées aux signaux, signaux reçus, masque
	\item limites (cf \code {getrlimit})
	\item etc.
    \end {itemize}

\end {frame}


%%%%%%%%%%%%%%%%%%%%%%%%%%%%%%%%%%%%%%%%%%%%%%%%%%%%%%%%%%%%%%%%%%%%%%%%%%%%%%
% Mémoire d'un processus
%%%%%%%%%%%%%%%%%%%%%%%%%%%%%%%%%%%%%%%%%%%%%%%%%%%%%%%%%%%%%%%%%%%%%%%%%%%%%%

\titreB {Mémoire d'un processus}

\begin {frame} {Espace mémoire d'un processus}
    L'espace mémoire d'un processus est découpé en 3 zones :

    \vspace* {2mm}

    \begin {minipage} [c] {.29\linewidth}
	\includegraphics [width=\linewidth] {\inc/mem-1}
    \end {minipage}
    \hfill
    \begin {minipage} [c] {.69\linewidth}
	\begin {itemize}
	    \fB
	    \item segment « text »
		\begin {itemize}
		    \fC
		    \item programme (code compilé)
		    \item adresse 0 pas utilisée : pourquoi ?
		\end {itemize}
	    \item segment « data »
		\begin {itemize}
		    \fC
		    \item variables globales (+ \code {static} locales)
		    \item tas (mémoire allouée par \code {malloc})
		    \item extension explicite (via \code {malloc})
		\end {itemize}
	    \item segment « stack » : la pile d'exécution
		\begin {itemize}
		    \fC
		    \item variables locales
		    \item arguments des fonctions
		    \item adresses de retour
		    \item extension implicite (utilisation de la pile)
		\end {itemize}
	    \item d'autres zones peuvent être ajoutées
		\begin {itemize}
		    \fC
		    \item bibliothèques partagées
		    \item mémoire partagée entre processus
		    \item \implique cf semestre 5
		\end {itemize}
	\end {itemize}
    \end {minipage}

\end {frame}

\begin {frame} {Espace mémoire d'un processus}
    Anatomie d'un fichier exécutable :

    \vspace* {3mm}

    \begin {minipage} [c] {.29\linewidth}
	\includegraphics [width=\linewidth] {\inc/mem-2}
    \end {minipage}
    \hfill
    \begin {minipage} [c] {.69\linewidth}
	\begin {itemize}
	    \fC
	    \item en-tête
		\begin {itemize}
		    \fD
		    \item nombre magique
		    \item description des différentes parties
		    \item taille du bss (= données non initialisées)
		\end {itemize}
	    \item table des symboles :
		\begin {itemize}
		    \fD
		    \item adresse de chaque symbole global
		    \item noms et emplacements de l'utilisation des symboles
			« non résolus »
		\end {itemize}
	    \item code binaire : code compilé
	    \item données initialisées : initialisation des variables globales
		\begin {itemize}
		    \fD
		    \item ex : \code {int x = 5 ;}
		\end {itemize}
	    \item informations de debug :
		\begin {itemize}
		    \fD
		    \item  associations <fichier, numéro de ligne, adresse dans le code>
		\end {itemize}
	\end {itemize}
    \end {minipage}
\end {frame}

\begin {frame} {Espace mémoire d'un processus}
    Initialisation de l'espace mémoire à partir du fichier
    exécutable :

    \begin {center}
	\includegraphics [width=.8\linewidth] {\inc/mem-3}
    \end {center}

    Note : le segment « data » est initialisé à partir du fichier,
    le reste du segment (i.e. la taille du bss) est initialisé à 0.

\end {frame}

%%%%%%%%%%%%%%%%%%%%%%%%%%%%%%%%%%%%%%%%%%%%%%%%%%%%%%%%%%%%%%%%%%%%%%%%%%%%%%
% Ordonnancement
%%%%%%%%%%%%%%%%%%%%%%%%%%%%%%%%%%%%%%%%%%%%%%%%%%%%%%%%%%%%%%%%%%%%%%%%%%%%%%

\titreB {Commutation et ordonnancement}

\begin {frame} {Commutation et ordonnancement}
    Deux problèmes distincts :
    \begin {enumerate}
	\item comment réaliser la commutation entre deux processus ?
	    \\
	    \implique commutation
	\item comment choisir le processus à faire tourner ?
	    \\
	    \implique ordonnancement
    \end {enumerate}
\end {frame}

\begin {frame} {Rappel : états d'un processus}
    \begin {center}
	\includegraphics [width=.8\linewidth] {\inc/etats}
    \end {center}
\end {frame}

\begin {frame} {Commutation de processus}
    Commutation = changer le processus courant par un autre

    \begin {itemize}
	\item variable « processus courant »
	    \begin {itemize}
		\item pointe sur le descripteur d'un processus
		\item référence donc indirectement :
		    \begin {itemize}
			\item le pid, les temps CPU, etc.
			\item l'espace mémoire du processus
			\item la sauvegarde des registres du processus
		    \end {itemize}
	    \end {itemize}
	\item lorsqu'un événement (interruption/exception) survient :
	    \begin {itemize}
		\item les registres du processus sont sauvegardés
		\item restauration lors de la fin d'événement
	    \end {itemize}
    \end {itemize}

    La commutation est donc simplement le changement de la variable
    « processus courant »
\end {frame}

\begin {frame} {Commutation de processus}
    \begin {center}
	\includegraphics [width=.8\linewidth] {\inc/ps-commut}
    \end {center}
\end {frame}

\begin {frame} {Ordonnancement}
    Ordonnancement = choix du processus à « mettre sur le processeur »

    \begin {itemize}
	\item Rappel : prérequis matériel = interruption d'horloge
	    \begin {itemize}
		\item système « préemptif »
	    \end {itemize}

	\item Sans interruption d'horloge : système « non préemptif »

	    \begin {itemize}
		\item le noyau ne peut pas reprendre le contrôle
		    périodiquement

		\item les différents processus doivent coopérer pour
		    se «~donner~» le processeur

		\item si un processus ne coopère pas (ex: bug, pirate) \\
		    \implique 
		    pas de solution

		\item processus fou \implique bouton « reset »

		\item exemples : Windows $\leq$ 3.11, MacOS $\leq$ 9.x

	    \end {itemize}
    \end {itemize}

\end {frame}

\begin {frame} {Ordonnancement -- Critères}
    Choisir le meilleur processus suivant quel critère ?
    \begin {center}
	\includegraphics [width=.7\linewidth] {\inc/crit}
    \end {center}

    Critères parfois contradictoires \implique compromis
\end {frame}

\begin {frame} {Ordonnancement -- Critères}
    \begin {itemize}
	\item maximiser l'utilisation des ressources physiques :
	    \begin {itemize}
		\item processeur, périphériques \\
		\item un bon système est un système utilisé, pas un
		    système oisif
		\item \implique ne pas jeter l'argent par les fenêtres
	    \end {itemize}
	\item augmenter le débit 
	    \begin {itemize}
		\item nombre de processus terminés par unité de temps
	    \end {itemize}
	\item augmenter l'affinité
	    \begin {itemize}
		\item conserver le processus sur le même processeur
		    pour bénéficier du cache
	    \end {itemize}
	\item diminuer le temps d'attente
	    \begin {itemize}
		\item remettre le processus sur le processeur dès la fin
		    d'une E/S
	    \end {itemize}
	\item minimiser la durée jusqu'à la terminaison
	    \begin {itemize}
		\item conserver le processeur le plus longtemps possible
	    \end {itemize}
	\item améliorer le temps de réponse interactif
	    \begin {itemize}
		\item privilégier les processus attendant le plus
	    \end {itemize}
	\item diminuer la latence
	    \begin {itemize}
		\item ... entre un événement (exemple : signal) et
		    la remise du processus sur le processeur
		\item prérequis pour les systèmes temps-réel
	    \end {itemize}
    \end {itemize}
\end {frame}

\begin {frame} {Ordonnancement -- Algorithmes}
    Algorithmes classiques d'ordonnancement :

    \begin {itemize}
	\fB
	\item FCFS (First Come, First Served) : premier arrivé, premier servi

	    \begin {itemize}
		\fC
		\item peu compatible avec un système préemptif
	    \end {itemize}

	\item SJF (Shortest Job First) : le plus court en premier

	    \begin {itemize}
		\fC
		\item algorithme optimal...
		    \begin {itemize}
			\fD
			\item minimise le temps d'attente de l'ensemble
			    des processus
		    \end {itemize}
		\item ... mais impraticable sans connaître le futur !
	    \end {itemize}

	\item RR (Round Robin) : tourniquet

	    \begin {itemize}
		\fC
		\item le plus simple
	    \end {itemize}

	\item RT (Real Time) : temps réel

	    \begin {itemize}
		\fC
		\item distingue deux (ou plus) catégories de processus :
		    \begin {itemize}
			\fD
			\item les processus « temps réel » : à ordonnancer
			    en priorité
			\item les processus « normaux » : passent après
			    les processus temps réel
		    \end {itemize}
	    \end {itemize}

    \end {itemize}
\end {frame}

\begin {frame} {Ordonnancement -- Exemple}
    Algorithme d'Unix (initial) = approximation de SJF

    \begin {itemize}
	\item chaque processus a un compteur : utilisation CPU
	\item à chaque interruption d'horloge, le compteur
	    du processus \textbf {courant} est incrémenté
	\item chaque seconde, le compteur de \textbf {tous} les processus
	    est divisé par 2
	\item priorité = compteur + valeur de \code {nice}
	    \begin {itemize}
		\item \code {int nice (int increment)}
	    \end {itemize}
	\item choix du processus : priorité minimum
	\item décision : à chaque retour en mode « utilisateur »
	    \begin {itemize}
		\item à chaque retour de primitive système
		\item à chaque fin d'interruption (horloge ou autre)
	    \end {itemize}
    \end {itemize}
\end {frame}

\begin {frame} {Ordonnancement -- Exemple}
    \begin {center}
	\includegraphics [width=\linewidth] {\inc/decay}
    \end {center}

    \begin {itemize}
	\fC
	\item plus un processus consomme, moins il est prioritaire
	\item division par 2 \implique historique de consommation
	    disparaît peu à peu

	\item privilégie les processus interactifs
	    \begin {itemize}
		\fD
		\item c'est-à-dire ceux qui ont récemment consommé
		    le moins de CPU
	    \end {itemize}
    \end {itemize}
\end {frame}

\begin {frame} {Ordonnancement -- Exemple}
    Algorithme d'Unix (initial) = approximation de SJF

    \begin {itemize}
	\item consommation future de CPU = prédiction basée sur
	    l'historique de consommation
	    \begin {itemize}
		\item historique récent
		\item atténuation rapide (fonction de décroissance
		    exponentielle)

	    \end {itemize}
	\item consommation future $h_{n+1}$ estimée toute les secondes
	    à partir de :
	    \begin {itemize}
		\item la consommation actuelle $c_n$ (augmentation
		    du compteur)
		\item l'historique récent $h_n$ (l'ancienne valeur
		    du compteur)
	    \end {itemize}
	\item prédiction : $h_{n+1} = \alpha c_n + (1-\alpha) h_n$,
	    avec $\alpha = 1/2$
	    \begin {itemize}
		\item si $\alpha = 1$, pas de prise en compte de l'historique
		\item si $\alpha = 0$, pas d'intérêt
		\item si $\alpha = 1/2$, historique décroit exponentiellement
		    \begin {itemize}
			\item $h_{n+1} = \frac {1} {2} c_n
					+ \frac {1} {4} c_{n-1}
					+ \frac {1} {8} c_{n-2}
					\ldots
					+ \frac {1} {2^{n}} c_0
					+ \frac {1} {2^{n}} h_0$
		    \end {itemize}
	    \end {itemize}
	\item sélection du processus tel que $h_{n+1}$ soit le minimum

    \end {itemize}
\end {frame}
