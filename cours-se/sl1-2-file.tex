\def\inc{inc1-2-file}

\titreA {Gestion des fichiers}

%%%%%%%%%%%%%%%%%%%%%%%%%%%%%%%%%%%%%%%%%%%%%%%%%%%%%%%%%%%%%%%%%%%%%%%%%%%%%%
% Accès aux fichiers
%%%%%%%%%%%%%%%%%%%%%%%%%%%%%%%%%%%%%%%%%%%%%%%%%%%%%%%%%%%%%%%%%%%%%%%%%%%%%%

\titreB {Accès aux fichiers}

\begin {frame} {Accès aux fichiers}
    Un fichier :
    \begin {itemize}
	\item a un nom (en fait, plusieurs...)
	\item est accessible via un chemin (absolu, relatif)
	\item possède des attributs :
	    \begin {itemize}
		\item type
		\item propriétaire, groupe, permissions
		\item taille
		\item dates
		    \begin {itemize}
			\item de dernière modification des données
			\item date de dernière modification des attributs
			\item date de dernier accès
		    \end {itemize}
		\item emplacement des données sur le disque
	    \end {itemize}
	\item 2 types de fichiers
	    \begin {itemize}
		\item fichiers « réguliers »
		\item répertoires
		\item ... en réalité, il y en a d'autres (plus tard...)
	    \end {itemize}
    \end {itemize}
\end {frame}

\begin {frame} {Accès aux fichiers}
    Un fichier a une structure simple : suite linéaire d'octets
    \begin {center}
	\includegraphics [width=.6\linewidth] {\inc/str-fich}
    \end {center}

    \begin {itemize}
	\item innovation d'Unix
	    \begin {itemize}
		\item dans les systèmes antérieurs : les fichiers avaient
		    un type (texte, base de données, etc.)
		\item pas de type \implique simplification du noyau
	    \end {itemize}

	\item la structure dépend de l'application qui accède au fichier

	    \begin {itemize}
		\item texte : suite de caractères séparés par l'octet
		    de code 10
		\item binaire exécutable : contient un en-tête qui
		    décrit les différentes parties (code, données,
		    infos de debug, etc.)
		\item document LibreOffice : cf application LibreOffice
		\item etc.
	    \end {itemize}

	\item notion de « \textit {magic number} » :
	    \begin {itemize}
		\item suite d'octets au début d'un fichier pour l'identifier
		\item ex: \code {\#!} (script), \code {\%PDF} (fichier PDF),
		    \code {0xffd8} (JPEG), etc.
		\item commande \code {file}
	    \end {itemize}
    \end {itemize}
\end {frame}

\begin {frame} {Accès aux fichiers}

    \begin {itemize}
	\item nom de fichier : aucune signification pour le noyau
	    \begin {itemize}
		\item je peux appeler mon exécutable \code {toto.titi.tata}
		    si j'en ai envie
		\item je peux appeler un fichier texte \code {toto.xls}
		    si j'en ai envie
		    \vspace* {0.6mm}
		\item certaines applications attendent un suffixe
		    \begin {itemize}
			\item ex : l'application « compilateur C »
			    suppose que les sources C finissent par «
			    \code {.c} »
			\item ce n'est pas le cas général
		    \end {itemize}
	    \end {itemize}
    \end {itemize}
\end {frame}

\begin {frame} {Ouverture de fichier}

    \prototype {
	\code {int open (const char *path, int flags)} \\
	\code {int open (const char *path, int flags, mode\_t mode)}
    }

    \begin {itemize}
	\item 2 formes pour cette primitive (exception qui confirme...)
	    \begin {itemize}
		\item première forme : ouverture «~simple~»
		\item deuxième forme : ouverture avec création du fichier
		    \\
		    \implique \code {mode} est la permission initiale du fichier
		    (avec application du masque de création, voir
		    plus tard)
	    \end {itemize}

	\item résultat : descripteur d'ouverture (ou -1)

	    \begin {itemize}
		\item utilisé par les autres primitives
	    \end {itemize}
    \end {itemize}
\end {frame}

\begin {frame} {Ouverture de fichier}
    \begin {itemize}
	\item \code {flags} : mode d'ouverture
	    \begin {center}
		\vspace* {-5mm}
		\includegraphics [width=.5\linewidth] {\inc/flags-open}

		{\fD Position des bits purement imaginaire...}
	    \end {center}

	\item Exemples~:

	    \begin {itemize}
		\item \code {open ("toto", O\_RDONLY)}

		    Ouverture en lecture seule (le fichier doit exister)

		\item \code {open ("toto", O\_WRONLY | O\_CREAT | O\_TRUNC, 0666)}

		    Création d'un nouveau fichier (ou remise à 0 d'un
		    fichier existant) et ouverture en écriture seule

		\item \code {open ("toto", O\_RDWR | O\_CREAT | O\_APPEND, 0666)}

		    Création d'un nouveau fichier (s'il n'existait
		    pas déjà) et ouverture en lecture/écriture avec
		    ajout à la fin

	    \end {itemize}
    \end {itemize}
\end {frame}

\begin {frame} {Ouverture de fichier}
    \begin {itemize}
	\item Quelles permissions mettre lors d'une création ?
	    \begin {itemize}
		\item règle de base : \code {mode = 0666}

		    \begin {itemize}
			\item si pas de contrainte de sécurité particulière
			\item si pas exécutable (sauf si vous écriviez
			    un éditeur de liens)
		    \end {itemize}
		\item laisser faire le masque de création de fichiers
		    (plus tard)
	    \end {itemize}

	\item Attention : si \code {mode} non fourni, \code {open} prendra
	    ce qu'il y a sur la pile à l'endroit attendu \\
	    \implique vraisemblablement n'importe quoi

    \end {itemize}
\end {frame}

\begin {frame} {Ouverture de fichier}
    Trois ouvertures par défaut :
    \ctableau {} {|l|l|l|} {
	\rc 0 & entrée standard \\
	\rc 1 & sortie standard \\
	\rc 2 & sortie d'erreur standard \\
    }

    \vspace* {3mm}
    Ces descripteurs sont ouverts par le Shell
    \begin {itemize}
	\item Par défaut : le terminal
	\item Redirection possible depuis/vers un fichier
	    \\
	    \implique ne jamais supposer que l'entrée standard est
	    forcément le clavier (ou la sortie standard l'écran)
    \end {itemize}
\end {frame}

\begin {frame} {Fermeture de fichier}
    Ne pas oublier de fermer les fichiers après utilisation
    \prototype {
	\code {int close (int fd)}
    }

    \begin {itemize}
	\item fermeture automatique à la terminaison du processus
	\item bonne pratique : fermer dès que possible
	\item plus tard (tubes) : fermer dès que possible est \textbf
	    {crucial}...
	\item autant s'habituer à le faire dès maintenant !

    \end {itemize}
\end {frame}

\begin {frame} {Accès au fichier}
    Une fois le fichier ouvert, on peut lire et écrire~:

    \prototype {
	\code {ssize\_t read (int fd, void *buf, size\_t nb)} \\
	\code {ssize\_t write (int fd, const void *buf, size\_t nb)}
    }

    \begin {itemize}
	\item retourne le nombre d'octets transférés (ou -1)
	    \begin {itemize}
		\item 0 en fin de fichier
	    \end {itemize}
	\item \code {fd} : descripteur d'ouverture (retourné par \code {open})
	\item \code {buf} : emplacement où le noyau écrit (pour \code {read})
	    ou lit (pour \code {write}) les données à transférer
	\item \code {nb} : nombre d'octets à transférer
	    \begin {itemize}
		\item nb d'octets transférés : pas forcément celui demandé
		\item exemple : \code {read (fd, buf, 500)} alors
		    que le fichier ne fait que 10 octets
		    \implique retour = 10
		\item exemple : \code {write (fd, buf, 500)} alors
		    que le disque est à 10 octets de la saturation
		    \implique retour = 10
	    \end {itemize}
    \end {itemize}

\end {frame}

\begin {frame} {Accès au fichier}
    Accès aléatoire~:
    \begin {center}
	\fB
	\code {off\_t lseek (int fd, off\_t offset, int apartir)}
    \end {center}

    \begin {itemize}
	\item Chaque ouverture de fichier possède un \textit {offset} \\
	    \begin {itemize}
		\item position courante dans le fichier ($\geq$ 0)
		\item avancée automatiquement avec \code {read} et
		    \code {write}

	    \end {itemize}

	\item \code {lseek} permet de modifier l'offset en fonction de \code {apartir}

	    \ctableau {\fC} {|l|l|} {
		\rc \code {SEEK\_SET}
		    & déplacer à la position absolue \\
		\rc \code {SEEK\_CUR}
		    & avancer à partir de la position actuelle \\
		\rc \code {SEEK\_END}
		    & avancer à partir de la fin du fichier \\
	    }

	\item code de retour de \code {lseek} : offset avant modification

    \end {itemize}
\end {frame}

\begin {frame} {Accès au fichier}
    Exemples~:
    \begin {itemize}
	\item \code {lseek (fd, 1000000, SEEK\_SET)}

	    Se déplacer à l'offset = 1 million d'octets

	\item \code {lseek (fd, -20, SEEK\_CUR)}

	    Revenir en arrière de 20 octets avant la position
	    actuelle

	\item \code {lseek (fd, 300000, SEEK\_END)}

	    Se déplacer 300$\thinspace$000 octets après la fin
	    actuelle du fichier
	    \begin {itemize}
		\item \code {read} renverra alors 0
		\item \code {write} pourra écrire de nouveaux octets.
		    Dans ce cas :
		    \begin {itemize}
			\item le système laisse un « trou » dans
			    le fichier
			\item en cas de lecture dans le « trou », tout
			    se passe comme si on avait écrit
			    300$\thinspace$000 fois l'octet 0
			\item la taille du fichier n'est pas la place
			    occupée sur le disque !

		    \end {itemize}
	    \end {itemize}

	\item \code {lseek (fd, 0, SEEK\_CUR)}

	    Où suis-je ?

    \end {itemize}
\end {frame}

%%%%%%%%%%%%%%%%%%%%%%%%%%%%%%%%%%%%%%%%%%%%%%%%%%%%%%%%%%%%%%%%%%%%%%%%%%%%%%
% Primitives systèmes et fonctions de bibliothèque
%%%%%%%%%%%%%%%%%%%%%%%%%%%%%%%%%%%%%%%%%%%%%%%%%%%%%%%%%%%%%%%%%%%%%%%%%%%%%%

\titreB {Primitives systèmes et fonctions de bibliothèque}

\begin {frame} {Primitives et fonctions de bibliothèque}
    Deux séries de fonctions pour accéder aux fichiers ?

    \begin {itemize}
	\item primitive systèmes : \code {open}, \code {close},
	    \code {read}, \code {write}, \code {close}, \code {lseek}

	\item fonctions de bibliothèque : \code {fopen}, \code {fclose},
	    \code {getc}, \code {scanf}, \code {fread}, \code {putc},
	    \code {printf}, \code {fwrite}, \code {fseek}, etc.

    \end {itemize}

    \vspace* {3mm}

    Duplication de fonctionnalités ?

    \vspace* {3mm}

    Pourquoi ?
\end {frame}

\begin {frame} {Exemple avec primitives systèmes}
    \lstinputlisting [basicstyle=\fD\lstmonstyle] {\inc/lib-open.c}
\end {frame}

\begin {frame} {Exemple avec fonctions de bibliothèque}
    \lstinputlisting [basicstyle=\fD\lstmonstyle] {\inc/lib-fopen.c}
\end {frame}

\begin {frame} {Primitives et fonctions de bibliothèque}
    Jeu des différences

    \ctableau {\fD} {|p{.45\linewidth}|p{.45\linewidth}|} {
	\rc \multicolumn {1} {|c|} {\textbf {primitives systèmes}}
	    & \multicolumn {1} {c|} {\textbf {fonctions de bibliothèque}}
	    \\ \hline
	\rc descripteur = \code {int} & descripteur = \code {FILE *}
	    \\
	\rc paramètres de \code {read}/\code {write} moins simples
	    & utilisation de \code {getc}/\code {putc} simple
	    \\
	\rc uniquement \code {read} ou \code {write} &
	    possibilité d'utilisation d'autres fonctions
	    (\code {printf}, \code {puts}, etc.)
	    \\
	\rc but = sécuriser les données
	    & but = aide à la programmation
	    \\
	\rc interface de plus bas niveau
	    & les fonctions de bibliothèque utilisent les
		primitives système (\implique haut niveau)
	    \\
	\rc code dans le noyau
	    & code ajouté au programme lors de l'édition de liens
	    \\
    }
\end {frame}

\begin {frame} {Efficacité}
    Qu'est-ce qui est le plus efficace ?

    \begin {itemize}
	\item approche naïve : les fonctions de bibliothèque appelant
	    les primitives systèmes « équivalentes », elles sont plus
	    lentes
	\item la réponse est plus complexe...
    \end {itemize}
\end {frame}

\begin {frame} {Efficacité}

    Rappel : déroulement d'une primitive système (exemple \code {read})

    \begin {itemize}
	\item instruction spéciale (TRAP, SVC, INT, etc. suivant le
	    processeur)
	\item provoque (entre autres) :
	    \begin {itemize}
		\item basculement en mode privilégié
		\item déroutement du programme vers une adresse spécifique
	    \end {itemize}
	\item vérifications
	    \begin {itemize}
		\item le descripteur d'ouverture est-il ouvert ?
		\item l'adresse du buffer est-elle valide ?
		\item l'adresse de la fin du buffer est-elle valide ?
		\item le nombre à transférer est-il valide ?
	    \end {itemize}
	\item faire l'entrée/sortie (logique)
	\item recopier les données dans l'espace mémoire du processus
	\item revenir à l'instruction suivant \code {read}
    \end {itemize}
    Bilan : beaucoup de vérifications, \textbf {surtout pour un seul octet}
\end {frame}

\begin {frame} {Bufferisation}

    Les fonctions d'entrées/sorties de la bibliothèque font de la
    «~\textbf {bufferisation}~»

    \begin {minipage} [c] {.38\linewidth}
    \begin {center}
	\vspace* {2mm}
	\includegraphics [width=\linewidth] {\inc/bufferisation}
    \end {center}
    \end {minipage}
    \hfill
    \begin {minipage} [c] {.60\linewidth}
	\begin {itemize}
	    \item buffer = tableau en mémoire
	    \item premier appel à \code {getc} : remplissage du buffer
		\\
		\implique appel à \code {read} \implique vérifications
	    \item après : simple appel de fonction + lecture en mémoire
		\\
		\implique très efficace
	    \item buffer de taille $n$
		\implique appel à \code {read} une fois sur $n$

	    \item en pratique, $n = 4096$ (p. ex.)

	\end {itemize}
    \end {minipage}
\end {frame}

\begin {frame} {Bufferisation}

    Bilan :
    \begin {itemize}
	\item si lecture de peu d'octets, les fonctions d'entrées/sorties
	    de la bibliothèque sont plus efficaces que les primitives
	    systèmes

	    \begin {itemize}
		\item moins de vérifications
		\item davantage d'appels de fonctions simples
	    \end {itemize}

	\item si lecture de beaucoup d'octets à la fois, les primitives
	    systèmes sont plus efficaces que les fonctions de
	    bibliothèque

	    \begin {itemize}
		\item pas d'overhead dû à une surcouche
		\item pas de temps passé pour une bufferisation superflue
	    \end {itemize}
    \end {itemize}

    \vspace* {3mm}

    À partir de quelle taille de lecture les primitives systèmes
    sont-elles moins efficaces que \code {getc} ? \implique exercice

\end {frame}


%%%%%%%%%%%%%%%%%%%%%%%%%%%%%%%%%%%%%%%%%%%%%%%%%%%%%%%%%%%%%%%%%%%%%%%%%%%%%%
% Attributs des fichiers
%%%%%%%%%%%%%%%%%%%%%%%%%%%%%%%%%%%%%%%%%%%%%%%%%%%%%%%%%%%%%%%%%%%%%%%%%%%%%%

\titreB {Attributs des fichiers}

\begin {frame} {Attributs des fichiers}
    À chaque fichier sont associés des attributs~:

    \begin {itemize}
	\item type
	\item propriétaire et groupe
	\item permissions
	\item taille
	\item dates
	    \begin {itemize}
		\item de dernière modification des données
		\item de dernière modification des attributs
		\item de dernière accès
	    \end {itemize}
	\item nombre de liens (voir plus tard)
	\item numéro de périphérique, numéro de fichier
	\item emplacement des données sur le disque
    \end {itemize}
    \vspace* {2mm}
    Le nom n'est pas un attribut du fichier (voir plus tard)
\end {frame}

\begin {frame} {Permissions}
    Permissions : 12 bits

    \begin {center}
	\includegraphics [width=.6\linewidth] {\inc/perm}
    \end {center}

    \begin {itemize}
	\item 9 bits habituels (3 bits pour propriétaire, groupe, autres)
	\item bit « sticky » : pour les répertoires, interdit la suppression
	    des fichiers qui s'y trouvent, sauf pour le propriétaire du
	    fichier ou du répertoire
	    (utile pour \code {/tmp} par exemple)
	\item bits « set-user-id-on-exec~» et «~set-group-id-on-exec~» \\
	    \implique plus tard (chapitre sur la gestion des processus)
    \end {itemize}
\end {frame}

\begin {frame} {Consultation des attributs}
    Récupération de tous les attributs en une seule opération :

    \prototype {
	\code {int stat (const char *path, struct stat *stbuf)} \\
	\code {int fstat (int fd, struct stat *stbuf)}
    }

    \begin {itemize}
	\item la structure \code {stat} contient, en retour, l'ensemble
	    des attributs, parmi lesquels :

	    \ctableau {\fD} {|l|l|} {
		\rc \code {st\_mode}
		    & type et permissions \\
		\rc \code {st\_uid}
		    & propriétaire \\
		\rc \code {st\_gid}
		    & groupe \\
		\rc \code {st\_size}
		    & taille en octets \\
		\rc \code {st\_atime}
		    & date de dernier accès \\
		\rc \code {st\_mtime}
		    & date de dernière modification des données \\
		\rc \code {st\_ctime}
		    & date de dernière modification des attributs \\
		\hline
	    }
	\item d'autres attributs sont dans cette structure
	\item ... mais pas tous (localisation des données sur le disque
	    inutile hors du noyau \implique pas remontée)

    \end {itemize}
\end {frame}

\begin {frame} {Consultation des attributs}
    \code {st\_mode} : type et permissions dans le même champ ?

    \begin {center}
	\includegraphics [width=.6\linewidth] {\inc/st-mode}
    \end {center}

    \begin {itemize}
	\item 12 bits de permissions (ex: \code {00752} = \code {rwxr-x-w-})
	    \begin {itemize}
		\item POSIX définit des constantes : \\
		    \code {S\_IRWXU | S\_IRGRP | S\_IXGRP | S\_IWOTH}
		\item ... mais POSIX définit aussi les valeurs numériques
		    \\
		    (très rare... elles sont plus pratiques à utiliser)
	    \end {itemize}
	\item 4 bits : type

	    \begin {minipage} [c] {.40\linewidth}
		\ctableau {\fD} {|l|l|} {
		    \rc 1 & répertoire \\
		    \rc 2 & fichier régulier \\
		    \rc ... & ... \\
		}
	    \end {minipage}
	    \hfill
	    \begin {minipage} [c] {.58\linewidth}
		Exemple
		(valeurs imaginaires)
	    \end {minipage}
    \end {itemize}
\end {frame}

\begin {frame} {Consultation des attributs}
    Comment utiliser \code {st\_mode} ?

    \begin {itemize}
	\item Récupérer les permissions : \code {stbuf.st\_mode \& 0777}
	\item Récupérer le type : \code {stbuf.st\_mode \& 0xf000}
	    \begin {itemize}
		\item pour tester...
		    {
		    \fC
		    \begin {tabular} {ll}
			... si répertoire
			    & \code {if ((stbuf.st\_mode \& 0xf000) == 0x1000)}
			    \\
			... si fichier
			    & \code {if ((stbuf.st\_mode \& 0xf000) == 0x2000)}
			    \\
		    \end {tabular}
		    }
		\item Utilisation des constantes POSIX
		    {
		    \fC
		    \begin {tabular} {ll}
			... si répertoire
			    & \code {if ((stbuf.st\_mode \& S\_IFMT) == S\_IFDIR)}
			    \\
			... si fichier
			    & \code {if ((stbuf.st\_mode \& S\_IFMT) == S\_IFREG)}
			    \\
		    \end {tabular}
		    }
		\item Encore mieux...
		    {
		    \fC
		    \begin {tabular} {ll}
			... si répertoire
			    & \code {if (S\_ISDIR (stbuf.st\_mode))}
			    \\
			... si fichier
			    & \code {if (S\_ISREG (stbuf.st\_mode))}
			    \\
		    \end {tabular}
		    }
	    \end {itemize}
    \end {itemize}
\end {frame}

\begin {frame} {Consultation des attributs}
    Comment répondre à « puis-je lire/écrire/exécuter le fichier » ?

    \begin {itemize}
	\item \code {stat} ne permet pas de répondre à la question
	    \begin {itemize}
		\item tester les permissions ne suffit pas
		\item il faut d'abord savoir si on est le propriétaire,
		    un membre du groupe, ou un « autre »
	    \end {itemize}

	\item solution :

	    \prototype {\code {int access (const char *path, int mode)}}

	\item le paramètre \code {mode} vaut :

	    \ctableau {\fC} {|l|l|} {
		\rc \code {F\_OK} & teste si le fichier existe \\
		\rc \code {X\_OK} & teste l'accès en exécution \\
		\rc \code {W\_OK} & teste l'accès en écriture \\
		\rc \code {R\_OK} & teste l'accès en lecture \\
	    }

	\item accès autorisé : retour = 0, interdit : retour = 1

    \end {itemize}
\end {frame}

\begin {frame} {Modification des attributs}
    Pour modifier certains attributs :
    \begin {itemize}
	\item modifier les permissions
	    \prototype {
		\code {int chmod (const char *path, mode\_t mode)} \\
		\code {int fchmod (int fd, mode\_t mode)}
	    }

	\item modifier le propriétaire ou le groupe
	    \prototype {
		\code {int chown (const char *path, uid\_t uid)} \\
		\code {int fchown (int fd, uid\_t uid)} \\
		\code {int chgrp (const char *path, gid\_t gid)} \\
		\code {int fchgrp (int fd, gid\_t gid)} \\
	    }
	    \vspace* {-4mm}
	    \implique primitives restreintes à l'administrateur
    \end {itemize}
\end {frame}

\begin {frame} {Modification des attributs}
    \begin {itemize}
	\item modifier les dates :
	    \prototype {
		\code {int utime (const char *path, const struct utimbuf *buf)}
	    }

	    \begin {itemize}
		\item Champs de \code {struct utimbuf} :

		    \ctableau {\fD} {|l|l|} {
			\rc \code {actime} & dernier accès \\
			\rc \code {modtime} & dernière modification \\
		    }

		\item pas de troisième date (modification des
		    attributs)
		    \\
		    \implique modifiée par \code {utime} elle-même
	    \end {itemize}
    \end {itemize}
\end {frame}

%%%%%%%%%%%%%%%%%%%%%%%%%%%%%%%%%%%%%%%%%%%%%%%%%%%%%%%%%%%%%%%%%%%%%%%%%%%%%%
% Répertoires
%%%%%%%%%%%%%%%%%%%%%%%%%%%%%%%%%%%%%%%%%%%%%%%%%%%%%%%%%%%%%%%%%%%%%%%%%%%%%%

\titreB {Répertoires}

\begin {frame} {Qu'est-ce qu'un répertoire ?}

    Un répertoire est un fichier sur le disque
    \begin {itemize}
	\item type particulier : ce n'est pas un fichier régulier

	    \begin {itemize}
		\item certaines opérations sont interdites
		    \\
		    Exemple : \code {\alert {open ("repertoire", O\_WRONLY)}}
		    \vspace {0.5mm}

		\item \implique utilisation de primitives spécialisées

	    \end {itemize}

	\item structure particulière

	    \begin {itemize}
		\item c'est toujours une suite linéaire d'octets
		\item mais le noyau y met sa propre structure

	    \end {itemize}

	\item contenu : des références à des fichiers (réguliers
	    ou répertoires ou ...)

    \end {itemize}
\end {frame}

\begin {frame} {Qu'est-ce qu'un répertoire}

    Structure originelle des répertoires Unix (V7, 1977)~:

    \begin {center}
	\includegraphics [width=.6\linewidth] {\inc/rep-fmt-v7}
    \end {center}

    \begin {itemize}
	\item nom : limité à 14 caractères
	\item numéro d'inode : numéro de fichier sur le disque
	\item 2 entrées spéciales : « . » et « .. »
    \end {itemize}
\end {frame}

\begin {frame} {Numéro d'inode}
    Qu'est-ce qu'un numéro de fichier ?

    \begin {itemize}
	\item chaque fichier (régulier, répertoire) a un \textbf {inode}
	\item l'inode rassemble les attributs d'un fichier \\
	    ($\approx$ ce que retourne \code {stat} + localisation des
	    données)
	\item inodes rangés séquentiellement sur une portion
	    du disque \\
	    \implique un inode est donc repéré par son numéro
	    \begin {itemize}
		\item convention : inode du répertoire racine = 2
	    \end {itemize}
	\item un numéro d'inode identifie donc un fichier \\
	    \implique attributs et contenu

    \end {itemize}
\end {frame}

\begin {frame} {Qu'est-ce qu'un répertoire}
    \begin {center}
	\includegraphics [width=.9\linewidth] {\inc/arbo}
    \end {center}

    \begin {itemize}
	\item vision logique : arborescence
	\item vision physique : série d'inodes sur le disque \\
	    \implique répertoires et fichiers
	\item le nom ne fait pas partie des attributs de fichier \\
	    \implique un nom n'est qu'une entrée dans un répertoire
	\item \code {ls -i} : affiche les entrées (nom + numéro d'inode)
    \end {itemize}
\end {frame}

\begin {frame} {Qu'est-ce qu'un répertoire}
    Principales opérations
    \begin {itemize}
	\item \code {open ("/usr/include/stdio.h", ...)} \\
	    \begin {itemize}
		\item recherche dans les répertoires
		\item « traversée de répertoires »
		\item nécessite le droit d'\textbf {exécution} sur les
		    répertoires
		\item chemin absolu : commencer à l'inode 2
		\item chemin relatif : commencer à l'inode du répertoire
		    courant

	    \end {itemize}
	\item supprimer une entrée
	    \begin {itemize}
		\item vider l'entrée : numéro d'inode $\leftarrow$ 0
		\item le reste du répertoire n'est pas décalé (performances)
	    \end {itemize}
	\item ajouter une entrée
	    \begin {itemize}
		\item rechercher une entrée disponible
		\item ou agrandir le répertoire si nécessaire
	    \end {itemize}
    \end {itemize}
\end {frame}

\begin {frame} {Qu'est-ce qu'un répertoire}
    1982 : U. Berkeley diffuse BSD 4.2

    \begin {itemize}
	\item amélioration du système de fichiers : performances,
	    fonctionnalités
	\item augmentation de la taille maximum des noms ($\leq$ 255)
    \end {itemize}

    \begin {center}
	\includegraphics [width=.8\linewidth] {\inc/rep-fmt-ffs}
    \end {center}
\end {frame}

\begin {frame} {Lecture d'un répertoire}
    Comment lire le contenu d'un répertoire ?

    \begin {itemize}
	\item originellement : \code {open ("rep", O\_RDONLY)} \\
	    \begin {itemize}
		\item lire 2 octets pour l'inode et 14 pour le nom, etc.
		\item ignorer les numéros d'inode nuls
		\item recommencer jusqu'à la fin
	    \end {itemize}

	\item apparition du système de fichiers BSD
	    \begin {itemize}
		\item changer tous les programmes (ex : \code {ls})
	    \end {itemize}

	\item nouvelle primitive système BSD : \code {\alert {getdirentries}}
	    \begin {itemize}
		\item non normalisée par POSIX
	    \end {itemize}
	
	\item BSD propose de nouvelles fonctions de bibliothèque :
	    \begin {itemize}
		\item \code {opendir}, \code {readdir}, \code {closedir}
		\item normalisées par POSIX
		\item \implique à utiliser !
	    \end {itemize}

	\item nouveaux systèmes de fichiers
	    \begin {itemize}
		\item locaux : lfs, unionfs, zfs, ext$n$fs, etc
		\item réseau : nfs, smbfs, etc.
	    \end {itemize}
	    \implique \code {opendir} \& co : toujours utilisables
    \end {itemize}
\end {frame}

\begin {frame} {Lecture d'un répertoire}
    \prototype {
	\code {DIR *opendir (const char *path)} \\
	\code {struct dirent *readdir (DIR *dp)} \\
	\code {int closedir (DIR *dp)}
    }

    \begin {itemize}
	\item champs de la \code {struct dirent} normalisés par POSIX :

	    \ctableau {\fC} {|l|l|} {
		\rc \code {d\_ino} & numéro d'inode \\
		\rc \code {d\_name} & nom, terminé par un octet nul \\
	    }

	\item attention : il peut y avoir d'autres champs, mais ils
	    ne sont pas normalisés par POSIX \\
	    \implique à ne pas utiliser

	\item \code {readdir} renvoie successivement toutes les entrées
	    \begin {itemize}
		\item y compris «~.~» et «~..~»
		\item adresse renvoyée = variable locale \code {static}
		    de \code {readdir}
		    \\
		    \implique pas besoin de la libérer
	    \end {itemize}

    \end {itemize}
\end {frame}

\begin {frame} {Lecture d'un répertoire}
    Exemple~:

    \lstinputlisting [basicstyle=\fD\lstmonstyle] {\inc/ex-dir.c}
    
\end {frame}

\begin {frame} {Création et suppression de répertoires}
    \prototype {
	\code {int mkdir (const char *path, mode\_t mode)} \\
	\code {int rmdir (const char *path)}
    }

    \begin {itemize}
	\item \code {mkdir}
	    \begin {itemize}
		\item crée automatiquement les deux entrées «~.~» et «~..~»
		\item \code {mode} = permissions du répertoire
		    \begin {itemize}
			\item droit d'exécution : traverser le répertoire
			\item règle de base : \code {mode = 0777}
			    (sauf si conditions spécifiques)
			\item laisser faire le masque de création de fichiers
			    (plus tard)
		    \end {itemize}
	    \end {itemize}

	\item \code {rmdir} ne peut supprimer que les répertoires vides
	    \begin {itemize}
		\item à l'exception de «~.~» et «~..~» (qu'on ne
		    peut pas supprimer)
	    \end {itemize}

	\item commandes \code {mkdir} et \code {rmdir} : simples
	    appels aux primitives systèmes

    \end {itemize}
\end {frame}


%%%%%%%%%%%%%%%%%%%%%%%%%%%%%%%%%%%%%%%%%%%%%%%%%%%%%%%%%%%%%%%%%%%%%%%%%%%%%%
% Liens
%%%%%%%%%%%%%%%%%%%%%%%%%%%%%%%%%%%%%%%%%%%%%%%%%%%%%%%%%%%%%%%%%%%%%%%%%%%%%%

\titreB {Liens}

\begin {frame} {Liens physiques}
    En shell :

    \begin {itemize}
	\item Commande shell \code {ln} : \code {ln toto titi}
	\item \code {int link (const char *old, const char *new)}
	\item \implique ajoute un deuxième nom \code {titi} au fichier {toto}
    \end {itemize}

    \vspace* {3mm}

    Comment ça marche ? Rappel :

    \begin {center}
	\includegraphics [width=.8\linewidth] {\inc/arbo}
    \end {center}

    En réalité, la vision logique est biaisée...

\end {frame}

\begin {frame} {Liens physiques}
    L'arborescence est un graphe orienté G = <S, A> où :
    \begin {itemize}
	\item sommets $\in$ S : inodes (répertoires ou fichiers réguliers)
	\item arcs étiquetés $\in$ A : entrées de répertoires = noms
	    \begin {center}
		\includegraphics [width=.8\linewidth] {\inc/lien-1}
	    \end {center}
	\item si on exclut «~.~» et «~..~» : graphe sans cycle
	\item nl : nombre de liens (\code {st\_nlink} avec \code {stat})

    \end {itemize}
\end {frame}

\begin {frame} {Liens physiques -- Ajout}
    \begin {itemize}
	\item ajouter un arc dans le graphe
	\item \code {link ("/usr/include/unistd.h", "/home/toto")}

	    \begin {center}
		\includegraphics [width=.5\linewidth] {\inc/lien-2}
	    \end {center}

	\item ajouter un nom augmente le nombre de liens
	\item le nouveau nom a exactement le même statut que l'ancien
	    \begin {itemize}
		\item il n'y a pas de nom « principal », les 2 sont équivalents
	    \end {itemize}
    \end {itemize}
\end {frame}

\begin {frame} {Liens physiques -- Suppression}
    \begin {itemize}
	\item supprimer un arc dans le graphe
	\item \code {int unlink (const char *path)}
	\item \implique décrémente le nombre de liens
	\item si le nombre de liens = 0 \\
	    \implique l'espace occupé par le fichier est libéré sur le disque
	\item \code {unlink} est la primitive pour supprimer un fichier
	    \begin {itemize}
		\item ne fonctionne pas pour les répertoires
		    (rappel : \code {rmdir})
	    \end {itemize}
    \end {itemize}
\end {frame}

\begin {frame} {Liens physiques}
    Limitations des liens physiques :
    \begin {itemize}
	\item le nouveau nom doit être sur le même disque
	    \begin {itemize}
		\item une entrée dans un répertoire ne contient
		    qu'un numéro d'inode \\
		    \implique l'inode est cherché sur le même disque
	    \end {itemize}
	\item on ne peut pas créer de lien vers un répertoire
	    \begin {itemize}
		\item sinon, on pourrait créer des cycles dans le graphe
		\item difficile de faire un parcours récursif \\
		    (find, recopier une arborescence, tar, etc.)
	    \end {itemize}
	\item l'ancien nom doit exister
	    \begin {itemize}
		\item pas de lien vers un fichier qui n'existe pas encore
		\item exemple : un raccourci vers un fichier sur un
		    système de fichiers pas encore monté (montage à
		    la demande)

	    \end {itemize}
    \end {itemize}
\end {frame}

\begin {frame} {Liens symboliques}
    1982 : U. Berkeley diffuse BSD 4.2

    \begin {itemize}
	\item liens symboliques
	\item ex: \code {ln \alert {-s} /tmp /home/pda/temp}
	\item \code {int symlink (const char *old, const char *new)}
	\item nouveau type de fichier (en plus de régulier et répertoire)
	    \begin {itemize}
		\item avec \code {stat} : \code {S\_IFLNK} et
		    \code {S\_ISLNK()}
	    \end {itemize}
	\item un lien symbolique contient le nom de l'objet référencé
	    \begin {itemize}
		\item fichier, répertoire, lien symbolique, ou autre
		\item peut même ne pas exister
	    \end {itemize}
    \end {itemize}
\end {frame}

\begin {frame} {Liens symboliques -- Implémentation}
    \begin {itemize}
	\item un lien symbolique contient le nom de l'objet référencé
	\item nouveau type de fichier
	    \begin {center}
		\includegraphics [width=.7\linewidth] {\inc/lien-3}
	    \end {center}

	\item lorsqu'un lien symbolique est trouvé :
	    \begin {itemize}
		\item si nom absolu : on repart de la racine
		\item si nom relatif : on continue à partir du répertoire
		    dans lequel est le lien
	    \end {itemize}
    \end {itemize}
\end {frame}

\begin {frame} {Liens symboliques -- Lecture d'un lien}
    \prototype {\code {ssize\_t readlink (const char *path, char *buf, size\_t taille)}}

    \begin {itemize}
	\item \code {path} est un fichier de type « lien symbolique »
	\item le contenu du lien est placé dans \code {buf}
	\item au plus \code {taille} octets placés dans \code {buf}
	    \begin {itemize}
		\item troncature possible
	    \end {itemize}
	\item renvoie le nombre d'octets placés dans \code {buf} (ou -1)
    \end {itemize}
\end {frame}

\begin {frame} {Liens symboliques -- Retour sur stat}
    Comment se comporte \code {stat} avec les liens symboliques ?

    \begin {itemize}
	\item sur un lien symbolique, \code {stat} lit les attributs du
	    fichier pointé

	    \begin {itemize}
		\item -1 si le lien référence un objet inexistant
	    \end {itemize}

	\item pour savoir si le fichier est un lien symbolique :

	    \code {int lstat (const char *path, struct stat *stbuf)}

	    \begin {itemize}
		\item \code {lstat} lit les attributs, même
		    s'il s'agit d'un lien symbolique

		\item exemple : \code {tar} archive les liens, pas les
		    fichiers pointés

	    \end {itemize}

    \end {itemize}
\end {frame}
