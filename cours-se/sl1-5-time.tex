\def\inc{inc1-5-time}

\titreA {Gestion du temps}

%%%%%%%%%%%%%%%%%%%%%%%%%%%%%%%%%%%%%%%%%%%%%%%%%%%%%%%%%%%%%%%%%%%%%%%%%%%%%%
% Introduction
%%%%%%%%%%%%%%%%%%%%%%%%%%%%%%%%%%%%%%%%%%%%%%%%%%%%%%%%%%%%%%%%%%%%%%%%%%%%%%

\titreB {Introduction}

\begin {frame} {Mesure du temps}
    Comment le noyau mesure le temps ?

    \begin {enumerate}
	\item Obtenir l'heure au démarrage
	\item Compter le temps qui passe
    \end {enumerate}
\end {frame}

\begin {frame} {Mesure du temps}
    Comment obtenir l'heure au démarrage ?
    \begin {itemize}
	\item Solution 1 : lire l'heure sur le périphérique RTC

	    \begin {itemize}
		\item RTC : \textit {Real Time Clock}
		\item horloge matérielle
		\item fonctionne sur batterie si courant coupé
	    \end {itemize}

	\item Solution 2 : demander l'heure au démarrage du noyau

	    \begin {itemize}
		\item solution « historique »
		\item encore aujourd'hui (exemple : Raspberry PI)
	    \end {itemize}

    \end {itemize}
\end {frame}

\begin {frame} {Mesure du temps}
    Comment compter le temps qui passe ?

    \begin {itemize}
	\item Le noyau doit reprendre la main à intervalle régulier
	\item Assistance matérielle indispensable
	    \begin {itemize}
		\item mécanisme d'interruption périodique du processeur
		\item historiquement : fréquence du secteur électrique
		    \begin {itemize}
			\item fréquence très stable (à l'inverse de la tension)
			\item aux États-Unis : 60 Hz \implique 60 interruptions
			    par seconde
			\item en Europe : 50 Hz \implique 50 interruptions
			    par seconde
		    \end {itemize}

		\item actuellement : composant matériel basé sur le quartz
		    \begin {itemize}
			\item fréquence programmable par le noyau
			\item en fonction de la configuration du noyau
			\item entre 50 et 1000 fois par seconde
			    (période entre 1 et 20 ms)
		    \end {itemize}
	    \end {itemize}
	\item À chaque interruption, incrémenter un compteur
	\item Lorsque le compteur atteint la fréquence
	    \begin {itemize}
		\item une seconde s'est écoulée
		\item incrémenter l'heure du système
	    \end {itemize}
    \end {itemize}
\end {frame}

\begin {frame} {Unités de temps}
    Le noyau utilise deux unités de temps :

    \begin {enumerate}
	\item instant précis dans le temps
	\item courte durée (ex : consommation de CPU)
    \end {enumerate}
\end {frame}

\begin {frame} {Unités de temps -- time\_t}

    Instant précis (à la seconde) dans le temps : type \code {time\_t}

    \begin {itemize}
	\item exemple : heure courante, date de fichier, etc.
	\item valeur : nombre de secondes depuis «~The Epoch~»
	    \begin {itemize}
		\item Epoch : premier janvier 1970, 0h0'0", UTC
		\item UTC : temps universel coordonné
		    \\
		    {\fC (avant 1986 : GMT = heure du méridien de
		    Greenwich)}

	    \end {itemize}

	\item l'heure est conservée en UTC
	    \begin {itemize}
		\item indépendamment du fuseau horaire
	    \end {itemize}

	\item \code {time\_t} historiquement sur 32 bits
	    \begin {itemize}
		\item bogue de l'an 2038 (nombre de secondes $\geq 2^{31}$)
		\item solution : passer à 64 bits (ex : Linux $\geq 3.17$)
		\item nombreux formats de fichiers avec des dates
		    sur 32 bits
	    \end {itemize}

	\item valeur de type \code {time\_t}...
	    \begin {itemize}
		\item ... facile à tenir à jour par le noyau
		\item ... pas facile à lire pour un humain
		    \begin {itemize}
			\item ce n'est pas le problème du noyau
		    \end {itemize}
	    \end {itemize}
    \end {itemize}
\end {frame}

\begin {frame} {Unités de temps -- time\_t}

    Conversion d'un \code {time\_t} en une valeur intelligible :
    \begin {itemize}
	\item problème «~intéressant~», car prise en compte :
	    \begin {itemize}
		\item du fuseau horaire
		\item de l'heure d'été et de l'heure d'hiver
		\item des changements de dates des heures d'été et d'hiver
		    \begin {itemize}
			\item ex : pas de changement d'heure avant 1976
			    en France
		    \end {itemize}
		\item des années bissextiles
		\item des secondes intercalaires
		    \begin {itemize}
			\item variations de la vitesse de rotation de la Terre
			\item ex : 1 seconde ajoutée après 23h59'59"
			    le 31/12/2016
		    \end {itemize}
	    \end {itemize}

	\item ... mais ne concerne pas le noyau
	\item \implique à la charge des fonctions de bibliothèque
	    \begin {itemize}
		\item \code {localtime}, \code {asctime}, \code {strftime}, etc.
	    \end {itemize}

    \end {itemize}
\end {frame}

\begin {frame} {Unités de temps -- clock\_t}
    courte durée : type \code {clock\_t}

    \begin {itemize}
	\item exemple : temps CPU consommé par un processus
	\item valeur : nombre de tops d'horloge (ou «~\textit {ticks}~»)
	\item unité dépend de la configuration du noyau
	    \begin {itemize}
		\item POSIX fournit la primitive
		    \code {long sysconf (int paramètre})
		\item \code {paramètre} : paramètre de configuration interrogé
		\item exemple : \code {freq = sysconf (\_SC\_CLK\_TCK)}
		    \\
		    \implique donne le nombre de tops d'horloge par seconde
	    \end {itemize}
	\item mesure de la consommation CPU :
	    \begin {itemize}
		\item à chaque interruption d'horloge, le noyau incrémente
		    le compteur du processus courant
		\item \implique consommation approximative
	    \end {itemize}
    \end {itemize}
\end {frame}

%%%%%%%%%%%%%%%%%%%%%%%%%%%%%%%%%%%%%%%%%%%%%%%%%%%%%%%%%%%%%%%%%%%%%%%%%%%%%%
% Heure courante
%%%%%%%%%%%%%%%%%%%%%%%%%%%%%%%%%%%%%%%%%%%%%%%%%%%%%%%%%%%%%%%%%%%%%%%%%%%%%%

\titreB {Heure courante}

\begin {frame} {Heure courante}
    \prototype {
	\code {time\_t time (time\_t *heure)} \\
	\code {int stime (time\_t *heure)}
    }

    \begin {itemize}
	\item \code {time} récupère l'heure courante
	    \begin {itemize}
		\item comme valeur de retour (ou -1)
		\item et à l'adresse indiquée
	    \end {itemize}
	\item \code {stime} modifie l'heure courante
	    \begin {itemize}
		\item primitive réservée à l'administrateur
	    \end {itemize}
    \end {itemize}

    \vspace* {3mm}

    Note : l'heure courante est parfois appelée «~\textit {wall clock}~» \\
    (i.e. l'heure qu'on peut lire sur l'horloge murale)
\end {frame}

\begin {frame} {Heure courante -- gettimeofday}
    \prototype {
	\code {int gettimeofday (struct timeval *tv, struct timezone *tz)}
    }

    \begin {itemize}
	\item ajout de l'U. de Berkeley
	\item précision accrue
	\item contenu de la \code {struct timeval} :
	    \ctableau {\fD} {|ll|l|} {
		\code {time\_t} & \code {tv\_sec}
		    & nombre de secondes depuis « The Epoch » \\
		\code {suseconds\_t}  & \code {tv\_usec}
		    & nombre de micro-secondes \\
	    }

	    \vspace* {1mm}

	\item attention : ce n'est pas parce qu'il y a un champ dont
	    l'unité est la $\mu$s que la granularité de la mesure
	    du temps est la $\mu$s

	\item \code {time} est maintenant devenue une fonction de
	    bibliothèque qui appelle la primitive \code {gettimeofday}

    \end {itemize}
\end {frame}

\begin {frame} {Heure courante -- gettimeofday}
    Exemple de \code {struct timeval}~:

    \vspace* {3mm}
    
    \begin {minipage} {.35\linewidth}
    \ctableau {\fD} {|l|r|} {
	\code {tv\_sec}
	    & 1$\thinspace$500$\thinspace$000$\thinspace$000
	    \\
	\code {tv\_usec}
	    & 123$\thinspace$456
	    \\
    }
    \end {minipage}
    \hfill
    $\Leftrightarrow$
    \hfill
    \begin {minipage} {.55\linewidth}
	\fC 14/07/2017 à 2h40'00" et 123456 $\mu s$ UTC
    \end {minipage}


    \begin {center}
	\includegraphics [width=.9\linewidth] {\inc/timeval}
    \end {center}
\end {frame}


\begin {frame} {Heure courante}
    Exemple d'utilisation :

    \lstinputlisting [basicstyle=\fD\lstmonstyle] {\inc/ex-lib.c}

    \begin {itemize}
	\item « primitive système » : \code {time}
	\item fonctions de bibliothèque : \code {localtime}, \code
	    {asctime} et \code {strftime}
    \end {itemize}
\end {frame}

%%%%%%%%%%%%%%%%%%%%%%%%%%%%%%%%%%%%%%%%%%%%%%%%%%%%%%%%%%%%%%%%%%%%%%%%%%%%%%
% Temps CPU
%%%%%%%%%%%%%%%%%%%%%%%%%%%%%%%%%%%%%%%%%%%%%%%%%%%%%%%%%%%%%%%%%%%%%%%%%%%%%%

\titreB {Temps CPU}

\begin {frame} {Temps CPU}
    \prototype {
	\code {clock\_t times (struct tms *buf)} \\
    }

    \begin {itemize}
	\item place la consommation CPU à l'adresse pointée par \code {buf}
	    \begin {itemize}
		\item contenu de la \code {struct tms}~:

		    \vspace* {-3mm}

		    \ctableau {\fD} {|l|l|} {
			\code {tms\_utime}
			    & consommation CPU du processus
				en mode utilisateur \\
			\code {tms\_stime}
			    & consommation CPU du processus
				en mode système \\
			\code {tms\_cutime}
			    & consommation CPU cumulée des fils
				en mode utilisateur \\
			\code {tms\_cstime}
			    & consommation CPU cumulée des fils
				en mode système \\
		    }

		    \vspace* {1mm}

		\item tous les champs sont de type \code {clock\_t}
	    \end {itemize}

	\item retourne le temps réellement écoulé depuis un moment
	    arbitraire dans le passé (ou -1)

	    \begin {itemize}
		\item typiquement le démarrage du noyau
		\item pas très utile
		\item peut déborder la taille allouée à un \code {clock\_t}
		\item bref : valeur de retour à ignorer (sauf pour test
		    d'erreur)...
	    \end {itemize}

    \end {itemize}
\end {frame}

\begin {frame} {Temps CPU}
    Récupération de la consommation CPU :

    \begin {center}
	\includegraphics [width=.5\linewidth] {\inc/times}
    \end {center}

    \begin {itemize}
	\item état « zombie » : permet de transmettre 3 informations
	    \begin {itemize}
		\item code de retour (primitive \code {exit})
		\item consommation CPU agrégée en mode utilisateur
		\item consommation CPU agrégée en mode système
	    \end {itemize}
	\item informations transmises lors du \code {wait} par le père
	\item on ne peut pas avoir la consommation d'un fils non terminé
    \end {itemize}

\end {frame}

\begin {frame} {Temps CPU}
    \prototype {
	\code {int getrusage (int qui, struct rusage *res)}
    }

    \begin {itemize}
	\item primitive \code {times} limitée à la consommation CPU
	\item \implique besoin d'une primitive plus générale pour
	    l'ensemble des ressources consommées par un processus
	\item paramètre \code {qui} :
	    \ctableau {\fD} {|l|l|} {
		\code {RUSAGE\_SELF}
		    & consommation du processus lui-même \\
		\code {RUSAGE\_CHILDREN}
		    & consommation cumulée des fils \\
	    }
	    \vspace* {1mm}
	\item exemples (non exhaustifs) de champs de \code {struct rusage} :

	    \ctableau {\fD} {|l|l|} {
		\code {ru\_utime} & CPU en mode utilisateur \\
		\code {ru\_stime} & CPU en mode système \\
		\code {ru\_maxrss} & mémoire maximum utilisée \\
		\code {ru\_inblock} & nombre de lectures disques \\
		\code {ru\_outblock} & nombre d'écritures disques \\
	    }
    \end {itemize}
\end {frame}

%%%%%%%%%%%%%%%%%%%%%%%%%%%%%%%%%%%%%%%%%%%%%%%%%%%%%%%%%%%%%%%%%%%%%%%%%%%%%%
% Dates des fichiers
%%%%%%%%%%%%%%%%%%%%%%%%%%%%%%%%%%%%%%%%%%%%%%%%%%%%%%%%%%%%%%%%%%%%%%%%%%%%%%

\titreB {Dates des fichiers}

\begin {frame} {Dates des fichiers}
    \vspace* {-2mm}
    \prototype {
	\code {int utime (const char *path, const struct utimbuf *ut)}
	\\
	\code {int utimes (const char *path, const struct timeval tv [2])}
    }

    \vspace* {-2mm}

    \begin {itemize}
	\item Rappel des attributs des fichiers :

	    \ctableau {\fD} {|l|l|} {
		\code {st\_mtime}
		    & date de dernière modification des données \\
		\code {st\_ctime}
		    & date de dernière modification de l'inode \\
		\code {st\_atime}
		    & date de dernier accès \\
	    }

	    \vspace* {1mm}

	\item Primitive \code {utime} : modification des dates d'un
	    fichier
	\item Champs de la \code {struct utimbuf} :
	    \ctableau {\fD} {|l|l|} {
		\code {actime} & date de dernier accès \\
		\code {modtime} & date de dernière modification des
		    données \\
	    }

	    \vspace* {1mm}

	\item Pas de dernière modification de l'inode \implique pourquoi ?

	\item Primitive \code {utimes} : idem \code {utime}, mais avec
	    des \code {struct timeval} (comme \code {gettimeofday})
	    \begin {itemize}
		\item \code {tv[0]} : dernier accès
		\item \code {tv[1]} : dernière modification des données
	    \end {itemize}
    \end {itemize}
\end {frame}

%%%%%%%%%%%%%%%%%%%%%%%%%%%%%%%%%%%%%%%%%%%%%%%%%%%%%%%%%%%%%%%%%%%%%%%%%%%%%%
% Alarmes de processus
%%%%%%%%%%%%%%%%%%%%%%%%%%%%%%%%%%%%%%%%%%%%%%%%%%%%%%%%%%%%%%%%%%%%%%%%%%%%%%

\titreB {Alarmes de processus}

\begin {frame} {Alarmes de processus}
    Voir chapitre suivant (signaux)
\end {frame}

%%%%%%%%%%%%%%%%%%%%%%%%%%%%%%%%%%%%%%%%%%%%%%%%%%%%%%%%%%%%%%%%%%%%%%%%%%%%%%
% Précision de la mesure du temps
%%%%%%%%%%%%%%%%%%%%%%%%%%%%%%%%%%%%%%%%%%%%%%%%%%%%%%%%%%%%%%%%%%%%%%%%%%%%%%

\titreB {Précision de la mesure du temps}

\begin {frame} {Précision de la mesure du temps}
    La précision de la mesure de la consommation de temps CPU est
    approximative

    \begin {center}
	\includegraphics [width=.9\linewidth] {\inc/precision}
    \end {center}

    \begin {itemize}
	\item comptabilisation du temps pour un processus lorsque
	    l'horloge interrompt le CPU
	    \begin {itemize}
		\item pas de prise en compte du temps de B avant son E/S
		    \\
		    \implique temps imputé à C
		\item l'interruption disque est traitée alors que le
		    A est sur le CPU \\
		    \implique temps pour traitement d'E/S de B
			imputé à A
	    \end {itemize}

	\item \implique faire plusieurs mesures
    \end {itemize}

\end {frame}
