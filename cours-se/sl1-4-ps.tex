\def\inc{inc1-4-ps}

\titreA {Gestion des processus}

%%%%%%%%%%%%%%%%%%%%%%%%%%%%%%%%%%%%%%%%%%%%%%%%%%%%%%%%%%%%%%%%%%%%%%%%%%%%%%
% Introduction
%%%%%%%%%%%%%%%%%%%%%%%%%%%%%%%%%%%%%%%%%%%%%%%%%%%%%%%%%%%%%%%%%%%%%%%%%%%%%%

\titreB {Introduction}

\begin {frame} {Définition d'un processus (haut niveau)}
    Définition (haut niveau) :

    \begin {quote}
	un processus est une instance d'un programme en cours d'exécution
    \end {quote}

    ... mais pas n'importe quelle exécution :

    \begin {itemize}
	\item j'exécute \code {ls /tmp} : le processus correspond à
	    l'exécution du programme \code {ls} avec les données
	    \code {/tmp}

	\item quelqu'un d'autre exécute \code {ls /tmp} en même
	    temps : ce n'est pas la même exécution, même si c'est le
	    même programme et les mêmes données

	    \begin {itemize}
		\item ce n'est pas le même processus
		\item même si le « quelqu'un d'autre », c'est moi
	    \end {itemize}

    \end {itemize}
\end {frame}

\begin {frame} {Attributs d'un processus}
    Un processus possède des attributs :

    \begin {itemize}
	\fB
	\item état (prêt à tourner, en attente, etc.)
	\item identificateur de processus (pid)
	\item identificateur de processus parent (ppid)
	\item propriétaire (uid), groupe (gid)
	\item ouvertures de fichiers
	\item répertoire courant
	\item terminal de contrôle
	\item localisation en mémoire
	\item consommation de temps CPU
	\item etc.
    \end {itemize}

\end {frame}

\begin {frame} {Définition d'un processus (bas niveau)}
    Définition (bas niveau) :

    \begin {quote}
	un processus est décrit par :
	\begin {itemize}
	    \fB
	    \item un espace mémoire pour le programme et les données
	    \item des attributs
	    \item un contexte matériel
		\begin {itemize}
		    \fC
		    \item registres du processeur
		    \item traduction d'adresses
		\end {itemize}
	\end {itemize}
    \end {quote}

    { \fB (voir \code {task\_struct} sur
	\url {http://www.tldp.org/LDP/tlk/ds/ds.html} )}

    \vspace* {2mm}

    À chaque fois :
    \begin {itemize}
	\fB
	\item qu'un processus est retiré du processeur
	    \begin {itemize}
		\fC
		\item son contexte est sauvegardé (espace mémoire, registres
		    du processeur, etc.)
	    \end {itemize}
	\item qu'un processus est mis sur le processeur
	    \begin {itemize}
		\fC
		\item son contexte est restauré
	    \end {itemize}
    \end {itemize}
\end {frame}

\begin {frame} {Espace mémoire d'un processus}
    L'espace mémoire d'un processus est découpé en 3 zones :

    \begin {minipage} [c] {.29\linewidth}
	\includegraphics [width=\linewidth] {\inc/ps-mem}
    \end {minipage}
    \hfill
    \begin {minipage} [c] {.69\linewidth}
	\begin {itemize}
	    \fB
	    \item segment « text »
		\begin {itemize}
		    \fC
		    \item programme (code compilé)
		    \item adresse 0 pas utilisée : pourquoi ?
		\end {itemize}
	    \item segment « data »
		\begin {itemize}
		    \fC
		    \item variables globales (+ \code {static} locales)
		    \item tas (mémoire allouée par \code {malloc})
		    \item extension explicite (via \code {malloc})
		\end {itemize}
	    \item segment « stack » : la pile d'exécution
		\begin {itemize}
		    \fC
		    \item variables locales
		    \item arguments des fonctions
		    \item adresses de retour
		    \item extension implicite (utilisation de la pile)
		\end {itemize}
	    \item d'autres zones peuvent être ajoutées
		\begin {itemize}
		    \fC
		    \item bibliothèques partagées
		    \item mémoire partagée entre processus
		    \item \implique cf semestre 5
		\end {itemize}
	\end {itemize}
    \end {minipage}

\end {frame}

%%%%%%%%%%%%%%%%%%%%%%%%%%%%%%%%%%%%%%%%%%%%%%%%%%%%%%%%%%%%%%%%%%%%%%%%%%%%%%
% Gestion des attributs
%%%%%%%%%%%%%%%%%%%%%%%%%%%%%%%%%%%%%%%%%%%%%%%%%%%%%%%%%%%%%%%%%%%%%%%%%%%%%%

\titreB {Gestion des attributs}

\begin {frame} {Identité}
    \prototype {
	\code {pid\_t getpid (void)} 
	\hspace* {10mm}
	\code {pid\_t getppid (void)} \\
	\code {uid\_t getuid (void)}
	\hspace* {10mm}
	\code {gid\_t getgid (void)}
    }

    \begin {itemize}
	\item Primitives simples...
	\item Ne renvoient pas -1 en cas d'erreur (pas d'erreur possible)
	\item Exemple :
	    \lstinputlisting [basicstyle=\fD\lstmonstyle] {\inc/getpid.c}
    \end {itemize}
\end {frame}

\begin {frame} {Identité}
    \prototype {
	\code {int setuid (uid\_t uid)}
	\hspace* {10mm}
	\code {int setgid (gid\_t gid)}
    }

    \begin {itemize}
	\item Primitives restreintes à l'administrateur (uid = 0)

	\item Utilisées lors de l'admission sur le système :
	    \begin {itemize}
		\item \code {/bin/login}, \code {sshd} ou équivalent pour
		    X-Window
		\item lancé par le processus numéro 1 ou un de ses
		    descendants
	    \end {itemize}
    \end {itemize}
\end {frame}

\begin {frame} {Identité}

    Algorithme de l'admission sur le système :
    \begin {enumerate}
	\item demander login et mot de passe
	\item chercher l'entrée dans \code {/etc/passwd}
	    \begin {itemize}
		\item lignes de la forme : \\
		    {\fD \code {toto:sel+mot-de-passe-chiffré:uid:gid:nom:répertoire:shell}}
		\item sel et mot de passe chiffré sont mis dans
		    un fichier séparé (\code {/etc/shadow} sur Linux)
	    \end {itemize}
	\item chiffrer le mot de passe (avec le « sel » cité
	    dans l'entrée)
	\item comparer le mot de passe avec l'entrée
	\item si identique
	    \begin {itemize}
		\item générer un nouveau processus
		\item \code {setuid (uid)} / \code {setgid (gid)}
		    dans ce nouveau processus
		\item lancer l'exécution du shell (ou du
		    «~window manager~»)
	    \end {itemize}
    \end {enumerate}
\end {frame}

\begin {frame} {Masque de création de fichiers}
    \prototype {
	\code {mode\_t umask (mode\_t masque)}
    }

    \begin {itemize}
	\item Création de fichier (\code {open (}... \code
	    {O\_CREAT}...\code {)}, \code {mkdir} ou \code {mknod})

	    \begin {itemize}
		\item permissions du fichier créé = mode \code {\& \~{}} umask
		\begin {center}
		    \includegraphics [width=.7\linewidth] {\inc/umask}
		\end {center}
	    \end {itemize}

	\item Recommandation : dans les programmes \implique 0666 ou 0777
	    \begin {itemize}
		\item les programmes sont généraux
		\item laisser l'utilisateur gérer son niveau de confidentialité
		    \begin {itemize}
			\item commande \code {umask} du Shell
		    \end {itemize}
		\item sauf pour les programmes sensibles à la sécurité
	    \end {itemize}

	\item \code {umask} renvoie l'ancien masque (avant modification)

    \end {itemize}
\end {frame}

\begin {frame} {Répertoire courant}
    \prototype {
	\code {int chdir (const char *path)}
    }

    \begin {itemize}
	\item Modifie le répertoire courant du processus
	\item Rappel : le répertoire courant est un attribut du processus
	    \begin {itemize}
		\item changer dans un processus n'affecte pas les autres
		    processus
	    \end {itemize}

	\item Pas de primitive système pour récupérer le nom
	    du répertoire courant

	    \begin {itemize}
		\item c'est une fonction de bibliothèque
		    \prototype {
			\code {char *getcwd (char *buf, size\_t max)}
		    }
		\item comment fonctionne-t'elle ?
	    \end {itemize}

    \end {itemize}
\end {frame}

\begin {frame} {Racine courante}
    \prototype {
	\code {int chroot (const char *path)}
    }

    \begin {itemize}
	\item Modifie la racine courante du processus
		\begin {center}
		    \includegraphics [width=.9\linewidth] {\inc/chroot}
		\end {center}

	\item Pas de possibilité de contournement
	    \begin {itemize}
		\item à la nouvelle racine, «~\code {..}~» reste au
		    même endroit
		\item on ne peut que descendre dans l'arborescence
	    \end {itemize}
	\item Accessible seulement à l'administrateur du système
	\item Primitive non normalisée par POSIX
    \end {itemize}
\end {frame}

\begin {frame} {Racine courante}
    \code {chroot} est un système de confinement :

    \begin {itemize}
	\item restreindre l'environnement à certains fichiers
	    seulement
	    \begin {itemize}
		\item exemple : comptes utilisateurs spécifiques
		    pour accéder à une application seulement
		\item exemple : serveur « FTP anonyme » \\
		    service FTP pour distribuer des fichiers, restreint
		    à une portion de l'arborescence

	    \end {itemize}

	\item \code {chroot} : base des systèmes de « conteneurs » modernes
		\begin {itemize}
		    \item LXC sur Linux, Jails sur FreeBSD
		    \item complété par d'autres systèmes de confinement
			\begin {itemize}
			    \item visibilité restreinte des processus
			    \item visibilité restreinte des connexions réseau
			    \item etc.
			\end {itemize}
		    \item machines « virtuelles » à moindre coût
		\end {itemize}

    \end {itemize}
\end {frame}

%%%%%%%%%%%%%%%%%%%%%%%%%%%%%%%%%%%%%%%%%%%%%%%%%%%%%%%%%%%%%%%%%%%%%%%%%%%%%%
% Création des processus
%%%%%%%%%%%%%%%%%%%%%%%%%%%%%%%%%%%%%%%%%%%%%%%%%%%%%%%%%%%%%%%%%%%%%%%%%%%%%%

\titreB {Création des processus}

\begin {frame} {Création des processus}
    \prototype {
	\code {pid\_t fork (void)}
    }

    Créer un processus $\Longleftrightarrow$ dupliquer le processus

    \begin {itemize}
	\item Primitive \code {fork} = photocopieuse à processus
	\item Processus = mémoire + attributs + contexte matériel
	    \begin {itemize}
		\item (presque) tout est dupliqué
		\item mémoire dupliquée \implique variables dupliquées
		\item les registres du CPU aussi : chaque programme
		    évoluera (registre PC) séparément
	    \end {itemize}
	\item Duplication \implique informations \textbf {héritées}
	    du processus père
	    \begin {itemize}
		\item répertoire courant, ouvertures de fichiers, etc.
		\item sauf quelques attributs~: pid, ppid, temps CPU, etc.
	    \end {itemize}
	\item Après une photocopie, on peut écrire sur chaque feuille
	    de papier, ça ne modifie pas l'autre feuille
	    \begin {itemize}
		\item c'est la même chose avec \code {fork}
	    \end {itemize}
    \end {itemize}
\end {frame}

\begin {frame} {Création des processus}
    \begin {minipage} [c] {.30\linewidth}
	\lstinputlisting [basicstyle=\fE\lstmonstyle] {\inc/fork.c}
    \end {minipage}
    \hfill
    \begin {minipage} [c] {.69\linewidth}
	\includegraphics [width=1.1\linewidth] {\inc/fork}
    \end {minipage}

    \vspace* {3mm}

    Après \code {fork} : les variables \code {x} et \code {pid}
    évoluent séparément
\end {frame}

\begin {frame} {Création des processus}
    \begin {itemize}
	\item Particularité de \code {fork} :
	    \begin {itemize}
		\item un seul appel, deux retours
		    \begin {itemize}
			\item comme \code {fork} duplique le processus,
			    tout se passe pour le deuxième comme s'il
			    avait lui-même appelé \code {fork}
		    \end {itemize}

	    \end {itemize}
	\item Valeur de retour de \code {fork} :
	    \begin {itemize}
		\item -1 en cas d'erreur (classique)
		\item 0 pour le processus fils
		\item une valeur > 0 pour le processus père
		    \begin {itemize}
			\item c'est le pid du fils
		    \end {itemize}
	    \end {itemize}
	\item Pas de déterminisme :
	    \begin {itemize}
		\item après \code {fork}, on ne sait pas si le père ou
		    le fils est remis sur le processeur en premier
		\item pas un problème : les processus évoluent indépendamment
	    \end {itemize}
    \end {itemize}
\end {frame}

\begin {frame} {Création des processus}
    \begin {itemize}
	\item Quel moyen mnémotechnique pour la valeur de retour ?
	    \begin {itemize}
		\item Chaque processus connaît son père
		    \begin {itemize}
			\item attribut ppid, primitive \code {getpid}
			\item information facile à retrouver
			\item \implique pas besoin de récupérer
			    le pid du père avec \code {fork}
			\item \code {fork} renvoie donc 0 pour le fils
		    \end {itemize}

		\item Pas de moyen facile pour récupérer le pid du fils
		    \begin {itemize}
			\item il peut y en avoir beaucoup, lequel renvoyer ?
			\item pas d'attribut, pas de primitive
			\item \implique seul moyen de récupérer le pid du
			    fils = \code {fork}
			\item \code {fork} renvoie au père le pid du fils
		    \end {itemize}
	    \end {itemize}
    \end {itemize}
\end {frame}

\begin {frame} {Création des processus}
    \begin {block} {\casseroler {Conseils pour apprivoiser \code {fork}}}

    \begin {itemize}
	\item \code {switch} avec 3 cas : \code {-1}, \code {0} et
	    \code {default}

	    \begin {itemize}
		\item pour être sûr de n'oublier aucun cas
	    \end {itemize}
	\item dans le cas 0 (fils), faire appel à une fonction \code {fils}
	    et terminer par \code {exit}
	    \begin {itemize}
		\item isoler le code du fils
		\item placer un « cordon sanitaire » afin que le fils
		    n'exécute pas le code prévu pour le père
	    \end {itemize}
    \end {itemize}
    \end {block}
\end {frame}

\begin {frame} {Création des processus}
    Exemple :

    \lstinputlisting [basicstyle=\fC\lstmonstyle] {\inc/ex-fork.c}
\end {frame}

\begin {frame} {Terminaison des processus}
    \prototype {
	\code {void exit (int code)}
    }

    \begin {itemize}
	\item Termine le processus en cours
	\item Pas de retour
	    \begin {itemize}
		\item ... et donc pas -1 en cas d'erreur
	    \end {itemize}
	\item Presque toutes les ressources sont libérées
	    \begin {itemize}
		\item mémoire, CPU, etc.
		\item cas particulier pour les processus zombies (plus tard)
	    \end {itemize}
    \end {itemize}
\end {frame}

\begin {frame} {Terminaison des processus}

    \begin {block} {\casseroler {Argument de \code {exit}}}
    \begin {itemize}
	\item code $\in$ [0..255]
	    \begin {itemize}
		\item si vous voyez \code {\alert {exit (-1)}} dans un
		    programme, c'est que son auteur n'a pas assimilé
		    son cours de système...
	    \end {itemize}
	\item valeur 0 : ok
	    \begin {itemize}
		\item convention utilisée par les shells
		\item si le père n'est pas un shell, on peut
		    utiliser une autre convention
	    \end {itemize}
	\item constantes POSIX : \code {EXIT\_SUCCESS} et \code
	    {EXIT\_FAILURE}

    \end {itemize}
    \end {block}
\end {frame}

\begin {frame} {Terminaison des processus}
    En réalité, \code {exit} est une fonction de bibliothèque
    \begin {itemize}
	\item vide les buffers des fichiers ouverts par \code {fopen}
	\item appelle les fonctions enregistrées par \code {atexit}
	\item la vraie primitive s'appelle \code {\_exit}
	    \begin {itemize}
		\item jamais appelée directement
	    \end {itemize}
    \end {itemize}
\end {frame}

\begin {frame} {Terminaison des processus}
    \prototype {
	\code {pid\_t wait (int *raison)} \\
	\code {pid\_t waitpid (pid\_t pid, int *raison, int options)} \\
    }

    \vspace* {-8mm}

    \begin {itemize}
	\item Attend la terminaison d'un des processus fils
	    \begin {itemize}
		\item \code {wait} attend n'importe quel fils
		    \begin {itemize}
			\item si un processus fils est déjà terminé,
			    pas d'attente
			\item si pas de processus fils, renvoie -1
			\item s'il y a au moins un processus fils, attente
			\item si attente interrompue par un signa, renvoie -1
		    \end {itemize}
		\item \code {waitpid} attend un processus spécifique
		    (voir manuel)
	    \end {itemize}
	\item La terminaison peut avoir plusieurs causes :
	    \begin {itemize}
		\item le fils appelle \code {exit}
		\item le fils reçoit un signal
		    \begin {itemize}
			\item \framebox {\fD CTRL} \framebox {\fD C}, violation
			    de segment, etc.
		    \end {itemize}
		\item ce peut être autre chose qu'une terminaison
		    \begin {itemize}
			\item processus ayant atteint un point d'arrêt avec
			    un débogueur
		    \end {itemize}

	    \end {itemize}
    \end {itemize}
\end {frame}

\begin {frame} {Terminaison des processus}
    \begin {itemize}
	\item Entier pointé par \code {raison} : raison de la terminaison
	    \ctableau {\fE} {|p{.2\linewidth}|p{.2\linewidth}|p{.2\linewidth}|p{.2\linewidth}|} {
		\rca \multicolumn {1}{|c|}{\textbf {}} & 
		    \multicolumn {1}{c|}{\textbf {code retour}} &
		    \multicolumn {1}{c|}{\textbf {poids fort}} &
		    \multicolumn {1}{c|}{\textbf {poids faible}}
		    \\
		\rcb processus stoppé en mode trace &
		    identificateur~du pro\-ces\-sus &
		    numéro du signal &
		    \multicolumn {1}{c|}{0177}
		    \\
		\rca processus terminé par exit &
		    identificateur~du pro\-ces\-sus &
		    argument de exit sur 8 bits &
		    \multicolumn {1}{c|}{0}
		    \\
		\rcb processus terminé par signal &
		    identificateur~du pro\-ces\-sus &
		    \multicolumn {1}{c|}{0} &
		    numéro du signal (+0200 si core)
		    \\
		\rca wait interrompue par signal &
		    \multicolumn {1}{c|}{-1} &
		    \multicolumn {1}{c|}{?} &
		    \multicolumn {1}{c|}{?}
		    \\
	    }
	    \vspace* {3mm}

	\item POSIX simplifie le travail : macros les plus courantes

	    \ctableau {\fC} {|l|l|} {
		\rc Arrêt avec \code {exit} ? & WIFEXITED () \\
		\rc \implique si oui, code de retour & WEXITSTATUS () \\
		\rc Arrêt sur signal ? & WIFSIGNALED () \\
		\rc \implique si oui, numéro du signal & WTERMSIG () \\
	    }
    \end {itemize}
\end {frame}

\begin {frame} {Terminaison des processus}
    Exemple :

    \lstinputlisting [basicstyle=\fD\lstmonstyle] {\inc/ex-wait.c}
\end {frame}

\begin {frame} {Cas particulier -- Processus zombie}
    Définition :

    \begin {quote}
	Un processus zombie est un processus terminé, dont le
	père n'a pas encore enregistré la terminaison avec wait
    \end {quote}

    \begin {itemize}
	\item un processus zombie est « quasiment » terminé
	    \begin {itemize}
		\item presque toutes ses ressources sont libérées...
		\item ... sauf le descripteur du processus
		    \begin {itemize}
			\item il contient la raison de la terminaison
			\item ainsi qu'un résumé de l'utilisation
			    des ressources
		    \end {itemize}
		\item sans limitation de durée
		\item il reste visible avec \code {ps}
	    \end {itemize}


	\item lorsque le père utilise \code {wait} :

	    \begin {itemize}
		\item il collecte les informations nécessaires
		\item le descripteur de processus est libéré
		    \begin {itemize}
			\item le processus disparaît alors complètement
			    du système
		    \end {itemize}

	    \end {itemize}
    \end {itemize}
\end {frame}

\begin {frame} {Cas particulier -- Processus orphelin}
    Que se passe-t'il lorsque le père d'un processus se termine ?

    \vspace* {2mm}

    \begin {minipage} [c] {.24\linewidth}
	\includegraphics [width=\linewidth] {\inc/orphan}
    \end {minipage}
    \hfill
    \begin {minipage} [c] {.75\linewidth}
	\begin {itemize}
	    \item père se termine \implique zombie
	    \item dès que le père devient zombie, le fils est «~reparenté~»
		    (ppid $\leftarrow$ 1)
	    \item le processus 1 est spécial
		\begin {itemize}
		    \item ne s'arrête jamais
		    \item (re-)démarre les programmes du système
		    \item algorithme :
			\lstinputlisting [basicstyle=\fE\lstmonstyle] {\inc/algo-ps1.c}
		\end {itemize}
	\end {itemize}
    \end {minipage}

    \vspace* {2mm}

    \implique le fils est immédiatement reparenté \\
    \implique le statut d'orphelin n'existe donc pas dans le noyau
\end {frame}

\begin {frame} {États d'un processus}
    États d'un processus :
    \begin {center}
	\includegraphics [width=.8\linewidth] {\inc/etats}
    \end {center}
\end {frame}


%%%%%%%%%%%%%%%%%%%%%%%%%%%%%%%%%%%%%%%%%%%%%%%%%%%%%%%%%%%%%%%%%%%%%%%%%%%%%%
% Exécution d'un fichier
%%%%%%%%%%%%%%%%%%%%%%%%%%%%%%%%%%%%%%%%%%%%%%%%%%%%%%%%%%%%%%%%%%%%%%%%%%%%%%

\titreB {Exécution d'un fichier}

\begin {frame} {Exécution d'un fichier}
    \prototype {
	\fC \code {int execl (char *path, char *arg, ... )} \\
	\fC \code {int execv (char *path, char *argv [])} \\
	\fC \code {int execle (char *path, char *arg, ..., char *envp [])} \\
	\fC \code {int execve (char *path, char *argv [], char *envp [])} \\
	\fC \code {int execlp (char *fichier, char *arg0, ...)} \\
	\fC \code {int execvp (char *fichier, char *argv [])}
    }

    \begin {itemize}
	\item Primitives \code {exec*} : remplacent le programme du
	    processus courant par un nouveau programme et ses arguments

	    \begin {itemize}
		\item le contexte du processus reste quasiment inchangé
		    \begin {itemize}
			\item attributs inchangés : pid, ppid, uid
			    (sauf exception), umask, répertoire courant,
			    consommation de ressources, etc.

			\item attributs modifiés : référence à l'exécutable,
			    ouvertures de fichiers (sauf exception, notamment
			    pour 0, 1 et 2), etc.
		    \end {itemize}
		\item mémoire initialisée avec le nouveau programme
		    \begin {itemize}
			\item segments text, data et stack
		    \end {itemize}
	    \end {itemize}
	\item Valeur de retour = -1 (toujours)
	    \begin {itemize}
		\item retour \implique nouveau programme non chargé
		    \implique erreur
	    \end {itemize}
    \end {itemize}
\end {frame}

\begin {frame} {Exécution d'un fichier}
    \begin {center}
	\includegraphics [width=.8\linewidth] {\inc/exec}
    \end {center}
\end {frame}

\begin {frame} {Exécution d'un fichier -- Exemple}
    Pas de retour pour \code {exec*} \implique le plus souvent utilisée
    avec \code {fork}

    \vspace* {3mm}

    \lstinputlisting [basicstyle=\fD\lstmonstyle] {\inc/ex-exec.c}
\end {frame}

\begin {frame} {Exécution d'un fichier -- Exemple}
    Autre exemple : algorithme (grossier) du Shell

    \begin {enumerate}
	\item lire une ligne
	\item découper la ligne en éléments
	\item si éléments [0] $\in$ commandes internes
	    \begin {itemize}
		\item alors exécuter la fonction correspondante
		\item revenir en 1
	    \end {itemize}
	\item localiser le fichier éléments [0] dans \code {PATH}
	\item si pas trouvé, alors erreur et revenir en 1
	\item \code {fork} \implique fils
	    \begin {itemize}
		\item \code {exec (} éléments \code {)}
	    \end {itemize}
	\item si éléments [end] $\neq$ \code {\&}
	    \begin {itemize}
		\item alors attendre la fin du fils
	    \end {itemize}
	\item revenir en 1
    \end {enumerate}
\end {frame}

\begin {frame} {Exécution d'un fichier -- Environnement}
    En plus des arguments, \code {exec*} passe l'environnement :

    \vspace* {4mm}

    \begin {minipage} [c] {.35\linewidth}
	\includegraphics [width=1.2\linewidth] {\inc/env}
    \end {minipage}
    \hfill
    \begin {minipage} [c] {.64\linewidth}
	\begin {itemize}
	    \item \code {extern char **environ}
	    \item fonction de bibliothèque \code {getenv}
	    \item environnement modifié par le Shell
		\begin {itemize}
		    \item hérité par tous les processus lancés
			par le Shell
		    \item rappel : utiliser \code {export} en Shell pour
			exporter une variable d'environnement

		\end {itemize}
	\end {itemize}
    \end {minipage}
\end {frame}

\begin {frame} {Exécution d'un fichier}
    Six formes pour \code {exec*} :

    \begin {itemize}
	\item passage des arguments :
	    \ctableau {\fC} {|l|l|} {
		\rc \code {execl}
		    & en liste : \code {execl (..., "echo", "a, "b", NULL)} \\
		\rc \code {execl}
		    & en vecteur : \code {execv (..., tabargv)} \\
	    }

	    \vspace* {2mm}

	\item recherche dans la variable shell \code {PATH} :
	    \ctableau {\fC} {|l|l|} {
		\rc \code {exec[vl]}
		    & non : \code {execv ("/bin/echo", tabargv)} \\
		\rc \code {exec[vl]p}
		    & oui : \code {execvp ("echo", tabargv)} \\
	    }

	    \vspace* {2mm}

	\item passage de l'environnement :

	    \ctableau {\fC} {|l|l|} {
		\rc \code {exec[vl]}
		    & implicite : \code {execv ("/bin/echo", tabargv)} \\
		\rc \code {exec[vl]e}
		    & explicite : \code {execve ("echo", tabargv, tabenvp)} \\
	    }

    \end {itemize}

    \vspace* {2mm}

    Une seule de ces formes est une primitive système (laquelle ?) \\
    \implique les autres sont des fonctions de bibliothèque
\end {frame}

\begin {frame} {Nombres magiques}
    \code {exec*} peut exécuter plusieurs sortes de fichiers :

    \begin {itemize}
	\item Fichiers binaires (compilés)
	    \begin {itemize}
		\item plusieurs formats possibles
		    \begin {itemize}
			\item évolution des formats,
			    compatibilité avec anciennes versions
			\item sur FreeBSD : mode de compatibilité Linux
		    \end {itemize}
	    \end {itemize}
	\item Fichiers interprétés
	    \begin {itemize}
		\item fichiers non directement exécutables
		    \begin {itemize}
			\item recours à un interprète
			\item Shell, Awk, Perl, Tcl, Ruby, Python, etc.
		    \end {itemize}
		\item l'interprète ouvre le fichier et l'« exécute »
	    \end {itemize}
    \end {itemize}

    \vspace* {1mm}

    Présence d'un « nombre magique » en début de fichier

    \ctableau {\fC} {|l|l|} {
	\rc fichiers binaires
	    & \code {0x7f 'E' 'L' 'F'} (format ELF sous Linux) \\
	\rc fichiers interprétés
	    &  \code {'\#' '!'} (suivi du chemin de l'interprète) \\
    }

\end {frame}

%%%%%%%%%%%%%%%%%%%%%%%%%%%%%%%%%%%%%%%%%%%%%%%%%%%%%%%%%%%%%%%%%%%%%%%%%%%%%%
% Droits d'exécution
%%%%%%%%%%%%%%%%%%%%%%%%%%%%%%%%%%%%%%%%%%%%%%%%%%%%%%%%%%%%%%%%%%%%%%%%%%%%%%

\titreB {Droits d'exécution}

\begin {frame} {Droits d'exécution}
    \begin {center}
	\includegraphics [width=.8\linewidth] {\inc/droits}
    \end {center}

    \begin {itemize}
	\item \code {fork} : le fils hérite de l'uid (celui de tata)
	\item \code {exec} : ne change pas l'uid (celui de tata)
	\item le fichier \code {secret.txt} de l'utilisateur toto
	    ne peut pas être ouvert par un processus appartenant à tata
	\item \implique le système de droits fonctionne bien !

    \end {itemize}
\end {frame}

\begin {frame} {Droits d'exécution -- Élévation de privilège}
    Dans certains cas, il faut pouvoir exécuter un programme avec
    des droits plus élevés :

    \begin {itemize}
	\item un utilisateur change son mot de passe \implique
	    il doit écrire le nouveau mot de passe chiffré dans \code
	    {/etc/passwd}

	    \begin {itemize}
		\fC
		\item \code {/etc/passwd} n'est pas modifiable par
		    l'utilisateur \\
		    \fC
		    \code {\$ ls -l /etc/passwd} \\
		    \code {-rw-r--r-- 1 root 12345 Jan 1  1970 /etc/passwd}

	    \end {itemize}

	\item un utilisateur insère la carte SD de son appareil photo \\
	    \implique
	    le système de fichiers sur la carte doit être « monté »

	    \begin {itemize}
		\fC
		\item l'opération de montage (primitive système \code
		    {mount}) n'est accessible qu'à l'administrateur
	    \end {itemize}

	\item un utilisateur souhaite utiliser la commande \code {ping}

	    \begin {itemize}
		\fC
		\item \code {ping} accède aux couches réseau de bas
		    niveau et nécessite les privilèges de 
		    l'administrateur 
	    \end {itemize}

    \end {itemize}
\end {frame}

\begin {frame} {Droits d'exécution -- Élévation de privilège}
    Retour sur les permissions de fichiers :

    \begin {center}
	\includegraphics [width=.7\linewidth] {\inc/perm}
    \end {center}

    \begin {itemize}
	\item bit «~\textit {set-user-id-on-exec}~» (ou bit «~suid~»)
	\item bit «~\textit {set-group-id-on-exec}~» (ou bit «~sgid~»)
    \end {itemize}
\end {frame}

\begin {frame} {Droits d'exécution -- Bit set-user-id-on-exec}

    \begin {itemize}
	\item Appel à \code {exec} : si le bit «~suid~» est à 1,
	    alors l'uid du processus devient l'uid du propriétaire
	    du fichier

	    \begin {itemize}
		\item autrement dit : commande exécutée avec
		    les droits du propriétaire (de la commande) et non
		    ceux de l'utilisateur

	    \end {itemize}

	\item Exemple :
	    \begin {itemize}
		\item la commande \code {/bin/passwd} appartient à root
		\item dans ses permissions, le bit «~suid~» est à 1
		\item \implique le processus peut donc modifier
		    \code {/etc/passwd}
		\item \implique l'utilisateur peut changer son mot de
		    passe !
	    \end {itemize}
    \end {itemize}
\end {frame}

\begin {frame} {Droits d'exécution -- Bit set-user-id-on-exec}

    \begin {itemize}
	\item Problème 1 : \code {/bin/passwd} doit connaître l'uid de
	    l'utilisateur qui change son mot de passe...
	    \\
	    \implique 2 notions d'uid distinctes :

	    \vspace* {-2mm}

	    \ctableau {\fC} {|l|l|l|} {
		\rc uid réel
		    & l'humain derrière son terminal
		    & \code {uid\_t getuid (void)}
		    \\
		\rc uid effectif
		    & l'uid servant à tester les droits
		    & \code {uid\_t geteuid (void)}
		    \\
	    }
    \end {itemize}
\end {frame}

\begin {frame} {Droits d'exécution -- Bit set-user-id-on-exec}
    \begin {itemize}
	\item Problème 2 : pour certaines opérations, il faut utiliser
	    l'uid réel et non effectif
	    \begin {itemize}
		\item exemple : création de fichier
		    \implique propriétaire = uid effectif
		\item parfois, il repasser temporairement sous l'identité
		    de l'utilisateur réel
		    \begin {itemize} 
			\item exemple : pour créer le fichier sous la bonne identité
		    \end {itemize}

		\item d'où une troisième notion : uid « sauvé »
		    \begin {itemize}
			\item l'uid sauvé permet de sauver l'uid effectif
			    si jamais on le change (avec
			    \code {int seteuid (uid\_t euid)})
			\item permet de passer sous l'identité de
			    l'uid réel, puis de repasser à nouveau
			    sous l'identité privilégiée
		    \end {itemize}
	    \end {itemize}
    \end {itemize}
\end {frame}

\begin {frame} {Droits d'exécution -- Bit set-group-id-on-exec}
    Application des mêmes principes au groupe :

    \begin {itemize}
	\item bit «~set-group-id-on-exec~»
	\item 3 identités
	    \begin {itemize}
		\item gid réel
		\item gid effectif
		\item gid sauvé
	    \end {itemize}
    \end {itemize}

    \vspace* {5mm}
    Note :

    \begin {itemize}
	\item \code {ls} affiche ces bits
	\item exemple :

	    \vspace* {1mm}

	    {\fD
	    \code {\$ ls -l /usr/bin/passwd /usr/bin/crontab} \\
	    \code {-rw\alert {s}r-xr-x 1 root root\ \ \ \  12345 Jan 1  1970 /usr/bin/passwd} \\
	    \code {-rwxr-\alert {s}r-x 1 root crontab 23456 Jan 1  1970 /usr/bin/crontab}
	    }

    \end {itemize}
\end {frame}

\begin {frame} {Droits d'exécution -- Élévation de privilège}
    Bits «~suid~» et «~sgid~» : changent le niveau de privilège

    \begin {itemize}
	\item le plus souvent : pour élever le niveau
	    \begin {itemize}
		\item même si ça peut arriver de le diminuer
	    \end {itemize}

	\item ce sont des problèmes de sécurité potentiels

	    \begin {itemize}
		\item privilèges \implique attention à la programmation !
		\item bien vérifier les droits
		\item pas de « trou » de sécurité \\
		    \implique débordement de tampon, test des primitives,
		    etc.
	    \end {itemize}

	\item limiter le nombre d'exécutables avec ces bits

	    \begin {itemize}
		\item Exemple sur turing.u-strasbg.fr (Ubuntu 14.04) :

		    \ctableau {\fB} {|l|l|} {
			\rc bit « suid » & 32 fichiers \\
			\rc bit « sgid » & 30 fichiers \\
		    }
	    \end {itemize}
    \end {itemize}
\end {frame}


%%%%%%%%%%%%%%%%%%%%%%%%%%%%%%%%%%%%%%%%%%%%%%%%%%%%%%%%%%%%%%%%%%%%%%%%%%%%%%
% Redirections
%%%%%%%%%%%%%%%%%%%%%%%%%%%%%%%%%%%%%%%%%%%%%%%%%%%%%%%%%%%%%%%%%%%%%%%%%%%%%%

\titreB {Redirections et partage d'ouvertures de fichiers}

\begin {frame} {Redirections}
    Le Shell permet de réalise des redirections :

    \begin {itemize}
	\item \code {\$ wc -l < entree > resultat 2> erreurs}
	\item Rappel : 3 ouvertures par défaut :
	    \ctableau {\fC} {|l|l|} {
		\rc 0 & entrée standard \\
		\rc 1 & sortie standard \\
		\rc 2 & sortie d'erreur standard \\
	    }
	    \vspace* {1mm}
	\item Redirection = modification du descripteur 0, 1 ou 2
	\item À faire dans le fils (et pas dans le père), avant \code {exec}
    \end {itemize}
\end {frame}

\begin {frame} {Redirections}
    Exemple : le shell redirige la sortie standard
    \begin {enumerate}
	\item lire une ligne
	\item découper la ligne en éléments
	\item si éléments [0] $\in$ commandes internes
	    \begin {itemize}
		\item alors exécuter la fonction correspondante
		\item revenir en 1
	    \end {itemize}
	\item localiser le fichier éléments [0] dans \code {PATH}
	\item si pas trouvé, alors erreur et revenir en 1
	\item \code {fork} \implique fils
	    \begin {itemize}
		\item \code {\alert {close (1)}}
		\item \code {\alert {open ("toto", O\_WRONLY | O\_CREAT...)}}
		    \implique renvoie 1
		\item \code {exec (} éléments \code {)}
	    \end {itemize}
	\item si éléments [end] $\neq$ \code {\&}
	    \begin {itemize}
		\item alors attendre la fin du fils
	    \end {itemize}
	\item revenir en 1
    \end {enumerate}
\end {frame}

\begin {frame} {Redirections}
    Plusieurs manières de modifier les descripteurs :

    \begin {enumerate}
	\item fermer un descripteur puis ouvrir un nouveau fichier

	    \begin {itemize}
		\item cf exemple précédent
		\item par construction, \code {open} prend le plus
		    petit descripteur disponible
	    \end {itemize}
    \end {enumerate}
\end {frame}

\begin {frame} {Redirections}
    Plusieurs manières de modifier les descripteurs :

    \begin {enumerate}
	\addtocounter {enumi} {1}

	\item primitive \code {int dup (int fd)} : duplique une
	    ouverture de fichier

	    \begin {itemize}
		\item exemple :
		    \lstinputlisting [basicstyle=\fD\lstmonstyle] {\inc/ex-dup.c}
		\item intérêt : ouvrir le fichier dans le père, avant
		    \code {fork}, afin d'éviter de générer un processus
		    en cas d'erreur
	    \end {itemize}

    \end {enumerate}
\end {frame}

\begin {frame} {Redirections}
    Plusieurs manières de modifier les descripteurs :

    \begin {enumerate}
	\addtocounter {enumi} {2}
	\item primitive \code {int dup2 (int oldfd, int newfd)}

	    \begin {itemize}
		\item exemple :
		    \lstinputlisting [basicstyle=\fD\lstmonstyle] {\inc/ex-dup2.c}
		\item intérêt 1 : \code {dup2} ferme le descripteur
		    de destination si nécessaire
		\item intérêt 2 : facilite la sélection
		    du nouveau descripteur
	    \end {itemize}
    \end {enumerate}
\end {frame}

\begin {frame} {Partage d'ouvertures de fichiers -- dup/dup2}

    Structures de données du noyau :
    \begin {minipage} [c] {.65\linewidth}
	\begin {itemize}
	    \item une table globale pour toutes les ouvertures
		de fichiers
	    \item une table par processus pour ses descripteurs
	\end {itemize}
    \end {minipage}
    \hfill
    \begin {minipage} [c] {.34\linewidth}
	\lstinputlisting [basicstyle=\fE\lstmonstyle] {\inc/prtg-dup.c}
    \end {minipage}

    \begin {center}
	\includegraphics [width=.9\linewidth] {\inc/prtg-dup}
    \end {center}
\end {frame}

\begin {frame} {Partage d'ouvertures de fichiers -- dup/dup2}
    Lecture des données dans le fichier :
    \begin {minipage} [c] {.65\linewidth}
	\begin {itemize}
	    \item offset partagé entre fd1 et 5
	    \item offset propre à fd2
	\end {itemize}
    \end {minipage}
    \hfill
    \begin {minipage} [c] {.34\linewidth}
	\lstinputlisting [basicstyle=\fE\lstmonstyle] {\inc/prtg-dup.c}
    \end {minipage}

    \begin {center}
	\includegraphics [width=.8\linewidth] {\inc/prtg-data}
    \end {center}
\end {frame}

\begin {frame} {Partage d'ouvertures de fichiers -- fork}
    Que se passe-t'il si \code {fork} est appelé après une ouverture de
    fichier ?

    \begin {center}
	\includegraphics [width=.7\linewidth] {\inc/prtg-fork}
    \end {center}

    \implique partage de l'ouverture entre les deux processus
\end {frame}
