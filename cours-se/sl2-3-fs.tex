\def\inc{inc2-3-fs}

\titreA {Systèmes de fichiers}

\begin {frame} {Introduction}
    Systèmes de fichiers : abstraction fondamentale du noyau
    \\
    \implique indispensable pour empêcher l'accès direct au disque

    \begin {itemize}
	\item qu'est-ce qu'un système de fichiers ?
	\item qu'est-ce qu'un disque ?
	\item comment sont représentés les fichiers ?
    \end {itemize}
\end {frame}

\begin {frame} {Introduction}
    \begin {center}
	\includegraphics [width=\linewidth] {\inc/pile-0}
    \end {center}
\end {frame}

%%%%%%%%%%%%%%%%%%%%%%%%%%%%%%%%%%%%%%%%%%%%%%%%%%%%%%%%%%%%%%%%%%%%%%%%%%%%%%
% Structure des disques
%%%%%%%%%%%%%%%%%%%%%%%%%%%%%%%%%%%%%%%%%%%%%%%%%%%%%%%%%%%%%%%%%%%%%%%%%%%%%%

\titreB {Structure des disques}

\begin {frame} {Structure des disques}
    \begin {center}
	\includegraphics [width=\linewidth] {\inc/pile-1}
    \end {center}
\end {frame}


\begin {frame} {Structure des disques -- Structure historique}
    Structure originelle des disques sous Unix :

    \vspace* {2mm}

    \begin {minipage} [c] {.29\linewidth}
	\includegraphics [width=\linewidth] {\inc/disque}
    \end {minipage}
    \hfill
    \begin {minipage} [c] {.69\linewidth}
	\begin {itemize}
	    \fB
	    \item boot-block
		\begin {itemize}
		    \fC
		    \item bloc 0 : chargeur primaire
		    \item nécessaire pour le BIOS ou équivalent
		\end {itemize}
	    \item superblock
		\begin {itemize}
		    \fC
		    \item décrit le disque (taille des parties, date
			de dernière utilisation, etc.)
		\end {itemize}
	    \item table des inodes
		\begin {itemize}
		    \fC
		    \item inode = attributs d'un fichier, y compris
			la localisation des blocs du fichier sur le disque
		\end {itemize}
	    \item blocs de données
		\begin {itemize}
		    \fC
		    \item contenu des fichiers et des répertoires
		\end {itemize}
	    \item espace de swap
		\begin {itemize}
		    \fC
		    \item utilisé comme stockage temporaire
		\end {itemize}
	\end {itemize}
    \end {minipage}
\end {frame}

\begin {frame} {Structure des disques -- Structure historique}
    Cette organisation des disques a un format :

    \begin {itemize}
	\item imposé pour ce qui concerne le boot-block

	    \begin {itemize}
		\item le BIOS (ou équivalent : programme en ROM ou Flash)
		    s'attend à trouver le programme de démarrage dans
		    le premier secteur du disque
	    \end {itemize}

	\item libre pour le reste du disque

	    \begin {itemize}
		\item le système d'exploitation organise le disque
		    comme il l'entend
	    \end {itemize}

    \end {itemize}

    \vspace* {2mm}

    Nouveauté Unix :

    \begin {itemize}
	\item les disques sont structurés en secteurs physiques
	    \\
	    (par exemple : 256 ou 512 octets)

	\item Unix n'utilise que la notion de «~bloc~» de 512 octets
	    \\
	    \implique identique, indépendant de la taille des secteurs
    \end {itemize}
\end {frame}

\begin {frame} {Structure des disques -- Partitions}
    Besoin de découper les disques en entités plus petites \\
    \implique partitionnement

    \begin {itemize}
	\item exemple : répertoires « système » et «
	    utilisateurs » sur des partitions différentes
	\item faciliter les sauvegardes
	    \begin {itemize}
		\item partition « système » toutes les semaines
		\item partition « utilisateurs » tous les jours
	    \end {itemize}
	\item contingenter les espaces
	    \begin {itemize}
		\item si un utilisateur sature le disque, le système
		    continue de fonctionner
	    \end {itemize}
	\item paramétrage différent des systèmes de fichiers
    \end {itemize}
\end {frame}

\begin {frame} {Structure des disques -- Partitions BSD}
    U. de Berkeley : « disk-label » (partitions BSD)

    \begin {center}
	\includegraphics [width=.6\linewidth] {\inc/bsdpart}
    \end {center}

    Chaque partition contenant un système de fichiers abrite un
    superbloc, une table des inodes et des blocs de données


    Initialement : table de 8 partitions (maintenant : 16)
\end {frame}

\begin {frame} {Structure des disques -- Partitions DOS}
    Partitionnement DOS

    \begin {itemize}
	\item principe comparable aux partitions BSD...
	    \begin {itemize}
		\item ... mais beaucoup plus bricolé
		\item doit être utilisé si Windows est installé (dual boot)
		\item utilisé également par Linux
	    \end {itemize}


	\item secteur 0 = MBR (\textit {Master Boot Record\/})

	    \ctableau {\fE} {|r|l|} {
		446 octets & code de démarrage \\
		4 $\times$ 16 octets & table de 4 partitions « primaires » \\
		2 octets & nombre magique \\
	    }

	\item partition primaire avec type « Extended » \implique elle
	    contient des partitions secondaires (ou « étendues »)
	    
	    \begin {itemize}
		\item chaque partition secondaire débute par un
		    secteur EBR (\textit {Extended Boot Record\/})

		    \ctableau {\fE} {|r|l|} {
			446 octets & inutilisé \\
			16 octets & description de la partition \\
			16 octets & lien vers la partition secondaire suivante \\
			32 octets & inutilisé \\
			2 octets & nombre magique \\
		    }
	    \end {itemize}

    \end {itemize}
\end {frame}

\begin {frame} {Structure des disques -- Partitions DOS}

    \begin {center}
	\includegraphics [width=\linewidth] {\inc/dospart}
    \end {center}
\end {frame}

\begin {frame} {Structure des disques -- Partitions}
    Certains systèmes permettent des schémas hybrides

    \vspace* {3mm}

    Exemple : FreeBSD

    \begin {itemize}
	\item FreeBSD peut utiliser le partitionnement DOS
	    \begin {itemize}
		\item compatibilité (en dual-boot) avec un système Windows
	    \end {itemize}
	\item à l'intérieur de la partition DOS consacrée à FreeBSD,
	    utilisation des partitions BSD
    \end {itemize}

    FreeBSD peut aussi s'affranchir de la compatibilité avec les
    partitions DOS et utiliser tout le disque avec des partitions BSD.
\end {frame}

\begin {frame} {Structure des disques -- Partitions}
    Les partitions sont gérées par les pilotes de périphériques

    \begin {itemize}
	\item chaque partition correspond à un numéro de périphérique
	    distinct

	    \begin {itemize}
		\item en particulier le numéro de mineur
	    \end {itemize}

	\item exemple~: sur Linux

	    \begin {itemize}
		\item \code {/dev/hda}, \code {/dev/hdb}, etc. : les disques
		\item \code {/dev/hda1} à \code {hda4} : partitions
		    primaires
		\item \code {/dev/hda5} et suivants : partitions
		    secondaires

	    \end {itemize}

	\item exemple~: sur FreeBSD

	    \begin {itemize}
		\item \code {/dev/da0}, \code {/dev/da1}, etc. : les disques
		\item \code {/dev/da0s1} à \code {/dev/da0s4} : partitions
		    DOS
		\item \code {/dev/da0s2e} : partition e (au sens BSD) dans
		    la partition DOS numéro 2 du premier disque

	    \end {itemize}
    \end {itemize}
\end {frame}

\begin {frame} {Structure des disques -- Volumes}
    \begin {itemize}
	\item Historiquement : besoin de subdiviser les disques
	    \\
	    \implique partitions

	\item Évolution des besoins

	    \begin {itemize}
		\item agréger des disques pour créer des « disques
		    virtuels » de plus grande capacité

		\item pallier les défaillances d'un disque \implique
		    systèmes RAID

	    \end {itemize}

	    \implique gestionnaire de volume

	\item Un gestionnaire de volume abstrait la notion de disque
	    physique pour présenter à l'utilisateur la notion de
	    «~volume~»

	    \begin {itemize}
		\item un volume abrite un système de fichiers,
		    un espace de swap ou autre

		\item un volume peut correspondre à une partie d'un
		    disque (anciennes partitions) ou à un ensemble
		    de disques

	    \end {itemize}
    \end {itemize}
\end {frame}

\begin {frame} {Structure des disques -- Volumes}
    \begin {center}
	\includegraphics [width=\linewidth] {\inc/volume}
    \end {center}
\end {frame}

%%%%%%%%%%%%%%%%%%%%%%%%%%%%%%%%%%%%%%%%%%%%%%%%%%%%%%%%%%%%%%%%%%%%%%%%%%%%%%
% Ordonnancement des requêtes
%%%%%%%%%%%%%%%%%%%%%%%%%%%%%%%%%%%%%%%%%%%%%%%%%%%%%%%%%%%%%%%%%%%%%%%%%%%%%%

\titreB {Ordonnancement des requêtes}

\begin {frame} {Ordonnancement des requêtes}
    \begin {center}
	\includegraphics [width=\linewidth] {\inc/pile-2}
    \end {center}
\end {frame}

\begin {frame} {Ordonnancement des requêtes -- Géométrie}
    Rappel de la géométrie des disques durs :

    \begin {itemize}
	\item le disque est décomposé en \textbf {plateaux}
	\item chaque plateau contient des \textbf {pistes} concentriques
	\item les pistes sont découpées en \textbf {secteurs}
	\item l'ensemble des pistes sur une même verticale constitue
	    un \textbf {cylindre}
	\item les plateaux tournent à haute vitesse
	    \begin {itemize}
		\item par exemple : 5$\thinspace$400, 7$\thinspace$200,
		    15$\thinspace$000 tours par minute
	    \end {itemize}

	\item des \textbf {têtes de lecture} accèdent aux données
	\item chaque tête de lecture est attachée à un \textbf {bras}
	\item les bras sont solidaires et tournent de quelques
	    degrés autour d'un \textbf {axe}, pour accéder à tous
	    les cylindres

    \end {itemize}
\end {frame}

\begin {frame} {Ordonnancement des requêtes -- Géométrie}
    \begin {center}
	\includegraphics [width=.8\linewidth] {\inc/geometrie}
    \end {center}
\end {frame}

\begin {frame} {Ordonnancement -- Temps d'accès}

    \begin {itemize}
	\fB
	\item Exemple : soit un disque
	    \begin {itemize}
		\fC
		\item tournant à 7$\thinspace$200 tours par minute
		\item avec 32 secteurs par piste
		\item dont le temps moyen de déplacement du bras est de 9~ms
	    \end {itemize}
	\item le temps moyen de lecture d'un secteur est composé :
	    \begin {enumerate}
		\fC
		\item du temps de déplacement du bras pour atteindre
		    le cylindre~:
		    \begin {itemize}
			\fD
			\item 9~ms
		    \end {itemize}
		\item du temps d'attente de rotation des plateaux pour
		    que le secteur demandé soit positionné face à la tête :
		    \begin {itemize}
			\fD
			\item au maximum : $\frac {60} {7200} = 8.33$ ms
			    \implique
			    en moyenne : $\frac {8,33} {2} = 4.15$ ms
		    \end {itemize}
		\item du temps de lecture du secteur :
		    \begin {itemize}
			\fD
			\item $\frac {8.33} {32} = 0.26$ ms
		    \end {itemize}
		\item des temps de prise en compte de la requête, de
		    transfert, etc.
	    \end {enumerate}
	\item soit, si le déplacement du bras est fait en parallèle
	    de la rotation des plateaux :
	    \begin {itemize}
		\fC
		\item max(4.15, 9) + 0.26 = 9.26 ms
		\item soit $(3 \times 10^9) (9.26 \times 10^{-3})
			\approx 28 \times 10^6$ cycles sur un CPU
			à 3~GHz
	    \end {itemize}
    \end {itemize}
\end {frame}

\begin {frame} {Ordonnancement -- Algorithmes}
    Temps d'accès aux données très important
    \implique optimisation des requêtes en fonction...
    \begin {itemize}
	\fB
	\item de la position actuelle de la tête de lecture
	\item de la prochaine requête à traiter
    \end {itemize}

    \vspace* {2mm}

    Principaux algorithmes d'ordonnancement des requêtes :

    \begin {itemize}
	\fB
	\item FCFS (First Come, First Served) : premier arrivé, premier servi

	    \begin {itemize}
		\fC
		\item traitement des requêtes dans l'ordre d'arrivée
		\item équitable, mais peu optimum
	    \end {itemize}

	\item SSTF (Shortest Seek Time First) : le positionnement le
	    plus rapide en premier

	    \begin {itemize}
		\fC
		\item après chaque requête, sélectionner la requête
		    pour le secteur le plus proche en termes de temps
		    de positionnement
		\item intéressant, mais risque de famine

	    \end {itemize}

	\item LOOK : algorithme de l'ascenseur

	    \begin {itemize}
		\fC
		\item la tête parcourt le disque du début à la fin
		    (et de la fin au début), en traitant les requêtes
		    qui se trouvent sur le parcours
		\item équitable et efficace, notamment en cas de forte charge

	    \end {itemize}

    \end {itemize}
\end {frame}

\begin {frame} {Ordonnancement des requêtes}
    Évolution des supports de stockage~:

    \begin {itemize}
	\item davantage de capacité pour une même surface
	    \begin {itemize}
		\item pistes extérieures : plus
		    de secteurs que pistes intérieures
		    \begin {itemize}
			\item géométrie non régulière
			\item masquage de la géométrie réelle du disque
			\item adressage <cylindre, tête, secteur>
			    $\rightarrow$ <numéro de secteur>
		    \end {itemize}
	    \end {itemize}
	\item introduction d'intelligence dans les disques
	    \begin {itemize}
		\item les disques incluent un processeur et un \textit
		    {firmware}
		\item gestion au niveau du firmware :
		    \begin {itemize}
			\item des secteurs défectueux
			\item de multiples requêtes en parallèle
			    (\textit {tagged queuing})
			\item de la mémoire cache
		    \end {itemize}
	    \end {itemize}
	\item d'où :
	    \begin {itemize}
		\item ordonnancement des requêtes par le firmware
		\item mécanismes invisibles au niveau du noyau
		\item optimisation par le noyau moins sensible,
		    mais toujours utile
	    \end {itemize}
	\end {itemize}
\end {frame}

\begin {frame} {Ordonnancement des requêtes}
    Évolution des supports de stockage~:

    \begin {itemize}
	\item supports magnétiques $\rightarrow$ mémoires non volatiles
	    \begin {itemize}
		\item « disques » SSD (\textit {Solid State Device})
		    \begin {itemize}
			\item mémoire flash
			\item interface sur le bus similaire à celle
			    d'un disque magnétique
		    \end {itemize}
		\item absence de pièce mécanique
		\item temps d'accès : pas fonction de la
			géométrie
		\item rapide : environ 100~op/s (disques)
		    $\rightarrow$ 100~Kop/s (SSD)
		\item nouveaux défis :
		    \begin {itemize}
			\item le goulet d'étranglement se déplace
			\item assez de puissance CPU pour alimenter
			    les SSD ?
			\item réduire les traitements associés à chaque requête
			\item bus et mémoire assez rapides pour alimenter
			    les SSD ?
		    \end {itemize}

	    \end {itemize}
    \end {itemize}
\end {frame}

%%%%%%%%%%%%%%%%%%%%%%%%%%%%%%%%%%%%%%%%%%%%%%%%%%%%%%%%%%%%%%%%%%%%%%%%%%%%%%
% Buffer cache
%%%%%%%%%%%%%%%%%%%%%%%%%%%%%%%%%%%%%%%%%%%%%%%%%%%%%%%%%%%%%%%%%%%%%%%%%%%%%%

\titreB {Buffer cache}

\begin {frame} {Buffer cache}
    \begin {center}
	\includegraphics [width=\linewidth] {\inc/pile-3}
    \end {center}
\end {frame}

\begin {frame} {Buffer cache}
    Mécanisme du «~\textbf {buffer cache}~» : minimiser le nombre d'E/S

    \begin {itemize}
	\item bufferiser en mémoire les blocs du disque
	\item analogue à un « cache » au niveau du noyau
    \end {itemize}

    \begin {center}
	\includegraphics [width=\linewidth] {\inc/buffer}
    \end {center}
\end {frame}

\begin {frame} {Buffer cache}
    Chaque buffer doit comporter des attributs :
    \begin {itemize}
	\item emplacement du bloc
	    \begin {itemize}
		\item numéro de périphérique du disque
		\item numéro du bloc sur le disque
	    \end {itemize}
	\item état : inutilisé, valide, modifié
    \end {itemize}

    \vspace* {3mm}

    De plus, le noyau doit faire deux types d'accès :

    \begin {itemize}
	\item demande d'E/S sur un bloc \implique rechercher si un
	    buffer contient le bloc \code {bn} du périphérique
	    \code {dev}

	\item si le bloc n'est pas trouvé dans un buffer, prendre
	    le buffer le plus ancien (dernière utilisation la plus
	    ancienne)
	    \\
	    \implique politique LRU (\textit {Least Recently Used\/})

	    \begin {itemize}
		\item si ce buffer contient un bloc modifié, l'écrire
		    sur le disque et recommencer avec le buffer suivant
		\item sinon, utiliser ce buffer pour le nouveau bloc
	    \end {itemize}

    \end {itemize}
\end {frame}

\begin {frame} {Buffer cache}
    Deux types d'accès \implique chaque buffer est dans 2 listes

    \begin {itemize}
	\item chaque liste est doublement chaînée \\
	    \implique faciliter la suppression d'un élément aléatoire
	\item une liste pour retrouver le buffer associé à
	    \code {dev} et \code {bn}

	    \begin {itemize}
		\item en fait, multiples listes (fonction de hachage)
		\item ré-utilisation d'un buffer pour un nouveau bloc
		    \begin {itemize}
			\item suppression de l'ancienne liste
			\item ajout dans la nouvelle liste
		    \end {itemize}
	    \end {itemize}

	\item une liste pour gérer la politique LRU

	    \begin {itemize}
		\item en cas d'utilisation d'un buffer (en lecture ou écriture)
		    \begin {itemize}
			\item le buffer est supprimé de la liste LRU
			\item puis il est ré-inséré en fin de liste LRU
		    \end {itemize}
		\item en cas de recherche de buffer libre
		    \begin {itemize}
			\item le premier buffer de la liste LRU est utilisé
			\item s'il est modifié : recopie sur disque
			    et passage au suivant
			\item sinon : réutilisation pour le nouveau bloc
		    \end {itemize}
	    \end {itemize}

    \end {itemize}
\end {frame}

\begin {frame} {Buffer cache}
    \begin {center}
	\includegraphics [width=\linewidth] {\inc/bcache}
    \end {center}

    {\fD Note : flèches pleines = chaînage avant, flèches creuses =
    chaînage arrière}
\end {frame}

\begin {frame} {Buffer cache}
    Synthèse : plusieurs niveaux de cache

    \ctableau {\fD} {|l|c|c|c|} {
	& \textbf {cache CPU} & \textbf {buffer cache} & \textbf {cache disque} \\
	\textbf {géré par}
	    & CPU & noyau & firmware disque \\
	\textbf {cache entre...}
	    & mémoire centrale & disque & secteurs du disque \\
	\textbf {... et}
	    & SRAM CPU & mémoire centrale & mémoire interne disque \\
	\textbf {« invisible »\footnote {\fD la plupart du temps...
	    Par exemple, le noyau doit vider le cache CPU lors d'une
	    commutation de processus, ou bien les processus constatent
	    une différence de performance à l'utilisation du buffer
	    cache, etc.} par}

	    & tout programme & processus & noyau \\
    }
\end {frame}

%%%%%%%%%%%%%%%%%%%%%%%%%%%%%%%%%%%%%%%%%%%%%%%%%%%%%%%%%%%%%%%%%%%%%%%%%%%%%%
% UFS
%%%%%%%%%%%%%%%%%%%%%%%%%%%%%%%%%%%%%%%%%%%%%%%%%%%%%%%%%%%%%%%%%%%%%%%%%%%%%%

\titreB {UFS}

\begin {frame} {UFS}
    \begin {center}
	\includegraphics [width=\linewidth] {\inc/pile-4}
    \end {center}
\end {frame}


\begin {frame} {UFS}
    Comment organiser les données des fichiers sur le disque ?

    \vspace* {3mm}

    Contraintes :

    \begin {itemize}
	\item structure d'arborescence
	\item capacité de stocker des grands fichiers
	\item optimisation de l'accès à un bloc
	\item faciliter la modification d'un fichier
	\item permettre l'accès aléatoire (\code {lseek})
    \end {itemize}

    \vspace* {3mm}

    \implique système de fichiers original d'Unix : UFS
    \\
    (\textit {Unix File System})
\end {frame}

\begin {frame} {UFS -- Principes}
    \begin {minipage} [c] {.29\linewidth}
	\includegraphics [width=\linewidth] {\inc/disque}
    \end {minipage}
    \hfill
    \begin {minipage} [c] {.69\linewidth}
	\begin {itemize}
	    \fB
	    \item taille d'un bloc : 512 octets (originellement)
	    \item table des inodes
		\begin {itemize}
		    \fC
		    \item dimension choisie à l'initialisation
		    \item emplacement = numéro d'inode
		    \item numéros d'inodes spécifiques
			\begin {itemize}
			    \fD
			    \item 0 : invalide (indique une entrée vide
				dans un répertoire par exemple)
			    \item 1 : non utilisé (autrefois : « faux
				» fichier regroupant tous les secteurs
				défectueux)
			    \item 2 : racine du système de fichiers
			\end {itemize}
		    \item chaque inode décrit un fichier
			\begin {itemize}
			    \fD
			    \item ses attributs (cf \code {stat})
			    \item l'emplacement des données du fichier
			\end {itemize}
		\end {itemize}
	    \item blocs de données
		\begin {itemize}
		    \fC
		    \item contenu des fichiers et des répertoires
		    \item blocs libres
		\end {itemize}
	\end {itemize}
    \end {minipage}
\end {frame}

\begin {frame} {UFS -- Principes}

    \begin {itemize}
	\item dans l'inode : tableau de 13 éléments
	\item les 10 premières entrées sont les numéros sur
	    le disque des 10 premiers blocs du fichier
	\item l'entrée 10 donne le numéro d'un bloc, contenant
	    lui-même 128 (512 octets pour un bloc / 4 octets pour un
	    numéro de bloc) numéros de blocs sur le disque
	    \\
	    \implique Ce sont les blocs 11 à 137 du fichier
	\item l'entrée 11 donne le numéro d'un bloc contenant
	    128 numéros de blocs, chacun de ces 128 blocs contenant
	    lui-même 128 numéros de blocs sur le disque
	    \\
	    \implique Blocs 138 à 16521 $(=138+128^2 - 1)$ du fichier
	\item l'entrée 12 correspond à une triple indirection
	    \\
	    \implique Blocs 16522 à 2113673 du fichier
    \end {itemize}
\end {frame}

\begin {frame} {UFS -- Principes}
    \begin {center}
	\includegraphics [width=\linewidth] {\inc/inode}
    \end {center}
\end {frame}

\begin {frame} {UFS -- Bilan}

    \begin {itemize}
	\item accès rapide (direct) aux 10 premiers blocs d'un fichier
	    \\
	    $10 \times 512 = 5120$ octets
	\item possibilité d'avoir des très grands fichiers
	    \\
	    taille max = $(10 + 128 + 128^2 + 128^3) \times 512$ octets
	    \\
	    taille max $\approx (2^7)^3 \times 2^9 = 2^{21+9} = 2^{30}$
		octets = 1 Go

	\item accès via des indirections pour des grands fichiers
	    \begin {itemize}
		\item accès séquentiels : les blocs d'indirection
		    récemment utilisés sont vraisemblablement dans le
		    buffer cache
		    \begin {itemize}
			\item cas standard : pas d'E/S supplémentaire
		    \end {itemize}

		\item accès aléatoire possible
		    \begin {itemize}
			\item au prix des indirections
		    \end {itemize}

	    \end {itemize}
    \end {itemize}
\end {frame}

\begin {frame} {UFS -- Gestion des blocs libres}
    Comment gérer les blocs libres ?

    \begin {itemize}
	\item allouer un nouveau bloc pour agrandir un fichier
	\item libérer les blocs d'un fichier lorsqu'il est supprimé
	    \begin {itemize}
		\item rappel : libération lorsque le dernier lien est supprimé
	    \end {itemize}
    \end {itemize}
\end {frame}

\begin {frame} {UFS -- Gestion des blocs libres}
    Principe : le superbloc contient :

    \begin {itemize}
	\item un tableau \code {free} de 100 numéros de blocs
	\item une position \code {p} dans le tableau ($p \in [0,99]$, $p=0$ ici)
    \end {itemize}

    \begin {center}
	\includegraphics [width=.6\linewidth] {\inc/freelist}
    \end {center}

    Initialement : tous les blocs de données sont des blocs libres
\end {frame}

\begin {frame} {UFS -- Gestion des blocs libres}
    \begin {itemize}
	\item Allocation d'un nouveau bloc
	    \begin {itemize}
		\item $p \leq 98$ : le numéro de bloc \code {free[p++]} est utilisé
		\item $p = 99$ :
		    \begin {itemize}
			\item le numéro de bloc \code {free[p]} est utilisé
			\item tableau \code {free} réinitialisé avec
			    le tableau du bloc \code {free[p]}
			\item \code {p = 0}
		    \end {itemize}
	    \end {itemize}

	\item Libération d'un bloc $b$
	    \begin {itemize}
		\item $p > 0$ : \code {free[-{}-p] = b}
		\item $p = 0$ :
		    \begin {itemize}
			\item tableau \code {free} recopié
			    dans le bloc \code {b}
			\item \code {p = 100}
			\item \code {free [-{}-p] = b}
		    \end {itemize}
	    \end {itemize}

	\item ne pas oublier de sauvegarder le superbloc...

	    \begin {itemize}
		\item recopie périodique (i.e. toutes les 30 secondes)
	    \end {itemize}

    \end {itemize}
\end {frame}

\begin {frame} {UFS -- Gestion des blocs libres}
    Bilan :

    \begin {itemize}
	\item Stratégie simple...
	\item ... mais sensible à la dispersion
	    \begin {itemize}
		\item initialement : blocs bien rangés dans
		    l'ordre dans la liste des blocs libres
		\item premiers fichiers créés : les blocs sont consécutifs
		    \\
		    \implique bons temps d'accès
		\item au bout de plusieurs allocations et
		    désallocations : ordre assez aléatoire
		    \\
		    \implique temps d'accès dégradés
	    \end {itemize}
	\item Temps dégradés : solution simple... mais lourde...
	    \begin {enumerate}
		\item sauvegarder le système de fichiers (sur bande)
		\item réinitialiser le système de fichiers \\
		    \implique recréer une liste de blocs libres ordonnée
		\item restaurer le système de fichiers \\
		    \implique fichiers recréés avec des blocs
			consécutifs
	    \end {enumerate}
    \end {itemize}
\end {frame}

\begin {frame} {UFS -- Gestion des inodes libres}
    Comment allouer ou libérer un inode ?

    \begin {itemize}
	\item Superbloc contient un tableau de numéros d'inodes libres
	    \begin {itemize}
		\item analogue aux blocs libres...
		\item ... mais à un seul niveau
	    \end {itemize}
	\item Allocation :
	    \begin {itemize}
		\item si tableau non vide : prendre un élément
		\item si tableau vide : remplir le tableau par une
		    recherche linéaire d'inodes libres sur le disque
	    \end {itemize}
	\item Libération :
	    \begin {itemize}
		\item si tableau non plein : stocker le numéro dans
		    le tableau
		\item si tableau plein : ne rien faire
	    \end {itemize}
    \end {itemize}
\end {frame}

\begin {frame} {Évolution -- FFS}
    \begin {itemize}
	\item 1983 : évolution du système de fichiers
	    \begin {itemize}
		\item U. de Berkeley (BSD 4.2)
		\item FFS (Fast File System)
	    \end {itemize}
	\item Améliorations :
	    \begin {itemize}
		\item noms de fichiers : 14 $\rightarrow$ 255 caractères
		    \begin {itemize}
			\item cf cours sur les fichiers et répertoires
		    \end {itemize}
		\item introduction des liens symboliques
		    \begin {itemize}
			\item cf cours sur les fichiers et répertoires
		    \end {itemize}
		\item robustesse en cas de crash
		    \begin {itemize}
			\item duplication du superbloc
		    \end {itemize}
		\item meilleures performances
		    \begin {itemize}
			\item mesures : 3~\%
			    $\rightarrow$ 47~\%
			    de la bande passante du disque
		    \end {itemize}
	    \end {itemize}
    \end {itemize}
\end {frame}

\begin {frame} {FFS -- Améliorations des performances}
    \begin {itemize}
	\item augmentation de la taille des blocs
	    \begin {itemize}
		\item passage de 512 octets à 4096 octets
		    \begin {itemize}
			\item pour lire 4096 octets, il fallait 8 accès
			    disques (i.e. $8 \times 10$ ms)
			\item contre 1 seul accès (i.e. $1 \times 10$
			    ms) avec FFS
		    \end {itemize}
		\item effet secondaire : taille maximum des fichiers
		    \begin {itemize}
			\item taille max $
			    \approx (\frac {4096} {4})^3 \times 4096
			    = (2^{10})^3 \times 2^{12}
			    = 2^{42}$ octets = 4 To
		    \end {itemize}
	    \end {itemize}
	\item tenir compte de la géométrie des disques
	    \begin {itemize}
		\item ne plus allouer un bloc au hasard de la liste
		    des blocs libres
		\item mais choisir le « meilleur » emplacement pour
		    un bloc
		    \begin {itemize}
			\item exemple : allouer le bloc $n+1$ d'un
			    fichier à « proximité » (en termes de
			    temps d'accès) du bloc $n$
		    \end {itemize}
		\item nouvelle partie sur le disque : « bitmap des
		    blocs libres »
		    \begin {itemize}
			\item structure compacte (copiée en mémoire)
			\item permet de faire des recherches rapides
		    \end {itemize}
		\item choix \implique avoir de la marge
		    \begin {itemize}
			\item réserve de 10~\% de la place disponible
		    \end {itemize}
	    \end {itemize}
    \end {itemize}
\end {frame}

\begin {frame} {FFS}
    Remarquable : FFS conserve la structure des fichiers du système
    original
    \begin {center}
	\includegraphics [width=.6\linewidth] {\inc/inode}
    \end {center}
\end {frame}

\begin {frame} {FFS}
    Actuellement :

    \begin {itemize}
	\item FFS toujours utilisé
	\item taille de bloc par défaut : 32~Ko (FreeBSD 10.x)
	\item la géométrie est simplifiée :
	    \begin {itemize}
		\item on ne gère plus que des numéros de blocs (et pas
		    cylindres, tête, secteur)
		\item proximité des numéros de blocs : bonne approximation
		    de la proximité en terme de temps d'accès
	    \end {itemize}
	\item Linux : EXT2 $\approx$ FFS
    \end {itemize}
\end {frame}

%%%%%%%%%%%%%%%%%%%%%%%%%%%%%%%%%%%%%%%%%%%%%%%%%%%%%%%%%%%%%%%%%%%%%%%%%%%%%%
% Montage
%%%%%%%%%%%%%%%%%%%%%%%%%%%%%%%%%%%%%%%%%%%%%%%%%%%%%%%%%%%%%%%%%%%%%%%%%%%%%%

\titreB {Montage}

\begin {frame} {Montage}
    Comment utiliser plusieurs disques ?

    \begin {center}
	\includegraphics [width=.8\linewidth] {\inc/mount-0}
    \end {center}
\end {frame}

\begin {frame} {Montage}
    Comment utiliser plusieurs disques ?

    \begin {itemize}
	\item solution 1 : mentionner explicitement les disques dans
	    les chemins

	    \begin {itemize}
		\item exemple : \code {C:/toto/exo1.c}
	    \end {itemize}

	\item solution 2 : réaliser un montage

	    \begin {itemize}
		\item la notion de disque disparaît du chemin
		\item mieux intégré à l'arborescence
	    \end {itemize}
    \end {itemize}
\end {frame}

\begin {frame} {Montage}
    Montage = association entre :
    \begin {itemize}
	\item un répertoire d'un disque existant (déjà monté)
	    \begin {itemize}
		\item répertoire = « point de montage »
		\item tout ce qui était dans le répertoire original
		    devient inaccessible pendant le montage
	    \end {itemize}
	\item la racine d'un nouveau disque
    \end {itemize}

    \begin {center}
	\includegraphics [width=.7\linewidth] {\inc/mount-1}
    \end {center}
\end {frame}

\begin {frame} {Montage -- Implémentation}
    \begin {center}
	\includegraphics [width=.6\linewidth] {\inc/mount-2}
    \end {center}

    Traversée descendante des points de montage :
    \begin {itemize}
	\item chemin contenant \code {/home} :
	    \begin {itemize}
		\item le noyau accède au répertoire \code {/home}
		\item l'inode est marquée comme « point de montage »
		\item le noyau recherche l'inode dans la table des
		    montages
		\item il accède ensuite à la racine de l'autre disque
	    \end {itemize}
    \end {itemize}
\end {frame}

\begin {frame} {Montage -- Implémentation}
    \begin {center}
	\includegraphics [width=.6\linewidth] {\inc/mount-2}
    \end {center}

    Traversée montante des points de montage :
    \begin {itemize}
	\item chemin contenant \code {..} à partir de la racine du disque 2 :
	    \begin {itemize}
		\item le numéro d'inode = 2 (\implique racine du disque)
		\item recherche du périphérique dans la table des montages
		\item le noyau accède ensuite au répertoire \code {home}
		    du disque 1
	    \end {itemize}
    \end {itemize}
\end {frame}

\begin {frame} {Montage -- Utilisation}
    \begin {itemize}
	\item primitives \code {mount} et \code {umount}
	    \begin {itemize}
		\item réservées à l'administrateur
		\item paramètres (originaux) :
		    \begin {itemize}
			\item répertoire existant (point de montage)
			\item numéro de périphérique
			\item flags (lecture seule, etc.)
		    \end {itemize}
	    \end {itemize}
	\item commandes \code {mount} et \code {umount}
	\item éviter les opérations manuelles à chaque démarrage :
	    \begin {itemize}
		\item fichier \code {/etc/fstab}
		\item script de montage lancé par \code {/sbin/init}
	    \end {itemize}
	\item \textit {automontage} « à la demande »
	    \begin {itemize}
		\item utilisation d'un fichier de configuration
		\item démon de surveillance des accès
		    \implique monte à la demande
		\item automontage des nouveaux périphériques
		    \begin {itemize}
			\item exemple : clef USB, téléphone, etc.
		    \end {itemize}
	    \end {itemize}
    \end {itemize}
\end {frame}


%%%%%%%%%%%%%%%%%%%%%%%%%%%%%%%%%%%%%%%%%%%%%%%%%%%%%%%%%%%%%%%%%%%%%%%%%%%%%%
% Virtual File System
%%%%%%%%%%%%%%%%%%%%%%%%%%%%%%%%%%%%%%%%%%%%%%%%%%%%%%%%%%%%%%%%%%%%%%%%%%%%%%

\titreB {Virtual File System}

\begin {frame} {Virtual File System}
    \begin {center}
	\includegraphics [width=\linewidth] {\inc/pile-5}
    \end {center}
\end {frame}

\begin {frame} {Virtual File System}
    1986 : NFS (\textit {Network File System}) de Sun Microsystems

    \begin {itemize}
	\item accès transparent à des fichiers distants
	\item montage de disques situés sur des serveurs
	\item accès à un fichier \implique requête transmise
	    au serveur distant
    \end {itemize}

    \vspace* {3mm}

    Comment faire cohabiter UFS/FFS et NFS ?
    \begin {itemize}
	\item solution : ajouter une couche d'indirection : VFS
	\item avantage : extensible à d'autres systèmes de fichiers
	    \begin {itemize}
		\item exemples : FAT, NTFS, CD, etc.
	    \end {itemize}
    \end {itemize}
\end {frame}

\begin {frame} {Virtual File System -- Principe}
    Couche d'indirection VFS :

    \begin {itemize}
	\item Virtualisation des fonctions
	    \begin {itemize}
		\item associées au système de fichiers lui-même :
		    \ctableau {\fD} {|l|l|} {
			\code {mount}
			    & montage du système de fichiers \\
			\code {umount}
			    & démontage du système de fichiers \\
			\code {root}
			    & obtenir la racine du système de fichiers \\
			\code {statvfs}
			    & lit les attributs du système de fichiers \\
			\code {...}
			    & etc. \\
		    }
		\item associées aux fichiers :
		    \begin {itemize}
			\item \code {open}, \code {close}, \code {rdwr}
			    (fonction commune pour lecture/écriture),
			    \code {getattr} (équivalent \code {stat}),
			    \code {mkdir}, \code {readdir}, etc.
		    \end {itemize}
		\item adresses des fonctions placées dans des structures
		    de description de chaque système de fichiers
		    (FFS, NFS)

	    \end {itemize}
	\item Virtualisation des inodes
	    \begin {itemize}
		\item pour le noyau, la référence à un fichier devient
		    le \textbf {vnode}

		\item le vnode référence à son tour un descripteur spécifique

		    \begin {itemize}
			\item inode pour FFS, nfsnode pour NFS, etc.
		    \end {itemize}
	    \end {itemize}
    \end {itemize}
\end {frame}

\begin {frame} {Virtual File System -- Implémentation}
    Structures de données en mémoire :
    \begin {center}
	\hspace* {-5mm}	% beuark...
	\includegraphics [width=1.1\linewidth] {\inc/vfs}
    \end {center}
\end {frame}
